% Options for packages loaded elsewhere
\PassOptionsToPackage{unicode}{hyperref}
\PassOptionsToPackage{hyphens}{url}
%
\documentclass[
]{book}
\usepackage{amsmath,amssymb}
\usepackage{lmodern}
\usepackage{iftex}
\ifPDFTeX
  \usepackage[T1]{fontenc}
  \usepackage[utf8]{inputenc}
  \usepackage{textcomp} % provide euro and other symbols
\else % if luatex or xetex
  \usepackage{unicode-math}
  \defaultfontfeatures{Scale=MatchLowercase}
  \defaultfontfeatures[\rmfamily]{Ligatures=TeX,Scale=1}
  \setmainfont[]{Arial}
\fi
% Use upquote if available, for straight quotes in verbatim environments
\IfFileExists{upquote.sty}{\usepackage{upquote}}{}
\IfFileExists{microtype.sty}{% use microtype if available
  \usepackage[]{microtype}
  \UseMicrotypeSet[protrusion]{basicmath} % disable protrusion for tt fonts
}{}
\makeatletter
\@ifundefined{KOMAClassName}{% if non-KOMA class
  \IfFileExists{parskip.sty}{%
    \usepackage{parskip}
  }{% else
    \setlength{\parindent}{0pt}
    \setlength{\parskip}{6pt plus 2pt minus 1pt}}
}{% if KOMA class
  \KOMAoptions{parskip=half}}
\makeatother
\usepackage{xcolor}
\usepackage{color}
\usepackage{fancyvrb}
\newcommand{\VerbBar}{|}
\newcommand{\VERB}{\Verb[commandchars=\\\{\}]}
\DefineVerbatimEnvironment{Highlighting}{Verbatim}{commandchars=\\\{\}}
% Add ',fontsize=\small' for more characters per line
\usepackage{framed}
\definecolor{shadecolor}{RGB}{248,248,248}
\newenvironment{Shaded}{\begin{snugshade}}{\end{snugshade}}
\newcommand{\AlertTok}[1]{\textcolor[rgb]{0.94,0.16,0.16}{#1}}
\newcommand{\AnnotationTok}[1]{\textcolor[rgb]{0.56,0.35,0.01}{\textbf{\textit{#1}}}}
\newcommand{\AttributeTok}[1]{\textcolor[rgb]{0.77,0.63,0.00}{#1}}
\newcommand{\BaseNTok}[1]{\textcolor[rgb]{0.00,0.00,0.81}{#1}}
\newcommand{\BuiltInTok}[1]{#1}
\newcommand{\CharTok}[1]{\textcolor[rgb]{0.31,0.60,0.02}{#1}}
\newcommand{\CommentTok}[1]{\textcolor[rgb]{0.56,0.35,0.01}{\textit{#1}}}
\newcommand{\CommentVarTok}[1]{\textcolor[rgb]{0.56,0.35,0.01}{\textbf{\textit{#1}}}}
\newcommand{\ConstantTok}[1]{\textcolor[rgb]{0.00,0.00,0.00}{#1}}
\newcommand{\ControlFlowTok}[1]{\textcolor[rgb]{0.13,0.29,0.53}{\textbf{#1}}}
\newcommand{\DataTypeTok}[1]{\textcolor[rgb]{0.13,0.29,0.53}{#1}}
\newcommand{\DecValTok}[1]{\textcolor[rgb]{0.00,0.00,0.81}{#1}}
\newcommand{\DocumentationTok}[1]{\textcolor[rgb]{0.56,0.35,0.01}{\textbf{\textit{#1}}}}
\newcommand{\ErrorTok}[1]{\textcolor[rgb]{0.64,0.00,0.00}{\textbf{#1}}}
\newcommand{\ExtensionTok}[1]{#1}
\newcommand{\FloatTok}[1]{\textcolor[rgb]{0.00,0.00,0.81}{#1}}
\newcommand{\FunctionTok}[1]{\textcolor[rgb]{0.00,0.00,0.00}{#1}}
\newcommand{\ImportTok}[1]{#1}
\newcommand{\InformationTok}[1]{\textcolor[rgb]{0.56,0.35,0.01}{\textbf{\textit{#1}}}}
\newcommand{\KeywordTok}[1]{\textcolor[rgb]{0.13,0.29,0.53}{\textbf{#1}}}
\newcommand{\NormalTok}[1]{#1}
\newcommand{\OperatorTok}[1]{\textcolor[rgb]{0.81,0.36,0.00}{\textbf{#1}}}
\newcommand{\OtherTok}[1]{\textcolor[rgb]{0.56,0.35,0.01}{#1}}
\newcommand{\PreprocessorTok}[1]{\textcolor[rgb]{0.56,0.35,0.01}{\textit{#1}}}
\newcommand{\RegionMarkerTok}[1]{#1}
\newcommand{\SpecialCharTok}[1]{\textcolor[rgb]{0.00,0.00,0.00}{#1}}
\newcommand{\SpecialStringTok}[1]{\textcolor[rgb]{0.31,0.60,0.02}{#1}}
\newcommand{\StringTok}[1]{\textcolor[rgb]{0.31,0.60,0.02}{#1}}
\newcommand{\VariableTok}[1]{\textcolor[rgb]{0.00,0.00,0.00}{#1}}
\newcommand{\VerbatimStringTok}[1]{\textcolor[rgb]{0.31,0.60,0.02}{#1}}
\newcommand{\WarningTok}[1]{\textcolor[rgb]{0.56,0.35,0.01}{\textbf{\textit{#1}}}}
\usepackage{longtable,booktabs,array}
\usepackage{calc} % for calculating minipage widths
% Correct order of tables after \paragraph or \subparagraph
\usepackage{etoolbox}
\makeatletter
\patchcmd\longtable{\par}{\if@noskipsec\mbox{}\fi\par}{}{}
\makeatother
% Allow footnotes in longtable head/foot
\IfFileExists{footnotehyper.sty}{\usepackage{footnotehyper}}{\usepackage{footnote}}
\makesavenoteenv{longtable}
\usepackage{graphicx}
\makeatletter
\def\maxwidth{\ifdim\Gin@nat@width>\linewidth\linewidth\else\Gin@nat@width\fi}
\def\maxheight{\ifdim\Gin@nat@height>\textheight\textheight\else\Gin@nat@height\fi}
\makeatother
% Scale images if necessary, so that they will not overflow the page
% margins by default, and it is still possible to overwrite the defaults
% using explicit options in \includegraphics[width, height, ...]{}
\setkeys{Gin}{width=\maxwidth,height=\maxheight,keepaspectratio}
% Set default figure placement to htbp
\makeatletter
\def\fps@figure{htbp}
\makeatother
\setlength{\emergencystretch}{3em} % prevent overfull lines
\providecommand{\tightlist}{%
  \setlength{\itemsep}{0pt}\setlength{\parskip}{0pt}}
\setcounter{secnumdepth}{5}
\usepackage{booktabs}
\usepackage{amsthm}
\makeatletter
\def\thm@space@setup{%
  \thm@preskip=8pt plus 2pt minus 4pt
  \thm@postskip=\thm@preskip
}
\makeatother
\ifLuaTeX
  \usepackage{selnolig}  % disable illegal ligatures
\fi
\usepackage[]{natbib}
\bibliographystyle{apalike}
\IfFileExists{bookmark.sty}{\usepackage{bookmark}}{\usepackage{hyperref}}
\IfFileExists{xurl.sty}{\usepackage{xurl}}{} % add URL line breaks if available
\urlstyle{same} % disable monospaced font for URLs
\hypersetup{
  pdftitle={Разработка telegram ботов на языке R},
  pdfauthor={Алексей Селезнёв},
  hidelinks,
  pdfcreator={LaTeX via pandoc}}

\title{Разработка telegram ботов на языке R}
\author{Алексей Селезнёв}
\date{2022-08-09}

\begin{document}
\maketitle

{
\setcounter{tocdepth}{1}
\tableofcontents
}
\hypertarget{ux432ux432ux435ux434ux435ux43dux438ux435}{%
\chapter*{Введение}\label{ux432ux432ux435ux434ux435ux43dux438ux435}}
\addcontentsline{toc}{chapter}{Введение}

\begin{center}\rule{0.5\linewidth}{0.5pt}\end{center}

\hypertarget{ux43fux440ux435ux434ux438ux441ux43bux43eux432ux438ux435}{%
\section*{Предисловие}\label{ux43fux440ux435ux434ux438ux441ux43bux43eux432ux438ux435}}
\addcontentsline{toc}{section}{Предисловие}

Аудитория telegram ежедневно растёт с геометрической прогрессией, этому способствует удобство мессенджера, наличие каналов, чатов, и конечно возможность создавать ботов.

Боты могут использоваться в совершенно разных целях, от автоматизации коммуникации с вашими клиентами до управления вашими собственными задачами.

По сути через бота можно используя telegram выполнять любые операции: отправлять, либо запрашивать данные, запускать задачи на сервере, собирать информацию в базу данных, отправлять электронные письма и так далее.

Этот веб учебник поможет вам освоить процесс разработки telegram ботов используя язык программирования R.

Материал изложен последовательно, от простого к сложному.

Первая глава посвящена отправке сообщений из R в Telegram, а в последней главе мы разработаем полноценного бота, который поддерживает последовательный логический диалог с пользователем.

В ходе всего учебника основным R пакетом который мы будем использовать будет \texttt{telegram.bot}.

Каждая глава заканчивается небольшим заданием и тестом, для того, что бы вы могли проверить насколько хорошо был воспринят материал.

\hypertarget{ux43dux430ux432ux44bux43aux438-ux43dux435ux43eux431ux445ux43eux434ux438ux43cux44bux435-ux434ux43bux44f-ux43fux440ux43eux445ux43eux436ux434ux435ux43dux438ux44f-ux443ux447ux435ux431ux43dux438ux43aux430}{%
\section*{Навыки необходимые для прохождения учебника}\label{ux43dux430ux432ux44bux43aux438-ux43dux435ux43eux431ux445ux43eux434ux438ux43cux44bux435-ux434ux43bux44f-ux43fux440ux43eux445ux43eux436ux434ux435ux43dux438ux44f-ux443ux447ux435ux431ux43dux438ux43aux430}}
\addcontentsline{toc}{section}{Навыки необходимые для прохождения учебника}

Тему построения ботов я отношу к продвинутым навыкам, не зависимо от выбранного язка программирования. Поэтому в этой книге не рассматриваются базовые вводные темы по основам языка R.

Для чтения и понимания книги вам необходимо обладать следующими навыками на языке программирования R:

\begin{itemize}
\tightlist
\item
  Понимать базовые програмные конструкции, т.е. циклы и условные ветвления.
\item
  Понимать что такое функция.
\item
  Разбираться в основных структурах данных языка.
\item
  Уметь работать со строками.
\item
  Владеть основами манипуляции данных с помощью пакета \texttt{dplyr}.
\item
  Иметь поверхностное понимание о том, что такое API.
\end{itemize}

Перечисленные выше темы выходят за рамки этой книги, но подробно рассматриваются в онлайн академии \href{https://r-for-marketing.netpeak.net/}{``Язык R для интернет-маркетинга''}.

\hypertarget{ux43eux431-ux430ux432ux442ux43eux440ux435}{%
\section*{Об авторе}\label{ux43eux431-ux430ux432ux442ux43eux440ux435}}
\addcontentsline{toc}{section}{Об авторе}

Меня зовут Алексей Селезнёв, уже более 10 лет я являюсь практикующим аналитиком. С 2016 года возглавляю отдел аналитики в агентстве интернет - маркетинга Netpeak.

Являюсь автором курсов по языку R: \href{https://needfordata.ru/r}{``Язык R для интернет - маркетинга''} и \href{https://www.youtube.com/playlist?list=PLD2LDq8edf4pgGg16wYMobvIYy_0MI0kF}{``Язык R для пользователей Excel''}.

С 2015 года активно пишу статьи по аналитике, на момент написания этих строк мной опубликовано уже более 120 статей в различных интернет изданиях. Веду собственный \href{https://alexeyseleznev.wordpress.com/}{блог}, хотя он по большей части он является агрегатором моих статей из различных источников.

В 2018 году завёл telegram канал \href{https://t.me/R4marketing}{R4marketing}, в котором делюсь полезными, русскоязычными материалами по языку R: ссылки на статьи, доклады, вебинары, заметки по применению языка R.

В 2020 году запустил \href{https://bit.ly/36kliAp}{YouTube канал}, в котором делюсь видео уроками по языку R И аналитике в целом.

\hypertarget{ux43fux440ux430ux432ux43aux438-ux438-ux43fux440ux435ux434ux43bux43eux436ux435ux43dux438ux44f}{%
\section*{Правки и предложения}\label{ux43fux440ux430ux432ux43aux438-ux438-ux43fux440ux435ux434ux43bux43eux436ux435ux43dux438ux44f}}
\addcontentsline{toc}{section}{Правки и предложения}

Перед публикацией учебника я несколько раз перечитал его, но всё же некоторые помарки в разметке, грамматические или синтаксические ошибки могли ускользнуть от моего внимания.

К тому же, возможно у вас есть идеи о том, какой информации в учебнике нехватает, или о том, что какая часть учебника потеряла свою актуальность.

По таким вопросам прошу писать мне либо на почту, либо напрямую в Telegram.

Email: \href{mailto:selesnow@gmail.com}{\nolinkurl{selesnow@gmail.com}}
Telegram: \href{http://t.me/AlexeySeleznev}{AlexeySeleznev}

\hypertarget{ux43fux43eux434ux434ux435ux440ux436ux430ux442ux44c-ux43fux440ux43eux435ux43aux442}{%
\section*{Поддержать проект}\label{ux43fux43eux434ux434ux435ux440ux436ux430ux442ux44c-ux43fux440ux43eux435ux43aux442}}
\addcontentsline{toc}{section}{Поддержать проект}

Учебник, и все необходимые материалы находятся в открытом доступе, но при желании вы можете поддержать этот проект любой произвольной сумме перейдя по \href{https://secure.wayforpay.com/payment/build_telegram_bot_using_r}{этой ссылке}.

\hypertarget{ux441ux43eux437ux434ux430ux451ux43c-ux431ux43eux442ux430-ux438-ux43eux442ux43fux440ux430ux432ux43bux44fux435ux43c-ux441-ux435ux433ux43e-ux43fux43eux43cux43eux449ux44cux44e-ux441ux43eux43eux431ux449ux435ux43dux438ux44f-ux432-telegram-1}{%
\chapter{Создаём бота, и отправляем с его помощью сообщения в telegram (1)}\label{ux441ux43eux437ux434ux430ux451ux43c-ux431ux43eux442ux430-ux438-ux43eux442ux43fux440ux430ux432ux43bux44fux435ux43c-ux441-ux435ux433ux43e-ux43fux43eux43cux43eux449ux44cux44e-ux441ux43eux43eux431ux449ux435ux43dux438ux44f-ux432-telegram-1}}

В этой главе мы разберёмся как создать телеграм бота, и отправлять с его помощью уведомления в telegram.

\hypertarget{ux441ux43eux437ux434ux430ux43dux438ux435-ux442ux435ux43bux435ux433ux440ux430ux43c-ux431ux43eux442ux430}{%
\section{Создание телеграм бота}\label{ux441ux43eux437ux434ux430ux43dux438ux435-ux442ux435ux43bux435ux433ux440ux430ux43c-ux431ux43eux442ux430}}

Для начала нам необходимо создать бота. Делается это с помощью специального бота \textbf{BotFather}, переходим по \href{https://t.me/BotFather}{ссылке} и пишем боту \texttt{/start}.

После чего вы получите сообщение со списком команд:

\begin{verbatim}
I can help you create and manage Telegram bots. If you're new to the Bot API, please see the manual (https://core.telegram.org/bots).

You can control me by sending these commands:

/newbot - create a new bot
/mybots - edit your bots [beta]

Edit Bots
/setname - change a bot's name
/setdescription - change bot description
/setabouttext - change bot about info
/setuserpic - change bot profile photo
/setcommands - change the list of commands
/deletebot - delete a bot

Bot Settings
/token - generate authorization token
/revoke - revoke bot access token
/setinline - toggle inline mode (https://core.telegram.org/bots/inline)
/setinlinegeo - toggle inline location requests (https://core.telegram.org/bots/inline#location-based-results)
/setinlinefeedback - change inline feedback (https://core.telegram.org/bots/inline#collecting-feedback) settings
/setjoingroups - can your bot be added to groups?
/setprivacy - toggle privacy mode (https://core.telegram.org/bots#privacy-mode) in groups

Games
/mygames - edit your games (https://core.telegram.org/bots/games) [beta]
/newgame - create a new game (https://core.telegram.org/bots/games)
/listgames - get a list of your games
/editgame - edit a game
/deletegame - delete an existing game
\end{verbatim}

Для создания нового бота отправляем команду \texttt{/newbot}.

BotFather попросит вас ввести имя и логин бота.

\begin{verbatim}
BotFather, [25.07.20 09:39]
Alright, a new bot. How are we going to call it? Please choose a name for your bot.

Alexey Seleznev, [25.07.20 09:40]
My Test Bot

BotFather, [25.07.20 09:40]
Good. Now let's choose a username for your bot. It must end in `bot`. Like this, for example: TetrisBot or tetris_bot.

Alexey Seleznev, [25.07.20 09:40]
@my_test_bot
\end{verbatim}

Имя вы можете ввести произвольное, а логин должен заканчиваться на \texttt{bot}.

Если вы всё сделали правильно, то получите следующее сообщение:

\begin{verbatim}
Done! Congratulations on your new bot. You will find it at t.me/my_test_bot. You can now add a description, about section and profile picture for your bot, see /help for a list of commands. By the way, when you've finished creating your cool bot, ping our Bot Support if you want a better username for it. Just make sure the bot is fully operational before you do this.

Use this token to access the HTTP API:
123456789:abcdefghijklmnopqrstuvwxyz

For a description of the Bot API, see this page: https://core.telegram.org/bots/api
\end{verbatim}

Далее вам понадобится полученный API токен, в моём примере это \texttt{123456789:abcdefghijklmnopqrstuvwxyz}.

Более подробно о возможностях \textbf{BotFather} можно узнать из \href{https://botcreators.ru/blog/botfather-instrukciya/}{этой публикации}. На этом шаге подготовительные работы по созданию бота завершены.

\hypertarget{ux443ux441ux442ux430ux43dux43eux432ux43aux430-ux43fux430ux43aux435ux442ux430-ux434ux43bux44f-ux440ux430ux431ux43eux442ux44b-ux441-ux442ux435ux43bux435ux433ux440ux430ux43c-ux431ux43eux442ux43eux43c-ux43dux430-r}{%
\section{Установка пакета для работы с телеграм ботом на R}\label{ux443ux441ux442ux430ux43dux43eux432ux43aux430-ux43fux430ux43aux435ux442ux430-ux434ux43bux44f-ux440ux430ux431ux43eux442ux44b-ux441-ux442ux435ux43bux435ux433ux440ux430ux43c-ux431ux43eux442ux43eux43c-ux43dux430-r}}

Я предполагаю, что у вас уже установлен язык R, и среда разработки RStudio. Если это не так, то вы можете посмотреть данный \href{https://youtu.be/wFUoaeGEMmY}{видео урок} о том, как их установить.

Для работы с Telegram Bot API мы будем использовать R пакет \href{https://CRAN.R-project.org/package=telegram.bot}{telegram.bot}.

Установка пакетов в R осуществляется функцией \texttt{install.packages()}, поэтому для установки нужного нам пакета используйте команду \texttt{install.packages("telegram.bot")}.

Более подробно узнать об установке различных пакетов можно из \href{https://youtu.be/1UvrWoZugic}{этого видео}.

После установки пакета его необходимо подключить:

\begin{Shaded}
\begin{Highlighting}[]
\FunctionTok{library}\NormalTok{(telegram.bot)}
\end{Highlighting}
\end{Shaded}

\hypertarget{ux43eux442ux43fux440ux430ux432ux43aux430-ux441ux43eux43eux431ux449ux435ux43dux438ux439-ux438ux437-r-ux432-telegram}{%
\section{Отправка сообщений из R в Telegram}\label{ux43eux442ux43fux440ux430ux432ux43aux430-ux441ux43eux43eux431ux449ux435ux43dux438ux439-ux438ux437-r-ux432-telegram}}

Созданного вами бота можно найти в Telegram по заданному при создании логину, в моём случае это \texttt{@my\_test\_bot}.

Отправьте боту любое сообщение, например ``Привет бот''. На данный момент это нам надо для того, что бы получить id вашего с ботом чата.

Теперь в R пишем следующий код.

\begin{Shaded}
\begin{Highlighting}[]
\FunctionTok{library}\NormalTok{(telegram.bot)}

\CommentTok{\# создаём экземпляр бота}
\NormalTok{bot }\OtherTok{\textless{}{-}} \FunctionTok{Bot}\NormalTok{(}\AttributeTok{token =} \StringTok{"123456789:abcdefghijklmnopqrstuvwxyz"}\NormalTok{)}

\CommentTok{\# Запрашиваем информацию о боте}
\FunctionTok{print}\NormalTok{(bot}\SpecialCharTok{$}\FunctionTok{getMe}\NormalTok{())}

\CommentTok{\# Получаем обновления бота, т.е. список отправленных ему сообщений}
\NormalTok{updates }\OtherTok{\textless{}{-}}\NormalTok{ bot}\SpecialCharTok{$}\FunctionTok{getUpdates}\NormalTok{()}

\CommentTok{\# Запрашиваем идентификатор чата}
\CommentTok{\# Примечание: перед запросом обновлений вы должны отправить боту сообщение}
\NormalTok{chat\_id }\OtherTok{\textless{}{-}}\NormalTok{ updates[[1L]]}\SpecialCharTok{$}\FunctionTok{from\_chat\_id}\NormalTok{()}
\end{Highlighting}
\end{Shaded}

Изначально мы создаём экземпляр нашего бота функцией \texttt{Bot()}, в качестве аргумента в неё необходимо передать полученный ранее токен.

Хранить токен в коде считается не лучшей практикой, поэтому вы можете хранить его в переменной среды, и считывать его из неё. По умолчанию в пакете \texttt{telegram.bot} реализована поддержка переменных среды следующего наименования: \texttt{R\_TELEGRAM\_BOT\_ИМЯ\_ВАШЕГО\_БОТА}. Вместо \texttt{ИМЯ\_ВАШЕГО\_БОТА} подставьте имя которое вы задали при создании, в моём случае будет переменная \texttt{R\_TELEGRAM\_BOT\_My\ Test\ Bot}.

Создать переменную среды можно несколькими способами, я расскажу о наиболее универсальном и кроссплатформенном. Создайте в вашей домашней директории (узнать её можно с помощью команды \texttt{path.expand("\textasciitilde{}")}) текстовый файл с названием \emph{.Renviron}. Сделать это также можно с помощью команды \texttt{file.edit(path.expand(file.path("\textasciitilde{}",\ ".Renviron")))}.

И добавьте в него следующую строку.

\begin{verbatim}
R_TELEGRAM_BOT_ИМЯ_ВАШЕГО_БОТА=123456789:abcdefghijklmnopqrstuvwxyz
\end{verbatim}

Далее вы можете использовать сохранённый в переменной среды токен с помощью функции \texttt{bot\_token()}, т.е. вот так:

\begin{Shaded}
\begin{Highlighting}[]
\NormalTok{bot }\OtherTok{\textless{}{-}} \FunctionTok{Bot}\NormalTok{(}\AttributeTok{token =} \FunctionTok{bot\_token}\NormalTok{(}\StringTok{"My Test Bot"}\NormalTok{))}
\end{Highlighting}
\end{Shaded}

Метод \texttt{getUpdates()}позволяет нам получить обновления бота, т.е. сообщения которые были ему отправлены. Метод \texttt{from\_chat\_id()}, позволяет получить идентификатор чата, из которого было отправлено сообщение. Этот идентификатор нам нужен для отправки сообщений от бота.

Помимо id чата из объекта полученного методом \texttt{getUpdates()} вы получаете и некоторую другую полезную информацию. Например, информацию о пользователе, отправившем сообщение.

\begin{Shaded}
\begin{Highlighting}[]
\NormalTok{updates[[1L]]}\SpecialCharTok{$}\NormalTok{message}\SpecialCharTok{$}\NormalTok{from}
\end{Highlighting}
\end{Shaded}

\begin{verbatim}
$id
[1] 000000000

$is_bot
[1] FALSE

$first_name
[1] "Alexey"

$last_name
[1] "Seleznev"

$username
[1] "AlexeySeleznev"

$language_code
[1] "ru"
\end{verbatim}

Итак, на данном этапе у нас уже есть всё, что необходимо для отправки сообщения от бота в телеграм. Воспользуемся методом \texttt{sendMessage()}, в который необходимо передать идентификатор чата, текст сообщения, и тип разметки текста сообщения. Тип разметки может быть Markdown или HTML и устанавливается аргументом \texttt{parse\_mode}.

\begin{Shaded}
\begin{Highlighting}[]
\CommentTok{\# Отправка сообщения}
\NormalTok{bot}\SpecialCharTok{$}\FunctionTok{sendMessage}\NormalTok{(chat\_id,}
                \AttributeTok{text =} \StringTok{"Привет, *жирный текст* \_курсив\_"}\NormalTok{,}
                \AttributeTok{parse\_mode =} \StringTok{"Markdown"}
\NormalTok{)}
\end{Highlighting}
\end{Shaded}

Если вам необходимо отправить сообщение от бота не в чат, а в публичный канал, то в \texttt{chat\_id} указывайте адрс вашего канала, например \texttt{\textquotesingle{}@MyTGChannel\textquotesingle{}}.

При необходимости отправить сообщение в приватный канал, вам необходимо скопировать ссылку на любое сообщение данного канала, из ссылки получить его идентификатор, и к этому идентификатору добавить -100.

\includegraphics{http://img.netpeak.ua/alsey/1JA922T.png}

Пример ссылки приватного канала: \url{https://t.me/c/012345678/11}

Соответвенно, к id 012345678 вам необходимо добавить -100, в таком случае в \texttt{chat\_id} надо указать -100012345678.

\textbf{Основы форматирования Markdown разметки:}

\begin{itemize}
\tightlist
\item
  Жирный шрифт выделяется с помощью знака *:

  \begin{itemize}
  \tightlist
  \item
    пример: \texttt{*жирный\ шритф*}
  \item
    результат: \textbf{жирный шритф}
  \end{itemize}
\item
  Курсив задаётся нижним подчёркиванием:

  \begin{itemize}
  \tightlist
  \item
    пример: \texttt{\_курсив\_}
  \item
    результат: \emph{курсив}
  \end{itemize}
\item
  Моноширинный шрифт, которым обычно выделяется программный код, задаётся с помощью апострофов:

  \begin{itemize}
  \tightlist
  \item
    пример: `моноширинный шрифт`
  \item
    результат: \texttt{моноширинный\ шрифт}
  \end{itemize}
\end{itemize}

\textbf{Основы форматирования HTML разметки:}

В HTML вы заворачиваете часть текста, которую надо выделать, в теги, пример \texttt{\textless{}тег\textgreater{}текст\textless{}/тег\textgreater{}}.

\begin{itemize}
\tightlist
\item
  \texttt{\textless{}тег\textgreater{}} - открывающий тег
\item
  \texttt{\textless{}/тег\textgreater{}} - закрывающий тег
\end{itemize}

\textbf{Теги HTML разметки}

\begin{itemize}
\tightlist
\item
  \texttt{\textless{}b\textgreater{}} - жирный шрифт

  \begin{itemize}
  \tightlist
  \item
    пример: \texttt{\textless{}b\textgreater{}жирный\ шрифт\textless{}/b\textgreater{}}
  \item
    результат \textbf{жирный шрифт}
  \end{itemize}
\item
  \texttt{\textless{}i\textgreater{}} - курсив

  \begin{itemize}
  \tightlist
  \item
    пример: \texttt{\textless{}i\textgreater{}курсив\textless{}/i\textgreater{}}
  \item
    результат: \emph{курсив}
  \end{itemize}
\item
  \texttt{\textless{}code\textgreater{}} - моноширинный шрифт

  \begin{itemize}
  \tightlist
  \item
    пример: \texttt{\textless{}code\textbackslash{}\textgreater{}моноширинный\ шрифт\textless{}/code\textbackslash{}\textgreater{}}
  \item
    результат: \texttt{моноширинный\ шрифт}
  \end{itemize}
\end{itemize}

Помимо текста вы можете отправлять и другой контент используя специальные методы:

\begin{Shaded}
\begin{Highlighting}[]
\CommentTok{\# Отправить изображение}
\NormalTok{bot}\SpecialCharTok{$}\FunctionTok{sendPhoto}\NormalTok{(chat\_id,}
  \AttributeTok{photo =} \StringTok{"https://telegram.org/img/t\_logo.png"}
\NormalTok{)}

\CommentTok{\# Отправка голосового сообщения}
\NormalTok{bot}\SpecialCharTok{$}\FunctionTok{sendAudio}\NormalTok{(chat\_id,}
  \AttributeTok{audio =} \StringTok{"http://www.largesound.com/ashborytour/sound/brobob.mp3"}
\NormalTok{)}

\CommentTok{\# Отправить документ}
\NormalTok{bot}\SpecialCharTok{$}\FunctionTok{sendDocument}\NormalTok{(chat\_id,}
  \AttributeTok{document =} \StringTok{"https://github.com/ebeneditos/telegram.bot/raw/gh{-}pages/docs/telegram.bot.pdf"}
\NormalTok{)}

\CommentTok{\# Отправить стикер}
\NormalTok{bot}\SpecialCharTok{$}\FunctionTok{sendSticker}\NormalTok{(chat\_id,}
  \AttributeTok{sticker =} \StringTok{"https://www.gstatic.com/webp/gallery/1.webp"}
\NormalTok{)}

\CommentTok{\# Отправить видео}
\NormalTok{bot}\SpecialCharTok{$}\FunctionTok{sendVideo}\NormalTok{(chat\_id,}
  \AttributeTok{video =} \StringTok{"http://techslides.com/demos/sample{-}videos/small.mp4"}
\NormalTok{)}

\CommentTok{\# Отправить gif анимацию}
\NormalTok{bot}\SpecialCharTok{$}\FunctionTok{sendAnimation}\NormalTok{(chat\_id,}
  \AttributeTok{animation =} \StringTok{"https://media.giphy.com/media/sIIhZliB2McAo/giphy.gif"}
\NormalTok{)}

\CommentTok{\# Отправить локацию}
\NormalTok{bot}\SpecialCharTok{$}\FunctionTok{sendLocation}\NormalTok{(chat\_id,}
  \AttributeTok{latitude =} \FloatTok{51.521727}\NormalTok{,}
  \AttributeTok{longitude =} \SpecialCharTok{{-}}\FloatTok{0.117255}
\NormalTok{)}

\CommentTok{\# Имитация действия в чате}
\NormalTok{bot}\SpecialCharTok{$}\FunctionTok{sendChatAction}\NormalTok{(chat\_id,}
  \AttributeTok{action =} \StringTok{"typing"}
\NormalTok{)}
\end{Highlighting}
\end{Shaded}

Т.е. например с помощью метода \texttt{sendPhoto()} вы можете отправить сохранённый в виде изображения график, который вы построили с помощью пакета \texttt{ggplot2}.

\hypertarget{ux43aux430ux43a-ux43eux442ux43fux440ux430ux432ux438ux442ux44c-ux432-telegram-ux442ux430ux431ux43bux438ux446ux443}{%
\section{Как отправить в telegram таблицу}\label{ux43aux430ux43a-ux43eux442ux43fux440ux430ux432ux438ux442ux44c-ux432-telegram-ux442ux430ux431ux43bux438ux446ux443}}

К сожалению на момент написания книги telegram не поддерживает полноценные таблицы в HTML или Markdown, но вы можете иметировать подобие таблицы. Для этого воспользуйтесь кодом представленной ниже функции \texttt{to\_tg\_table()}:

\begin{Shaded}
\begin{Highlighting}[]
\FunctionTok{library}\NormalTok{(purrr)}
\FunctionTok{library}\NormalTok{(tidyr)}
\FunctionTok{library}\NormalTok{(stringr)}

\CommentTok{\# функция для перевода data.frame в telegram таблицу }
\NormalTok{to\_tg\_table }\OtherTok{\textless{}{-}} \ControlFlowTok{function}\NormalTok{( table, }\AttributeTok{align =} \ConstantTok{NULL}\NormalTok{, }\AttributeTok{indents =} \DecValTok{3}\NormalTok{, }\AttributeTok{parse\_mode =} \StringTok{\textquotesingle{}Markdown\textquotesingle{}}\NormalTok{ ) \{}
  
  \CommentTok{\# если выравнивание не задано то выравниваем по левому краю}
  \ControlFlowTok{if}\NormalTok{ ( }\FunctionTok{is.null}\NormalTok{(align) ) \{}
    
\NormalTok{    col\_num }\OtherTok{\textless{}{-}} \FunctionTok{length}\NormalTok{(table)}
\NormalTok{    align   }\OtherTok{\textless{}{-}} \FunctionTok{str\_c}\NormalTok{( }\FunctionTok{rep}\NormalTok{(}\StringTok{\textquotesingle{}l\textquotesingle{}}\NormalTok{, col\_num), }\AttributeTok{collapse =} \StringTok{\textquotesingle{}\textquotesingle{}}\NormalTok{ )}
  
\NormalTok{  \}}
  
  \CommentTok{\# проверяем правильно ли заданно выравнивание}
  \ControlFlowTok{if}\NormalTok{ ( }\FunctionTok{length}\NormalTok{(table) }\SpecialCharTok{!=} \FunctionTok{nchar}\NormalTok{(align) ) \{}
    
\NormalTok{    align }\OtherTok{\textless{}{-}} \ConstantTok{NULL}
    
\NormalTok{  \}}
  
  \CommentTok{\# новое выравнивание столбцов }
\NormalTok{  side }\OtherTok{\textless{}{-}} \FunctionTok{sapply}\NormalTok{(}\DecValTok{1}\SpecialCharTok{:}\FunctionTok{nchar}\NormalTok{(align), }
         \ControlFlowTok{function}\NormalTok{(x) \{ }
\NormalTok{           letter }\OtherTok{\textless{}{-}} \FunctionTok{substr}\NormalTok{(align, x, x)}
           \ControlFlowTok{switch}\NormalTok{ (letter,}
                   \StringTok{\textquotesingle{}l\textquotesingle{}} \OtherTok{=} \StringTok{\textquotesingle{}right\textquotesingle{}}\NormalTok{,}
                   \StringTok{\textquotesingle{}r\textquotesingle{}} \OtherTok{=} \StringTok{\textquotesingle{}left\textquotesingle{}}\NormalTok{,}
                   \StringTok{\textquotesingle{}c\textquotesingle{}} \OtherTok{=} \StringTok{\textquotesingle{}both\textquotesingle{}}\NormalTok{,}
                   \StringTok{\textquotesingle{}left\textquotesingle{}}
\NormalTok{           )}
\NormalTok{  \})}
  
  \CommentTok{\# сохраняем имена}
\NormalTok{  t\_names      }\OtherTok{\textless{}{-}} \FunctionTok{names}\NormalTok{(table)}

  \CommentTok{\# вычисляем ширину столбцов}
\NormalTok{  names\_length }\OtherTok{\textless{}{-}} \FunctionTok{sapply}\NormalTok{(t\_names, nchar) }
\NormalTok{  value\_length }\OtherTok{\textless{}{-}} \FunctionTok{sapply}\NormalTok{(table, }\ControlFlowTok{function}\NormalTok{(x) }\FunctionTok{max}\NormalTok{(}\FunctionTok{nchar}\NormalTok{(}\FunctionTok{as.character}\NormalTok{(x))))}
\NormalTok{  max\_length   }\OtherTok{\textless{}{-}} \FunctionTok{ifelse}\NormalTok{(value\_length }\SpecialCharTok{\textgreater{}}\NormalTok{ names\_length, value\_length, names\_length)}
  
  \CommentTok{\# подгоняем размер имён столбцов под их ширину + указанное в indents к{-}во пробелов }
\NormalTok{  t\_names }\OtherTok{\textless{}{-}} \FunctionTok{mapply}\NormalTok{(str\_pad, }
                    \AttributeTok{string =}\NormalTok{ t\_names, }
                    \AttributeTok{width  =}\NormalTok{ max\_length }\SpecialCharTok{+}\NormalTok{ indents, }
                    \AttributeTok{side   =}\NormalTok{ side)}
  
  \CommentTok{\# объединяем названия столбцов}
\NormalTok{  str\_names }\OtherTok{\textless{}{-}} \FunctionTok{str\_c}\NormalTok{(t\_names, }\AttributeTok{collapse =} \StringTok{\textquotesingle{}\textquotesingle{}}\NormalTok{)}
  
  \CommentTok{\# аргументы для фукнции str\_pad}
\NormalTok{  rules }\OtherTok{\textless{}{-}} \FunctionTok{list}\NormalTok{(}\AttributeTok{string =}\NormalTok{ table, }\AttributeTok{width =}\NormalTok{ max\_length }\SpecialCharTok{+}\NormalTok{ indents, }\AttributeTok{side =}\NormalTok{ side)}

  \CommentTok{\# поочереди переводим каждый столбец к нужному виду}
\NormalTok{  t\_str }\OtherTok{\textless{}{-}}   \FunctionTok{pmap\_df}\NormalTok{( rules, str\_pad )}\SpecialCharTok{\%\textgreater{}\%}
              \FunctionTok{unite}\NormalTok{(}\StringTok{"data"}\NormalTok{, }\FunctionTok{everything}\NormalTok{(), }\AttributeTok{remove =} \ConstantTok{TRUE}\NormalTok{, }\AttributeTok{sep =} \StringTok{\textquotesingle{}\textquotesingle{}}\NormalTok{) }\SpecialCharTok{\%\textgreater{}\%}
              \FunctionTok{unlist}\NormalTok{(data) }\SpecialCharTok{\%\textgreater{}\%}
              \FunctionTok{str\_c}\NormalTok{(}\AttributeTok{collapse =} \StringTok{\textquotesingle{}}\SpecialCharTok{\textbackslash{}n}\StringTok{\textquotesingle{}}\NormalTok{) }
  
  \CommentTok{\# если таблица занимает более 4096 символов обрезаем её}
  \ControlFlowTok{if}\NormalTok{ ( }\FunctionTok{nchar}\NormalTok{(t\_str) }\SpecialCharTok{\textgreater{}=} \DecValTok{4021}\NormalTok{ ) \{}
    
    \FunctionTok{warning}\NormalTok{(}\StringTok{\textquotesingle{}Таблица составляет более 4096 символов!\textquotesingle{}}\NormalTok{)}
\NormalTok{    t\_str }\OtherTok{\textless{}{-}} \FunctionTok{substr}\NormalTok{(t\_str, }\DecValTok{1}\NormalTok{, }\DecValTok{4021}\NormalTok{)}
    
\NormalTok{  \}}
  
  \CommentTok{\# символы выделения блока кода согласно выбранной разметке}
\NormalTok{  code\_block }\OtherTok{\textless{}{-}} \ControlFlowTok{switch}\NormalTok{(parse\_mode, }
                       \StringTok{\textquotesingle{}Markdown\textquotesingle{}} \OtherTok{=} \FunctionTok{c}\NormalTok{(}\StringTok{\textquotesingle{}\textasciigrave{}\textasciigrave{}\textasciigrave{}\textquotesingle{}}\NormalTok{, }\StringTok{\textquotesingle{}\textasciigrave{}\textasciigrave{}\textasciigrave{}\textquotesingle{}}\NormalTok{),}
                       \StringTok{\textquotesingle{}HTML\textquotesingle{}} \OtherTok{=} \FunctionTok{c}\NormalTok{(}\StringTok{\textquotesingle{}\textless{}code\textgreater{}\textquotesingle{}}\NormalTok{, }\StringTok{\textquotesingle{}\textless{}/code\textgreater{}\textquotesingle{}}\NormalTok{))}
           
  \CommentTok{\# переводим в code}
\NormalTok{  res }\OtherTok{\textless{}{-}} \FunctionTok{str\_c}\NormalTok{(code\_block[}\DecValTok{1}\NormalTok{], str\_names, t\_str, code\_block[}\DecValTok{2}\NormalTok{], }\AttributeTok{sep =} \StringTok{\textquotesingle{}}\SpecialCharTok{\textbackslash{}n}\StringTok{\textquotesingle{}}\NormalTok{)}
  
  \FunctionTok{return}\NormalTok{(res)}
\NormalTok{\}}
\end{Highlighting}
\end{Shaded}

С помощью этой функци вы можете преобразовать любой data.frame и отправить в telegram:

\begin{Shaded}
\begin{Highlighting}[]
\CommentTok{\# преобразуем таблицу iris }
\NormalTok{tg\_table }\OtherTok{\textless{}{-}} \FunctionTok{to\_tg\_table}\NormalTok{( }\FunctionTok{head}\NormalTok{(iris, }\DecValTok{15}\NormalTok{) )}

\CommentTok{\# отправляем таблицу в telegram}
\NormalTok{bot}\SpecialCharTok{$}\FunctionTok{sendMessage}\NormalTok{(}\DecValTok{194336771}\NormalTok{, }
\NormalTok{                tg\_table,}
                \StringTok{"Markdown"}\NormalTok{)}
\end{Highlighting}
\end{Shaded}

В telegram это буедет выглядеть так:
\includegraphics{http://img.netpeak.ua/alsey/160295135597_kiss_150kb.png}

У функции \texttt{to\_tg\_table()} есть несколько дополнительных аргументов:

\begin{itemize}
\tightlist
\item
  align - выравнивнивание столбцов, тектовая строка, каждая буква соответвует одному столбцу, пример \texttt{\textquotesingle{}llrrc\textquotesingle{}}:

  \begin{itemize}
  \tightlist
  \item
    l - выравнивание по левой стороне
  \item
    r - выравнивание по правой стороне
  \item
    c - выравнивание по центру
  \end{itemize}
\item
  indents - количество пробелов для разделения столбцов.
\item
  parse\_mode - разметка сообщения, Markdown или HTML.
\end{itemize}

Пример с выравниванием столбцов:

\begin{Shaded}
\begin{Highlighting}[]
\CommentTok{\# преобразуем таблицу iris }
\NormalTok{tg\_table }\OtherTok{\textless{}{-}} \FunctionTok{to\_tg\_table}\NormalTok{( }\FunctionTok{head}\NormalTok{(iris, }\DecValTok{15}\NormalTok{), }
                         \AttributeTok{align =} \StringTok{\textquotesingle{}llccr\textquotesingle{}}\NormalTok{)}

\CommentTok{\# отправляем таблицу в telegram}
\NormalTok{bot}\SpecialCharTok{$}\FunctionTok{sendMessage}\NormalTok{(}\DecValTok{194336771}\NormalTok{, }
\NormalTok{                tg\_table,}
                \StringTok{"Markdown"}\NormalTok{)}
\end{Highlighting}
\end{Shaded}

\includegraphics{http://img.netpeak.ua/alsey/160295200291_kiss_174kb.png}

\hypertarget{ux43aux430ux43a-ux434ux43eux431ux430ux432ux438ux442ux44c-ux432-ux441ux43eux43eux431ux449ux435ux43dux438ux435-emoji}{%
\section{Как добавить в сообщение Emoji}\label{ux43aux430ux43a-ux434ux43eux431ux430ux432ux438ux442ux44c-ux432-ux441ux43eux43eux431ux449ux435ux43dux438ux435-emoji}}

Требования к телеграм ботам могут быть разные, в том числе заказчик может попросить вас добавить в сообщения бота какие то Emoji.

Получить полный список доступных смайлов можно по этой \href{https://apps.timwhitlock.info/emoji/tables/unicode}{ссылке}.

\begin{figure}
\centering
\includegraphics{http://img.netpeak.ua/alsey/160070556432_kiss_104kb.png}
\caption{Таблица смайлов}
\end{figure}

Из таблицы нас интересует поле \textbf{Unicode}. Скопиройте код нужного вам смайла, и замените \texttt{U+} на \texttt{\textbackslash{}U000}. Т.е. если вам необходимо отправить смайл, код котого в \href{https://apps.timwhitlock.info/emoji/tables/unicode}{таблице} \texttt{U+1F601}, то в коде на R вам необходимо добавить его в текст сообщения вот так - \texttt{\textbackslash{}U0001F601}.

Пример:

\begin{Shaded}
\begin{Highlighting}[]
\NormalTok{bot}\SpecialCharTok{$}\FunctionTok{sendMessage}\NormalTok{(chat\_id, }
                \StringTok{\textquotesingle{}Сообщение со смайлом \textbackslash{}U0001F601 код которого в таблице U+1F601\textquotesingle{}}\NormalTok{)}
\end{Highlighting}
\end{Shaded}

Результат:
\includegraphics{http://img.netpeak.ua/alsey/160070602417_kiss_156kb.png}

\hypertarget{ux43fux440ux43eux432ux435ux440ux43aux430-ux43fux43bux430ux43dux438ux440ux43eux432ux449ux438ux43aux430-ux437ux430ux434ux430ux447-windows-ux438-ux43eux442ux43fux440ux430ux432ux43aux430-ux443ux432ux435ux434ux43eux43cux43bux435ux43dux438ux44f-ux43e-ux437ux430ux434ux430ux447ux430ux445-ux440ux430ux431ux43eux442ux430-ux43aux43eux442ux43eux440ux44bux445-ux431ux44bux43bux430-ux437ux430ux432ux435ux440ux448ux435ux43dux430-ux430ux432ux430ux440ux438ux439ux43dux43e}{%
\section{Проверка планировщика задач Windows, и отправка уведомления о задачах, работа которых была завершена аварийно}\label{ux43fux440ux43eux432ux435ux440ux43aux430-ux43fux43bux430ux43dux438ux440ux43eux432ux449ux438ux43aux430-ux437ux430ux434ux430ux447-windows-ux438-ux43eux442ux43fux440ux430ux432ux43aux430-ux443ux432ux435ux434ux43eux43cux43bux435ux43dux438ux44f-ux43e-ux437ux430ux434ux430ux447ux430ux445-ux440ux430ux431ux43eux442ux430-ux43aux43eux442ux43eux440ux44bux445-ux431ux44bux43bux430-ux437ux430ux432ux435ux440ux448ux435ux43dux430-ux430ux432ux430ux440ux438ux439ux43dux43e}}

Для работы с планировщиком заданий Windows вам необходимо установить пакет \texttt{taskscheduleR}, и для удобства работы с данными установим пакет \texttt{dplyr}.

\begin{Shaded}
\begin{Highlighting}[]
\CommentTok{\# Установка пакетов}
\FunctionTok{install.packages}\NormalTok{(}\FunctionTok{c}\NormalTok{(}\StringTok{\textquotesingle{}taskscheduleR\textquotesingle{}}\NormalTok{, }\StringTok{\textquotesingle{}dplyr\textquotesingle{}}\NormalTok{))}
\CommentTok{\# Подключение пакетов}
\FunctionTok{library}\NormalTok{(taskscheduleR)}
\FunctionTok{library}\NormalTok{(dplyr)}
\end{Highlighting}
\end{Shaded}

Далее с помощью функции \texttt{taskscheduler\_ls()} мы запрашиваем информацию о задачах из нашего планировщика. С помощью функции \texttt{filter()} из пакета \texttt{dplyr} мы убираем из списка задач те, которые были успешно выполненны и имеют статус последнего результата 0, и те, которые ещё ни разу не запускались и имеют статус 267011, выключенные задачи, и задачи которые выполняются в данный момент.

\begin{Shaded}
\begin{Highlighting}[]
\CommentTok{\# запрашиваем список задач}
\NormalTok{task }\OtherTok{\textless{}{-}}\NormalTok{ task }\OtherTok{\textless{}{-}} \FunctionTok{taskscheduler\_ls}\NormalTok{() }\SpecialCharTok{\%\textgreater{}\%}
        \FunctionTok{filter}\NormalTok{(}\SpecialCharTok{!} \StringTok{\textasciigrave{}}\AttributeTok{Last Result}\StringTok{\textasciigrave{}}  \SpecialCharTok{\%in\%} \FunctionTok{c}\NormalTok{(}\StringTok{"0"}\NormalTok{, }\StringTok{"267011"}\NormalTok{) }\SpecialCharTok{\&} 
               \StringTok{\textasciigrave{}}\AttributeTok{Scheduled Task State}\StringTok{\textasciigrave{}} \SpecialCharTok{==} \StringTok{"Enabled"} \SpecialCharTok{\&} 
\NormalTok{               Status }\SpecialCharTok{!=} \StringTok{"Running"}\NormalTok{) }\SpecialCharTok{\%\textgreater{}\%}
        \FunctionTok{select}\NormalTok{(TaskName) }\SpecialCharTok{\%\textgreater{}\%}
        \FunctionTok{unique}\NormalTok{() }\SpecialCharTok{\%\textgreater{}\%}
        \FunctionTok{unlist}\NormalTok{() }\SpecialCharTok{\%\textgreater{}\%}
        \FunctionTok{paste0}\NormalTok{(., }\AttributeTok{collapse =} \StringTok{"}\SpecialCharTok{\textbackslash{}n}\StringTok{"}\NormalTok{)}
\end{Highlighting}
\end{Shaded}

В объекте \texttt{task} у нас теперь список задач, работа которых завершилась ошибкой, этот список нам надо отправить в Telegram.

Если рассмотреть каждую команду подробнее, то:

\begin{itemize}
\tightlist
\item
  \texttt{filter()} - фильтрует список задач, по описанным выше условиям
\item
  \texttt{select()} - оставляет в таблице только одно поле с названием задач
\item
  \texttt{unique()} - убирает дубли названий
\item
  \texttt{unlist()} - переводит выбранный столбец таблицы в вектор
\item
  \texttt{paste0()} - соединяет названия задач в одну строку, и ставит в качестве разделителя знак перевода строки, т.е. \texttt{\textbackslash{}n}.
\end{itemize}

Всё что нам остаётся - отправить этот результат в телеграм.

\begin{Shaded}
\begin{Highlighting}[]
\NormalTok{bot}\SpecialCharTok{$}\FunctionTok{sendMessage}\NormalTok{(chat\_id,}
                \AttributeTok{text =}\NormalTok{ task,}
                \AttributeTok{parse\_mode =} \StringTok{"Markdown"}
\NormalTok{)}
\end{Highlighting}
\end{Shaded}

Итак, на данный момент код бота выглядит вот так:

\begin{Shaded}
\begin{Highlighting}[]
\CommentTok{\# Подключение пакета}
\FunctionTok{library}\NormalTok{(telegram.bot)}
\FunctionTok{library}\NormalTok{(taskscheduleR)}
\FunctionTok{library}\NormalTok{(dplyr)}

\CommentTok{\# инициализируем бота}
\NormalTok{bot }\OtherTok{\textless{}{-}} \FunctionTok{Bot}\NormalTok{(}\AttributeTok{token =} \StringTok{"123456789:abcdefghijklmnopqrstuvwxyz"}\NormalTok{)}

\CommentTok{\# идентификатор чата}
\NormalTok{chat\_id }\OtherTok{\textless{}{-}} \DecValTok{123456789}

\CommentTok{\# запрашиваем список задач}
\NormalTok{task }\OtherTok{\textless{}{-}} \FunctionTok{taskscheduler\_ls}\NormalTok{() }\SpecialCharTok{\%\textgreater{}\%}
        \FunctionTok{filter}\NormalTok{(}\SpecialCharTok{!} \StringTok{\textasciigrave{}}\AttributeTok{Last Result}\StringTok{\textasciigrave{}}  \SpecialCharTok{\%in\%} \FunctionTok{c}\NormalTok{(}\StringTok{"0"}\NormalTok{, }\StringTok{"267011"}\NormalTok{)  }\SpecialCharTok{\&}
               \StringTok{\textasciigrave{}}\AttributeTok{Scheduled Task State}\StringTok{\textasciigrave{}} \SpecialCharTok{==} \StringTok{"Enabled"} \SpecialCharTok{\&} 
\NormalTok{               Status }\SpecialCharTok{!=} \StringTok{"Running"}\NormalTok{) }\SpecialCharTok{\%\textgreater{}\%}
        \FunctionTok{select}\NormalTok{(TaskName) }\SpecialCharTok{\%\textgreater{}\%}
        \FunctionTok{unique}\NormalTok{() }\SpecialCharTok{\%\textgreater{}\%}
        \FunctionTok{unlist}\NormalTok{() }\SpecialCharTok{\%\textgreater{}\%}
        \FunctionTok{paste0}\NormalTok{(., }\AttributeTok{collapse =} \StringTok{"}\SpecialCharTok{\textbackslash{}n}\StringTok{"}\NormalTok{)}

\CommentTok{\# если есть проблемные задачи отправляем сообщение}
\ControlFlowTok{if}\NormalTok{ ( task }\SpecialCharTok{!=} \StringTok{""}\NormalTok{ ) \{}

\NormalTok{  bot}\SpecialCharTok{$}\FunctionTok{sendMessage}\NormalTok{(chat\_id,}
                  \AttributeTok{text =}\NormalTok{ task,}
                  \AttributeTok{parse\_mode =} \StringTok{"Markdown"}
\NormalTok{  )}

\NormalTok{\}}
\end{Highlighting}
\end{Shaded}

\emph{При использовании приведённого выше примера подставьте в код токен вашего бота и ваш идентификатор чата.}

Вы можете добавлять условия фильтрации задач, например проверяя только те задачи, которые были созданны вами, исключая системные.

Так же вы можете вынести различные настройки в отдельный файл конфигурации, и хранить в нём id чата и токен. Читать конфиг можно например с помощью пакета \texttt{configr}.

\begin{verbatim}
[telegram_bot]
;настройки телеграм бота и чата, в который будут приходить уведомления
chat_id=12345678
bot_token=123456789:abcdefghijklmnopqrstuvwxyz"
\end{verbatim}

\begin{Shaded}
\begin{Highlighting}[]
\FunctionTok{library}\NormalTok{(configr)}

\CommentTok{\# чтение конфина}
\NormalTok{config }\OtherTok{\textless{}{-}} \FunctionTok{read.config}\NormalTok{(}\StringTok{\textquotesingle{}C:/путь\_к\_конфигу/config.cfg\textquotesingle{}}\NormalTok{, }\AttributeTok{rcmd.parse =} \ConstantTok{TRUE}\NormalTok{)}

\NormalTok{bot\_token }\OtherTok{\textless{}{-}}\NormalTok{ config}\SpecialCharTok{$}\NormalTok{telegram\_bot}\SpecialCharTok{$}\NormalTok{bot\_token}
\NormalTok{chat\_id     }\OtherTok{\textless{}{-}}\NormalTok{ config}\SpecialCharTok{$}\NormalTok{telegram\_bot}\SpecialCharTok{$}\NormalTok{chat\_id}
\end{Highlighting}
\end{Shaded}

\hypertarget{ux43dux430ux441ux442ux440ux430ux438ux432ux430ux435ux43c-ux440ux430ux441ux43fux438ux441ux430ux43dux438ux435-ux437ux430ux43fux443ux441ux43aux430-ux43fux440ux43eux432ux435ux440ux43aux438-ux437ux430ux434ux430ux447}{%
\section{Настраиваем расписание запуска проверки задач}\label{ux43dux430ux441ux442ux440ux430ux438ux432ux430ux435ux43c-ux440ux430ux441ux43fux438ux441ux430ux43dux438ux435-ux437ux430ux43fux443ux441ux43aux430-ux43fux440ux43eux432ux435ux440ux43aux438-ux437ux430ux434ux430ux447}}

Наиболее подробно процесс настройки запуска скриптов по расписанию описан в этой \href{https://netpeak.net/ru/blog/kak-nastroit-zapusk-r-skripta-po-raspisaniyu/}{статье} . Тут я лишь опишу шаги, которые для этого необходимо выполнить. Если какой-то из шагов вам не понятен, то обратитесь к статье на которую я указал ссылку.

Предположим, что мы сохранили код нашего бота в файл \texttt{check\_bot.R}. Для того, что бы запланировать регулярный запуск этого файла выполните следующие шаги:

\begin{enumerate}
\def\labelenumi{\arabic{enumi}.}
\tightlist
\item
  Пропишите в системную переменную Path путь к папке в которой установлен R, в Windows путь будет примерно таким: \texttt{C:\textbackslash{}Program\ Files\textbackslash{}R\textbackslash{}R-4.0.2\textbackslash{}bin}.
\item
  Создайте исполняемый bat файл, в котором будет всего одна строка \texttt{R\ CMD\ BATCH\ C:\textbackslash{}rscripts\textbackslash{}check\_bot\textbackslash{}check\_bot.R}. Замените \texttt{C:\textbackslash{}rscripts\textbackslash{}check\_bot\textbackslash{}check\_bot.R} на полный путь к вашему R файлу.
\item
  Далее настройте с помощью планировщика задач Windows расписание запуска, например на каждые пол часа.
\end{enumerate}

\hypertarget{ux437ux430ux43aux43bux44eux447ux435ux43dux438ux435}{%
\section{Заключение}\label{ux437ux430ux43aux43bux44eux447ux435ux43dux438ux435}}

В этой главе мы разобрались с тем, как создать бота, и отправлять с его помощью различные уведомления в telegram.

Я описал задачу контроля планировщика заданий Windows, но вы можете использовать материал этой главы для отправки любых уведомлений, от прогноза погоды до котировок акций на фондовой бирже, т.к. R позволяет вам подключиться к огромному количеству источников данных.

В следующей главе мы с вами разберёмся с тем, как добавить боту команды и фильтры сообщений, для того, что он мог не только отправлять уведомления, но и выполнять более сложные действия.

\hypertarget{ux442ux435ux441ux442ux44b-ux438-ux437ux430ux434ux430ux43dux438ux44f}{%
\section{Тесты и задания}\label{ux442ux435ux441ux442ux44b-ux438-ux437ux430ux434ux430ux43dux438ux44f}}

\hypertarget{ux442ux435ux441ux442ux44b}{%
\subsection{Тесты}\label{ux442ux435ux441ux442ux44b}}

Для закрепления материла рекомендую вам пройти тест доступный по \href{https://onlinetestpad.com/t/build-tg-bot-in-r-1}{ссылке}.

\hypertarget{ux437ux430ux434ux430ux43dux438ux44f}{%
\subsection{Задания}\label{ux437ux430ux434ux430ux43dux438ux44f}}

\begin{enumerate}
\def\labelenumi{\arabic{enumi}.}
\tightlist
\item
  Создайте с помощью \href{http://t.me/BotFather}{BotFather} бота.
\item
  Перейдите к диалогу с ботом, и узнайте идентификатор вашего с ботом чата.
\item
  Отправьте с помощью созданного бота в telegram первые 20 строк из встроенного в R набора данных \texttt{ToothGrowth}.
\end{enumerate}

Если вы всё сделали правильно то результат будет следующим:
\includegraphics{http://img.netpeak.ua/alsey/160390694961_kiss_150kb.png}

\hypertarget{ux434ux43eux431ux430ux432ux43bux44fux435ux43c-ux431ux43eux442ux443-ux43fux43eux434ux434ux435ux440ux436ux43aux443-ux43aux43eux43cux430ux43dux434-ux438-ux444ux438ux43bux44cux442ux440ux44b-ux441ux43eux43eux431ux449ux435ux43dux438ux439-ux43aux43bux430ux441ux441-updater-2}{%
\chapter{Добавляем боту поддержку команд и фильтры сообщений, класс Updater (2)}\label{ux434ux43eux431ux430ux432ux43bux44fux435ux43c-ux431ux43eux442ux443-ux43fux43eux434ux434ux435ux440ux436ux43aux443-ux43aux43eux43cux430ux43dux434-ux438-ux444ux438ux43bux44cux442ux440ux44b-ux441ux43eux43eux431ux449ux435ux43dux438ux439-ux43aux43bux430ux441ux441-updater-2}}

В этой главе мы разберёмся как оживить нашего бота и добавим ему поддержку команд, а также познакомимся с классом \texttt{Updater}.

В ходе главы мы напишем нескольких простых ботов, последний будет по заданной дате и коду страны определять является ли день в данной стране выходным или рабочим согласно производственного календаря. Но, как и прежде цель книги ознакомить вас с интерфейсом пакета \texttt{telegram.bot} для решения ваших собственных задач.

\hypertarget{ux43aux43bux430ux441ux441-updater}{%
\section{Класс Updater}\label{ux43aux43bux430ux441ux441-updater}}

\texttt{Updater} - это класс, который упрощает вам разработку телеграм бота, и использует под капотом класс \texttt{Dispetcher}. Назначение класса \texttt{Updater} заключается в том, что бы получить обновления от бота (в предыдущей главе мы использовали для этой цели метод \texttt{getUpdates()}), и передать их далее в \texttt{Dispetcher}.

В свою очередь \texttt{Dispetcher} содержит в себе созданные вами обработчики, т.е. объекты класса \texttt{Handler}.

\hypertarget{handlers---ux43eux431ux440ux430ux431ux43eux442ux447ux438ux43aux438}{%
\section{Handlers - обработчики}\label{handlers---ux43eux431ux440ux430ux431ux43eux442ux447ux438ux43aux438}}

С помощью обработчиков вы добавляете в \texttt{Dispetcher} реакции бота на различные события. На момент написания книги в \texttt{telegram.bot} добавлены следующие типы обработчиков:

\begin{itemize}
\tightlist
\item
  MessageHandler - Обработчик сообщений
\item
  CommandHandler - Обработчик команд
\item
  CallbackQueryHandler - Обработчик данных отправляемых из Inline клавиатур
\item
  ErrorHandler - Обработчик ошибок при запросе обновлений от бота
\end{itemize}

\hypertarget{ux434ux43eux431ux430ux432ux43bux44fux435ux43c-ux43fux435ux440ux432ux443ux44e-ux43aux43eux43cux430ux43dux434ux443-ux431ux43eux442ux443-ux43eux431ux440ux430ux431ux43eux442ux447ux438ux43a-ux43aux43eux43cux430ux43dux434}{%
\section{Добавляем первую команду боту, обработчик команд}\label{ux434ux43eux431ux430ux432ux43bux44fux435ux43c-ux43fux435ux440ux432ux443ux44e-ux43aux43eux43cux430ux43dux434ux443-ux431ux43eux442ux443-ux43eux431ux440ux430ux431ux43eux442ux447ux438ux43a-ux43aux43eux43cux430ux43dux434}}

Если вы никогда ранее не использовали ботов, и не в курсе, что такое команда, то команды боту необходимо отправлять с помощью прямого слеша \texttt{/} в качестве префикса.

Начнём мы с простых команд, т.е. научим нашего бота здороваться по команде \texttt{/hi}.

\begin{Shaded}
\begin{Highlighting}[]
\FunctionTok{library}\NormalTok{(telegram.bot)}

\CommentTok{\# создаём экземпляр класса Updater}
\NormalTok{updater }\OtherTok{\textless{}{-}} \FunctionTok{Updater}\NormalTok{(}\StringTok{\textquotesingle{}ТОКЕН ВАШЕГО БОТА\textquotesingle{}}\NormalTok{)}

\CommentTok{\# Пишем метод для приветсвия}
\NormalTok{say\_hello }\OtherTok{\textless{}{-}} \ControlFlowTok{function}\NormalTok{(bot, update) \{}

  \CommentTok{\# Имя пользователя с которым надо поздароваться}
\NormalTok{  user\_name }\OtherTok{\textless{}{-}}\NormalTok{ update}\SpecialCharTok{$}\NormalTok{message}\SpecialCharTok{$}\NormalTok{from}\SpecialCharTok{$}\NormalTok{first\_name}

  \CommentTok{\# Отправка приветственного сообщения}
\NormalTok{  bot}\SpecialCharTok{$}\FunctionTok{sendMessage}\NormalTok{(update}\SpecialCharTok{$}\NormalTok{message}\SpecialCharTok{$}\NormalTok{chat\_id,}
                  \AttributeTok{text =} \FunctionTok{paste0}\NormalTok{(}\StringTok{"Моё почтение, "}\NormalTok{, user\_name, }\StringTok{"!"}\NormalTok{),}
                  \AttributeTok{parse\_mode =} \StringTok{"Markdown"}\NormalTok{)}

\NormalTok{\}}

\CommentTok{\# создаём обработчик}
\NormalTok{hi\_hendler }\OtherTok{\textless{}{-}} \FunctionTok{CommandHandler}\NormalTok{(}\StringTok{\textquotesingle{}hi\textquotesingle{}}\NormalTok{, say\_hello)}

\CommentTok{\# добаляем обработчик в диспетчер}
\NormalTok{updater }\OtherTok{\textless{}{-}}\NormalTok{ updater }\SpecialCharTok{+}\NormalTok{ hi\_hendler}

\CommentTok{\# запускаем бота}
\NormalTok{updater}\SpecialCharTok{$}\FunctionTok{start\_polling}\NormalTok{()}
\end{Highlighting}
\end{Shaded}

\begin{quote}
Запустите приведённый выше пример кода, предварительно заменив `ТОКЕН ВАШЕГО БОТА' на реальный токен, который вы получили при создании бота через \emph{BotFather}.
\end{quote}

Метод \texttt{start\_polling()} класса \texttt{Updater}, который используется в конце кода, запускает бесконечный цикл запроса и обработки обновлений от бота.

Теперь откроем телеграм, и напишем нашему боту первую команду \texttt{/hi}.

\includegraphics{http://img.netpeak.ua/alsey/159742442588_kiss_208kb.png}

Теперь наш бот понимает команду \texttt{/hi}, и умеет с нами здороваться.

Схематически процесс построения такого простейшего бота можно изобразить следующим образом.

\includegraphics{http://img.netpeak.ua/alsey/159742510212_kiss_21kb.png}

\begin{enumerate}
\def\labelenumi{\arabic{enumi}.}
\tightlist
\item
  Создаём экземпляр класса \texttt{Updater};
\item
  Создаём методы, т.е. функции которые будет выполнять наш бот. В примере кода это функция \texttt{say\_hello()}. Функции, которые вами будут использоваться как методы бота должны иметь два обязательных аргумента - \emph{bot} и \emph{update}, и один необязательный - \emph{args}. Аргумент \emph{bot}, это и есть ваш бот, с его помощью вы можете отвечать на сообщения, отправлять сообщения, или использовать любые другие доступные боту методы. Аргумент \emph{update} это то, что бот получил от пользователя, по сути, то что в первой главе мы получали методом \texttt{getUpdates()}. Аргумент \emph{args} позволяет вам обрабатывать дополнительные данные отправленные пользователем вместе с командой, к этой теме мы ещё вернёмся немного позже;
\item
  Создаём обработчики, т.е. связываем какие-то действия пользователя с созданными на прошлом шаге методами. По сути обработчик это триггер, событие которое вызывает какую-то функцию бота. В нашем примере таким триггером является отправка команды \texttt{/hi}, и реализуется командой \texttt{hi\_hendler\ \textless{}-\ CommandHandler(\textquotesingle{}hi\textquotesingle{},\ say\_hello)}. Первый аргумент функции \texttt{CommandHandler()} позволяет вам задать команду, в нашем случае \texttt{hi}, на которую будет реагировать бот. Второй аргумент позволяет указать метод бота, мы будем вызывать метод \texttt{say\_hello}, который будет выполняться если пользователь вызвал указанную в первом аргументе команду;
\item
  Далее добавляем созданный обработчик в диспетчер нашего экземпляра класса \texttt{Updater}. Добавлять обработчики можно несколькими способами, в примере выше я использовал простейший, с помощью знака \texttt{+}, т.е. \texttt{updater\ \textless{}-\ updater\ +\ hi\_hendler}. То же самое можно сделать с помощью метода \texttt{add\_handler()}, который относится к классу \texttt{Dispatcher}, найти этот метод можно так: \texttt{updater\$dispatcher\$add\_handler()};
\item
  Запускаем бота с помощью команды \texttt{start\_polling()}.
\end{enumerate}

\hypertarget{ux43eux431ux440ux430ux431ux43eux442ux447ux438ux43a-ux442ux435ux43aux441ux442ux43eux432ux44bux445-ux441ux43eux43eux431ux449ux435ux43dux438ux439-ux438-ux444ux438ux43bux44cux442ux440ux44b}{%
\section{Обработчик текстовых сообщений и фильтры}\label{ux43eux431ux440ux430ux431ux43eux442ux447ux438ux43a-ux442ux435ux43aux441ux442ux43eux432ux44bux445-ux441ux43eux43eux431ux449ux435ux43dux438ux439-ux438-ux444ux438ux43bux44cux442ux440ux44b}}

Как отправлять боту команды мы разобрались, но иногда нам требуется, что бы бот реагировал не только на команды, но и на какие-то обычные, текстовые сообщения. Для этого необходимо использовать обработчики сообщений - \textbf{MessageHandler}.

Обычный \textbf{MessageHandler} будет реагировать на абсолютно все входящие сообщения. Поэтому зачастую обработчики сообщений используются вместе с фильтрами. Давайте научим бота здороваться не только по команде \texttt{/hi}, но и всегда, когда в сообщении отправленном боту встречается одно из следующих слов: привет, здравствуй, салют, хай, бонжур.

Пока мы не будем писать какие-то новые методы, т.к. у нас уже есть метод с помощью которого бот с нами здоровается. От нас требуется только создать нужный фильтр и обработчик сообщений.

\begin{Shaded}
\begin{Highlighting}[]
\FunctionTok{library}\NormalTok{(telegram.bot)}

\CommentTok{\# создаём экземпляр класса Updater}
\NormalTok{updater }\OtherTok{\textless{}{-}} \FunctionTok{Updater}\NormalTok{(}\StringTok{\textquotesingle{}ТОКЕН ВАШЕГО БОТА\textquotesingle{}}\NormalTok{)}

\CommentTok{\# Пишем метод для приветсвия}
\DocumentationTok{\#\# команда приветвия}
\NormalTok{say\_hello }\OtherTok{\textless{}{-}} \ControlFlowTok{function}\NormalTok{(bot, update) \{}

  \CommentTok{\# Имя пользователя с которым надо поздароваться}
\NormalTok{  user\_name }\OtherTok{\textless{}{-}}\NormalTok{ update}\SpecialCharTok{$}\NormalTok{message}\SpecialCharTok{$}\NormalTok{from}\SpecialCharTok{$}\NormalTok{first\_name}

  \CommentTok{\# Отправляем приветсвенное сообщение}
\NormalTok{  bot}\SpecialCharTok{$}\FunctionTok{sendMessage}\NormalTok{(update}\SpecialCharTok{$}\NormalTok{message}\SpecialCharTok{$}\NormalTok{chat\_id,}
                  \AttributeTok{text =} \FunctionTok{paste0}\NormalTok{(}\StringTok{"Моё почтение, "}\NormalTok{, user\_name, }\StringTok{"!"}\NormalTok{),}
                  \AttributeTok{parse\_mode =} \StringTok{"Markdown"}\NormalTok{,}
                  \AttributeTok{reply\_to\_message\_id =}\NormalTok{ update}\SpecialCharTok{$}\NormalTok{message}\SpecialCharTok{$}\NormalTok{message\_id)}

\NormalTok{\}}

\CommentTok{\# создаём фильтры}
\NormalTok{MessageFilters}\SpecialCharTok{$}\NormalTok{hi }\OtherTok{\textless{}{-}} \FunctionTok{BaseFilter}\NormalTok{(}\ControlFlowTok{function}\NormalTok{(message) \{}

  \CommentTok{\# проверяем, встречается ли в тексте сообщения слова: привет, здравствуй, салют, хай, бонжур}
  \FunctionTok{grepl}\NormalTok{(}\AttributeTok{x           =}\NormalTok{ message}\SpecialCharTok{$}\NormalTok{text,}
        \AttributeTok{pattern     =} \StringTok{\textquotesingle{}привет|здравствуй|салют|хай|бонжур\textquotesingle{}}\NormalTok{,}
        \AttributeTok{ignore.case =} \ConstantTok{TRUE}\NormalTok{)}
\NormalTok{  \}}
\NormalTok{)}

\CommentTok{\# создаём обработчик}
\NormalTok{hi\_hendler }\OtherTok{\textless{}{-}} \FunctionTok{CommandHandler}\NormalTok{(}\StringTok{\textquotesingle{}hi\textquotesingle{}}\NormalTok{, say\_hello) }\CommentTok{\# обработчик команды hi}
\NormalTok{hi\_txt\_hnd }\OtherTok{\textless{}{-}} \FunctionTok{MessageHandler}\NormalTok{(say\_hello, }\AttributeTok{filters =}\NormalTok{ MessageFilters}\SpecialCharTok{$}\NormalTok{hi)}

\CommentTok{\# добаляем обработчики в диспетчер}
\NormalTok{updater }\OtherTok{\textless{}{-}}\NormalTok{ updater }\SpecialCharTok{+}
\NormalTok{             hi\_hendler }\SpecialCharTok{+}
\NormalTok{             hi\_txt\_hnd}

\CommentTok{\# запускаем бота}
\NormalTok{updater}\SpecialCharTok{$}\FunctionTok{start\_polling}\NormalTok{()}
\end{Highlighting}
\end{Shaded}

\begin{quote}
Запустите приведённый выше пример кода, предварительно заменив `ТОКЕН ВАШЕГО БОТА' на реальный токен, который вы получили при создании бота через \emph{BotFather}.
\end{quote}

Теперь попробуем отправить боту несколько сообщений, в которых будут встречаться перечисленные ранее слова приветствия:
\includegraphics{https://img.netpeak.ua/alsey/159787794513_kiss_226kb.png}

Итак, в первую очередь мы научили бота не просто здороваться, а отвечать на приветствие. Сделали мы это с помощью аргумента \emph{reply\_to\_message\_id}, который доступен в методе \texttt{sendMessage()}, в который необходимо передать id сообщения на которое требуется ответить. Получить id сообщения можно вот так: \texttt{update\$message\$message\_id}.

Но главное, что мы сделали - добавили боту фильтр с помощью функции \texttt{BaseFilter()}:

\begin{Shaded}
\begin{Highlighting}[]
\CommentTok{\# создаём фильтры}
\NormalTok{MessageFilters}\SpecialCharTok{$}\NormalTok{hi }\OtherTok{\textless{}{-}} \FunctionTok{BaseFilter}\NormalTok{(}

  \CommentTok{\# анонимная фильтрующая функция}
  \ControlFlowTok{function}\NormalTok{(message) \{}

    \CommentTok{\# проверяем, встречается ли в тексте сообщения слова приветствия}
    \FunctionTok{grepl}\NormalTok{(}\AttributeTok{x           =}\NormalTok{ message}\SpecialCharTok{$}\NormalTok{text,}
          \AttributeTok{pattern     =} \StringTok{\textquotesingle{}привет|здравствуй|салют|хай|бонжур\textquotesingle{}}\NormalTok{,}
          \AttributeTok{ignore.case =} \ConstantTok{TRUE}\NormalTok{)}
\NormalTok{  \}}

\NormalTok{)}
\end{Highlighting}
\end{Shaded}

Как вы могли заметить, фильтры необходимо добавлять в объект \textbf{MessageFilters}, в котором изначально уже есть небольшой набор готовых фильтров. В нашем примере в объект \textbf{MessageFilters} мы добавили элемент \emph{hi}, это новый фильтр.

В функцию \texttt{BaseFilter()} вам необходимо передать фильтрующую функцию. По сути, фильтр - это просто функция, которая получает экземпляр сообщения и возвращает \emph{TRUE} или \emph{FALSE}. В нашем примере, мы написали простейшую функцию, которая с помощью базовой функции \texttt{grepl()} проверяет текст сообщения, и если он соответствует регулярному выражению \texttt{привет\textbar{}здравствуй\textbar{}салют\textbar{}хай\textbar{}бонжур} возвращает \emph{TRUE}.

Далее мы создаём обработчик сообщений \texttt{hi\_txt\_hnd\ \textless{}-\ MessageHandler(say\_hello,\ filters\ =\ MessageFilters\$hi)}. Первый аргумент функции \texttt{MessageHandler()} - метод, который будет вызывать обработчик, а второй аргумент - это фильтр по которому он будет вызываться. В нашем случае это созданный нами фильтр \texttt{MessageFilters\$hi}.

Ну и в итоге, мы добавляем в диспетчер созданный только, что обработчик \emph{hi\_txt\_hnd}.

\begin{Shaded}
\begin{Highlighting}[]
\NormalTok{updater }\OtherTok{\textless{}{-}}\NormalTok{ updater }\SpecialCharTok{+}
\NormalTok{             hi\_hendler }\SpecialCharTok{+}
\NormalTok{             hi\_txt\_hnd}
\end{Highlighting}
\end{Shaded}

Как я уже писал выше, в пакете \texttt{telegram.bot} и объекте \textbf{MessageFilters} уже есть набор встроенных фильтров, которые вы можете использовать:

\begin{itemize}
\tightlist
\item
  all - Все сообщения
\item
  text - Текстовые сообщения
\item
  command - Команды, т.е. сообщения которые начинаются на \texttt{/}
\item
  reply - Сообщения, которые являются ответом на другое сообщение
\item
  audio - Сообщения в которых содержится аудио файл
\item
  document - Сообщения с отправленным документом
\item
  photo - Сообщения с отправленными изображениями
\item
  sticker - Сообщения с отправленным стикером
\item
  video - Сообщения с видео
\item
  voice - Голосовые сообщения
\item
  contact - Сообщения в которых содержится контант телеграм пользователя
\item
  location - Сообщения с геолокацией
\item
  venue - Пересылаемые сообщения
\item
  game - Игры
\end{itemize}

Если вы хотите совместить некоторые фильтры в одном обработчике просто используйте знак \texttt{\textbar{}} - в качестве логического \textbf{ИЛИ}, и знак \texttt{\&} в качестве логического \textbf{И}. Например, если вы хотите что бы бот вызывал один и тот же метод когда он получает видео, изображение или документ используйте следующий пример создания обработчика сообщений:

\begin{Shaded}
\begin{Highlighting}[]
\NormalTok{handler }\OtherTok{\textless{}{-}} \FunctionTok{MessageHandler}\NormalTok{(callback,}
\NormalTok{  MessageFilters}\SpecialCharTok{$}\NormalTok{video }\SpecialCharTok{|}\NormalTok{ MessageFilters}\SpecialCharTok{$}\NormalTok{photo }\SpecialCharTok{|}\NormalTok{ MessageFilters}\SpecialCharTok{$}\NormalTok{document}
\NormalTok{)}
\end{Highlighting}
\end{Shaded}

\hypertarget{ux434ux43eux431ux430ux432ux43bux435ux43dux438ux435-ux43aux43eux43cux430ux43dux434-ux441-ux43fux430ux440ux430ux43cux435ux442ux440ux430ux43cux438}{%
\section{Добавление команд с параметрами}\label{ux434ux43eux431ux430ux432ux43bux435ux43dux438ux435-ux43aux43eux43cux430ux43dux434-ux441-ux43fux430ux440ux430ux43cux435ux442ux440ux430ux43cux438}}

Мы уже знаем, что такое команды, как их создавать и как заставить бота выполнить нужную команду. Но в некоторых случаях помимо названия команды, нам необходимо передать некоторые данные для её выполнения.

Ниже пример бота, который по заданной дате и стране возвращает вам тип дня из производственного календаря.

Приведённый ниже бот использует API производственного календаря \href{https://isdayoff.ru/}{isdayoff.ru}.

\begin{Shaded}
\begin{Highlighting}[]
\FunctionTok{library}\NormalTok{(telegram.bot)}

\CommentTok{\# создаём экземпляр класса Updater}
\NormalTok{updater }\OtherTok{\textless{}{-}} \FunctionTok{Updater}\NormalTok{(}\StringTok{\textquotesingle{}ТОКЕН ВАШЕГО БОТА\textquotesingle{}}\NormalTok{)}

\CommentTok{\# Пишем метод для приветсвия}
\DocumentationTok{\#\# команда приветвия}
\NormalTok{check\_date }\OtherTok{\textless{}{-}} \ControlFlowTok{function}\NormalTok{(bot, update, args) \{}

  \CommentTok{\# входящие данные}
\NormalTok{  day     }\OtherTok{\textless{}{-}}\NormalTok{ args[}\DecValTok{1}\NormalTok{]  }\CommentTok{\# дата}
\NormalTok{  country }\OtherTok{\textless{}{-}}\NormalTok{ args[}\DecValTok{2}\NormalTok{]  }\CommentTok{\# страна}

  \CommentTok{\# проверка введённых параметров}
  \ControlFlowTok{if}\NormalTok{ ( }\SpecialCharTok{!}\FunctionTok{grepl}\NormalTok{(}\StringTok{\textquotesingle{}}\SpecialCharTok{\textbackslash{}\textbackslash{}}\StringTok{d\{4\}{-}}\SpecialCharTok{\textbackslash{}\textbackslash{}}\StringTok{d\{2\}{-}}\SpecialCharTok{\textbackslash{}\textbackslash{}}\StringTok{d\{2\}\textquotesingle{}}\NormalTok{, day) ) \{}

    \CommentTok{\# Send Custom Keyboard}
\NormalTok{    bot}\SpecialCharTok{$}\FunctionTok{sendMessage}\NormalTok{(update}\SpecialCharTok{$}\NormalTok{message}\SpecialCharTok{$}\NormalTok{chat\_id,}
                    \AttributeTok{text =} \FunctionTok{paste0}\NormalTok{(day, }\StringTok{" {-} некорреткная дата, введите дату в формате ГГГГ{-}ММ{-}ДД"}\NormalTok{),}
                    \AttributeTok{parse\_mode =} \StringTok{"Markdown"}\NormalTok{)}

\NormalTok{  \} }\ControlFlowTok{else}\NormalTok{ \{}
\NormalTok{    day }\OtherTok{\textless{}{-}} \FunctionTok{as.Date}\NormalTok{(day)}
    \CommentTok{\# переводим в формат POSIXtl}
\NormalTok{    y }\OtherTok{\textless{}{-}} \FunctionTok{format}\NormalTok{(day, }\StringTok{"\%Y"}\NormalTok{)}
\NormalTok{    m }\OtherTok{\textless{}{-}} \FunctionTok{format}\NormalTok{(day, }\StringTok{"\%m"}\NormalTok{)}
\NormalTok{    d }\OtherTok{\textless{}{-}} \FunctionTok{format}\NormalTok{(day, }\StringTok{"\%d"}\NormalTok{)}

\NormalTok{  \}}

  \CommentTok{\# страна для проверки}
  \DocumentationTok{\#\# проверяем задана ли страна}
  \DocumentationTok{\#\# если не задана устанавливаем ru}
  \ControlFlowTok{if}\NormalTok{ ( }\SpecialCharTok{!}\NormalTok{ country }\SpecialCharTok{\%in\%} \FunctionTok{c}\NormalTok{(}\StringTok{\textquotesingle{}ru\textquotesingle{}}\NormalTok{, }\StringTok{\textquotesingle{}ua\textquotesingle{}}\NormalTok{, }\StringTok{\textquotesingle{}by\textquotesingle{}}\NormalTok{, }\StringTok{\textquotesingle{}kz\textquotesingle{}}\NormalTok{, }\StringTok{\textquotesingle{}us\textquotesingle{}}\NormalTok{) ) \{}

    \CommentTok{\# Send Custom Keyboard}
\NormalTok{    bot}\SpecialCharTok{$}\FunctionTok{sendMessage}\NormalTok{(update}\SpecialCharTok{$}\NormalTok{message}\SpecialCharTok{$}\NormalTok{chat\_id,}
                    \AttributeTok{text =} \FunctionTok{paste0}\NormalTok{(country, }\StringTok{" {-} некорретктный код страны, возможнные значения: ru, by, kz, ua, us. Запрошены данные по России."}\NormalTok{),}
                    \AttributeTok{parse\_mode =} \StringTok{"Markdown"}\NormalTok{)}

\NormalTok{    country }\OtherTok{\textless{}{-}} \StringTok{\textquotesingle{}ru\textquotesingle{}}

\NormalTok{  \}}

  \CommentTok{\# запрос данных из API}
  \CommentTok{\# компоновка HTTP запроса}
\NormalTok{  url }\OtherTok{\textless{}{-}} \FunctionTok{paste0}\NormalTok{(}\StringTok{"https://isdayoff.ru/api/getdata?"}\NormalTok{,}
                \StringTok{"year="}\NormalTok{,  y, }\StringTok{"\&"}\NormalTok{,}
                \StringTok{"month="}\NormalTok{, m, }\StringTok{"\&"}\NormalTok{,}
                \StringTok{"day="}\NormalTok{,   d, }\StringTok{"\&"}\NormalTok{,}
                \StringTok{"cc="}\NormalTok{,    country, }\StringTok{"\&"}\NormalTok{,}
                \StringTok{"pre=1\&"}\NormalTok{,}
                \StringTok{"covid=1"}\NormalTok{)}

  \CommentTok{\# получаем ответ}
\NormalTok{  res }\OtherTok{\textless{}{-}} \FunctionTok{readLines}\NormalTok{(url)}

  \CommentTok{\# интрепретация ответа}
\NormalTok{  out }\OtherTok{\textless{}{-}} \ControlFlowTok{switch}\NormalTok{(res,}
                \StringTok{"0"}   \OtherTok{=} \StringTok{"Рабочий день"}\NormalTok{,}
                \StringTok{"1"}   \OtherTok{=} \StringTok{"Нерабочий день"}\NormalTok{,}
                \StringTok{"2"}   \OtherTok{=} \StringTok{"Сокращённый рабочий день"}\NormalTok{,}
                \StringTok{"4"}   \OtherTok{=} \StringTok{"covid{-}19"}\NormalTok{,}
                \StringTok{"100"} \OtherTok{=} \StringTok{"Ошибка в дате"}\NormalTok{,}
                \StringTok{"101"} \OtherTok{=} \StringTok{"Данные не найдены"}\NormalTok{,}
                \StringTok{"199"} \OtherTok{=} \StringTok{"Ошибка сервиса"}\NormalTok{)}

  \CommentTok{\# отправляем сообщение}
\NormalTok{  bot}\SpecialCharTok{$}\FunctionTok{sendMessage}\NormalTok{(update}\SpecialCharTok{$}\NormalTok{message}\SpecialCharTok{$}\NormalTok{chat\_id,}
                  \AttributeTok{text =} \FunctionTok{paste0}\NormalTok{(day, }\StringTok{" {-} "}\NormalTok{, out),}
                  \AttributeTok{parse\_mode =} \StringTok{"Markdown"}\NormalTok{)}

\NormalTok{\}}

\CommentTok{\# создаём обработчик}
\NormalTok{date\_hendler }\OtherTok{\textless{}{-}} \FunctionTok{CommandHandler}\NormalTok{(}\StringTok{\textquotesingle{}check\_date\textquotesingle{}}\NormalTok{, check\_date, }\AttributeTok{pass\_args =} \ConstantTok{TRUE}\NormalTok{)}

\CommentTok{\# добаляем обработчик в диспетчер}
\NormalTok{updater }\OtherTok{\textless{}{-}}\NormalTok{ updater }\SpecialCharTok{+}\NormalTok{ date\_hendler}

\CommentTok{\# запускаем бота}
\NormalTok{updater}\SpecialCharTok{$}\FunctionTok{start\_polling}\NormalTok{()}
\end{Highlighting}
\end{Shaded}

\begin{quote}
Запустите приведённый выше пример кода, предварительно заменив `ТОКЕН ВАШЕГО БОТА' на реальный токен, который вы получили при создании бота через \emph{BotFather}.
\end{quote}

Мы создали бота, который в арсенале имеет всего один метод \texttt{check\_date}, данный метод вызывается одноимённой командой.

Но, помимо имени команды, данный метод ждёт от вас введения двух параметров, код страны и дату. Далее бот проверяется, является ли заданный день в указанной стране выходным, сокращённым или рабочим согласно официального производственного календаря.

Что бы создаваемый нами метод принимал дополнительные параметры вместе с командой, используйте аргумент \texttt{pass\_args\ =\ TRUE} в функции \texttt{CommandHandler()}, и при создании метода, помимо обязательных аргументов \emph{bot}, \emph{update} создайте опциональный - \emph{args}. Созданный таким образом метод будет принимать параметры, которые вы передаёте боту после названия команды. Параметры необходимо между собой разделять пробелом, в метод они поступят в виде текстового вектора.

Давайте запустим, и протестируем нашего бота.

\includegraphics{http://img.netpeak.ua/alsey/159803059802_kiss_204kb.png}

\hypertarget{ux437ux430ux43fux443ux441ux43aux430ux435ux43c-ux431ux43eux442ux430-ux432-ux444ux43eux43dux43eux432ux43eux43c-ux440ux435ux436ux438ux43cux435}{%
\section{Запускаем бота в фоновом режиме}\label{ux437ux430ux43fux443ux441ux43aux430ux435ux43c-ux431ux43eux442ux430-ux432-ux444ux43eux43dux43eux432ux43eux43c-ux440ux435ux436ux438ux43cux435}}

Последний шаг который нам осталось выполнить - запустить бота в фоновом режиме.

Для этого следуйте по описанному ниже алгоритму:

\begin{enumerate}
\def\labelenumi{\arabic{enumi}.}
\tightlist
\item
  Сохраните код бота в файл с расширением R. При работе в RStudio это делается через меню \emph{File}, командой \emph{Save As\ldots{}}.
\item
  Добавьте путь к папке bin, которая в свою очередь находится в папке в которую вы установили язык R в переменную Path, инструкция \href{https://www.java.com/ru/download/help/path.xml}{тут}.
\item
  Создайте обычный текстовый файл, в котором пропишите 1 строку: \texttt{R\ CMD\ BATCH\ C:\textbackslash{}Users\textbackslash{}Alsey\textbackslash{}Documents\textbackslash{}my\_bot.R}. Вместо *C:\Users\Alsey\Documents\my\_bot.R* пропишите путь к своему скрипту бота. При этом важно, что бы в пути не встречалась кириллица и пробелы, т.к. это может вызвать проблемы при запуске бота. Сохраните его, и замените его расширение с \emph{txt} на \emph{bat}.
\item
  Откройте планировщик заданий Windows, есть множество способов это сделать, например откройте любую папку и в адресс введите \texttt{\%windir\%\textbackslash{}system32\textbackslash{}taskschd.msc\ /s}. Другие способы запуска можно найти \href{https://remontka.pro/open-task-scheduler-windows/}{тут}.
\item
  В верхнем правом меню планировщика нажмите ``Создать задачу\ldots{}''.
\item
  На вкладке ``Общие'' задайте произвольное имя вашей задаче, и переключатель перевидите в состояние ``Выполнять для всех пользователей''.
\item
  Перейдите на вкладку ``Действия'', нажмите ``Создать''. В поле ``Программа или сценарий'' нажмите ``Обзор'', найдите созданный на втором шаге \emph{bat} файл, и нажмите ОК.
\item
  Жмём ОК, при необходимости вводим пароль от вашей учётной записи операционной системы.
\item
  Находим в планировщике созданную задачу, выделяем и в нижнем правом углу жмём кнопку ``Выполнить''.
\end{enumerate}

Наш бот запущен в фоновом режиме, и будет работать до тех пор, пока вы не остановите задачу, или не выключите ваш ПК или сервер на котором его запустили.

\hypertarget{ux43aux430ux43a-ux434ux43eux431ux430ux432ux438ux442ux44c-ux431ux43eux442ux430-ux432-ux433ux440ux443ux43fux43fux443}{%
\section{Как добавить бота в группу}\label{ux43aux430ux43a-ux434ux43eux431ux430ux432ux438ux442ux44c-ux431ux43eux442ux430-ux432-ux433ux440ux443ux43fux43fux443}}

Для того, что бы использовать бота в публичных или закрытых группах, изначально проверьте соответвующую настройку в \href{http://t.me/BotFather}{BotFather}. По умолчанию эта настройка должна быть включена. Находится она тут: \texttt{/mybots} -\textgreater{} \texttt{@bot\_username} -\textgreater{} Bot Settings -\textgreater{} Allow Groups?. Если настройка включена то вы увидите следующее сообщение:

\includegraphics{http://img.netpeak.ua/alsey/160404844191_kiss_22kb.png}

Далее добааляете бота в нужные группы и используете его через команды. Если вам необходимо сделать так, что бы бот прослушивал не только команды, но и все сообщения в группе, то вам необходимо назначить его администратором, посе чего вы увидите что бот имеет доступ ко всем сообщениям.

\includegraphics{http://img.netpeak.ua/alsey/160404861398_kiss_36kb.png}

\hypertarget{ux43aux430ux43a-ux434ux43eux431ux430ux432ux438ux442ux44c-ux43eux43fux438ux441ux430ux43dux438ux435-ux43aux43eux43cux430ux43dux434-ux432-ux438ux43dux442ux435ux440ux444ux435ux439ux441-ux431ux43eux442ux430}{%
\section{Как добавить описание команд в интерфейс бота}\label{ux43aux430ux43a-ux434ux43eux431ux430ux432ux438ux442ux44c-ux43eux43fux438ux441ux430ux43dux438ux435-ux43aux43eux43cux430ux43dux434-ux432-ux438ux43dux442ux435ux440ux444ux435ux439ux441-ux431ux43eux442ux430}}

Теперь вы умеете создавать полноценных ботов, которых помимо вас могут использовать другие пользователи. Но, для того, что бы облегчить поиск нужных команд вы можете добавить их в интефейс бота.

Выглядеть это будет вот так:
\includegraphics{http://img.netpeak.ua/alsey/160400158803_kiss_127kb.png}

Делается это через \href{@@BotFather}{BotFather} -\textgreater{} \texttt{@bot\_username} -\textgreater{} Edit Bot -\textgreater{} Edit Commands. Далее просто передаёте название команды и через тире их описание:

\begin{verbatim}
command1 - Description
command2 - Another description
\end{verbatim}

\hypertarget{ux437ux430ux43aux43bux44eux447ux435ux43dux438ux435-1}{%
\section{Заключение}\label{ux437ux430ux43aux43bux44eux447ux435ux43dux438ux435-1}}

В этой главе мы разобрались как написать полноценного бота, который не только умеет отправлять сообщения, но и реагировать на входящие сообщения и команды. Полученных знананий уже достаточно для решения большинства ваших задач.

В следующей главе речь пойдёт о том, как добавить боту клавиатуру, для более удобной работы.

Подписываетесь на мой \href{https://t.me/R4marketing}{telegram} и \href{https://www.youtube.com/R4marketing/?sub_confirmation=1}{youtube} каналы.

\hypertarget{ux442ux435ux441ux442ux44b-ux438-ux437ux430ux434ux430ux43dux438ux44f-1}{%
\section{Тесты и задания}\label{ux442ux435ux441ux442ux44b-ux438-ux437ux430ux434ux430ux43dux438ux44f-1}}

\hypertarget{ux442ux435ux441ux442ux44b-1}{%
\subsection{Тесты}\label{ux442ux435ux441ux442ux44b-1}}

Для закрепления материла рекомендую вам пройти тест доступный по \href{https://onlinetestpad.com/t/build-tg-bot-in-r-2}{ссылке}.

\hypertarget{ux437ux430ux434ux430ux43dux438ux44f-1}{%
\subsection{Задания}\label{ux437ux430ux434ux430ux43dux438ux44f-1}}

\begin{enumerate}
\def\labelenumi{\arabic{enumi}.}
\tightlist
\item
  Создайте бота, который будет по команде \texttt{/sum} и переданное в качестве дополнительных параметров произвольное количество перечисленных через пробел чисел, возвращать их сумму.
\end{enumerate}

Если вы всё сделали правильно результат должен быть таким:
\includegraphics{http://img.netpeak.ua/alsey/160400138798_kiss_126kb.png}

\hypertarget{ux43aux430ux43a-ux434ux43eux431ux430ux432ux438ux442ux44c-ux431ux43eux442ux443-ux43fux43eux434ux434ux435ux440ux436ux43aux443-ux43aux43bux430ux432ux438ux430ux442ux443ux440ux44b-3}{%
\chapter{Как добавить боту поддержку клавиатуры (3)}\label{ux43aux430ux43a-ux434ux43eux431ux430ux432ux438ux442ux44c-ux431ux43eux442ux443-ux43fux43eux434ux434ux435ux440ux436ux43aux443-ux43aux43bux430ux432ux438ux430ux442ux443ux440ux44b-3}}

В этой главе мы повысим юзабилити нашего бота за счёт добавления клавиатуры, которая сделает интерфейс бота интуитивно понятным, и простым в использовании.

\hypertarget{ux43aux430ux43aux438ux435-ux442ux438ux43fux44b-ux43aux43bux430ux432ux438ux430ux442ux443ux440-ux43fux43eux434ux434ux435ux440ux436ux438ux432ux430ux435ux442-ux442ux435ux43bux435ux433ux440ux430ux43c-ux431ux43eux442}{%
\section{Какие типы клавиатур поддерживает телеграм бот}\label{ux43aux430ux43aux438ux435-ux442ux438ux43fux44b-ux43aux43bux430ux432ux438ux430ux442ux443ux440-ux43fux43eux434ux434ux435ux440ux436ux438ux432ux430ux435ux442-ux442ux435ux43bux435ux433ux440ux430ux43c-ux431ux43eux442}}

На момент написания книги \texttt{telegram.bot} позволяет вам создать клавиатуры двух типов:

\begin{itemize}
\tightlist
\item
  Reply - Основная, обычная клавиатура, которая находится под панелью ввода текста сообщения. Такая клавиатура просто отправляет боту текстовое сообщение, и в качестве текста отправит тот текст, который написан на самой кнопке.
\item
  Inline - Клавиатура привязанная к конкретному сообщению бота. Данная клавиатура отправляет боту данные, привязанные к нажатой кнопке, эти данные могут отличаться от текста, написанного на самой кнопке. И обрабатываются такие кнопки через \textbf{CallbackQueryHandler}.
\end{itemize}

Для того, что бы бот открыл клавиатуру необходимо при отправке сообщения через метод \texttt{sendMessage()}, передать созданную ранее клавиатуру в аргумент \texttt{reply\_markup}.

Ниже мы разберём несколько примеров.

\hypertarget{reply-ux43aux43bux430ux432ux438ux430ux442ux443ux440ux430}{%
\section{Reply клавиатура}\label{reply-ux43aux43bux430ux432ux438ux430ux442ux443ux440ux430}}

Как я уже писал выше, это основная клавиатура управления ботом.

\begin{Shaded}
\begin{Highlighting}[]
\NormalTok{bot }\OtherTok{\textless{}{-}} \FunctionTok{Bot}\NormalTok{(}\AttributeTok{token =} \StringTok{"TOKEN"}\NormalTok{)}
\NormalTok{chat\_id }\OtherTok{\textless{}{-}} \StringTok{"CHAT\_ID"}

\CommentTok{\# Create Custom Keyboard}
\NormalTok{text }\OtherTok{\textless{}{-}} \StringTok{"Aren\textquotesingle{}t those custom keyboards cool?"}
\NormalTok{RKM }\OtherTok{\textless{}{-}} \FunctionTok{ReplyKeyboardMarkup}\NormalTok{(}
  \AttributeTok{keyboard =} \FunctionTok{list}\NormalTok{(}
    \FunctionTok{list}\NormalTok{(}\FunctionTok{KeyboardButton}\NormalTok{(}\StringTok{"Yes, they certainly are!"}\NormalTok{)),}
    \FunctionTok{list}\NormalTok{(}\FunctionTok{KeyboardButton}\NormalTok{(}\StringTok{"I\textquotesingle{}m not quite sure"}\NormalTok{)),}
    \FunctionTok{list}\NormalTok{(}\FunctionTok{KeyboardButton}\NormalTok{(}\StringTok{"No..."}\NormalTok{))}
\NormalTok{  ),}
  \AttributeTok{resize\_keyboard =} \ConstantTok{FALSE}\NormalTok{,}
  \AttributeTok{one\_time\_keyboard =} \ConstantTok{TRUE}
\NormalTok{)}

\CommentTok{\# Send Custom Keyboard}
\NormalTok{bot}\SpecialCharTok{$}\FunctionTok{sendMessage}\NormalTok{(chat\_id, text, }\AttributeTok{reply\_markup =}\NormalTok{ RKM)}
\end{Highlighting}
\end{Shaded}

Выше приведён пример из официальной справки пакета \texttt{telegram.bot}. Для создания клавиатуры используется функция \texttt{ReplyKeyboardMarkup()}, которая в свою очередь принимает список списков кнопок, которые создаются функцией \texttt{KeyboardButton()}.

Почему в \texttt{ReplyKeyboardMarkup()} необходимо передавать не просто список, а список списков? Дело в том, что вы передаёте основной список, и в нём отдельными списками вы задаёте каждый ряд кнопок, т.к. в один ряд можно расположить несколько кнопок.

Аргумент \texttt{resize\_keyboard} позволяет автоматически подбирать оптимальный размер кнопок клавиатуры, а аргумент \texttt{one\_time\_keyboard} позволяет скрывать клавиатуру после каждого нажатия на кнопку.

Давайте напишем простейшего бота, у которого будет 3 кнопки:
* Чат ID - Запросить чат ID диалога с ботом
* Моё имя - Запросить своё имя
* Мой логин - Запросить своё имя пользователя в телеграм

\begin{Shaded}
\begin{Highlighting}[]
\FunctionTok{library}\NormalTok{(telegram.bot)}

\CommentTok{\# создаём экземпляр класса Updater}
\NormalTok{updater }\OtherTok{\textless{}{-}} \FunctionTok{Updater}\NormalTok{(}\StringTok{\textquotesingle{}ТОКЕН ВАШЕГО БОТА\textquotesingle{}}\NormalTok{)}

\CommentTok{\# создаём методы}
\DocumentationTok{\#\# метод для запуска клавиатуры}
\NormalTok{start }\OtherTok{\textless{}{-}} \ControlFlowTok{function}\NormalTok{(bot, update) \{}

  \CommentTok{\# создаём клавиатуру}
\NormalTok{  RKM }\OtherTok{\textless{}{-}} \FunctionTok{ReplyKeyboardMarkup}\NormalTok{(}
    \AttributeTok{keyboard =} \FunctionTok{list}\NormalTok{(}
      \FunctionTok{list}\NormalTok{(}\FunctionTok{KeyboardButton}\NormalTok{(}\StringTok{"Чат ID"}\NormalTok{)),}
      \FunctionTok{list}\NormalTok{(}\FunctionTok{KeyboardButton}\NormalTok{(}\StringTok{"Моё имя"}\NormalTok{)),}
      \FunctionTok{list}\NormalTok{(}\FunctionTok{KeyboardButton}\NormalTok{(}\StringTok{"Мой логин"}\NormalTok{))}
\NormalTok{    ),}
    \AttributeTok{resize\_keyboard =} \ConstantTok{FALSE}\NormalTok{,}
    \AttributeTok{one\_time\_keyboard =} \ConstantTok{TRUE}
\NormalTok{  )}

  \CommentTok{\# отправляем клавиатуру}
\NormalTok{  bot}\SpecialCharTok{$}\FunctionTok{sendMessage}\NormalTok{(update}\SpecialCharTok{$}\NormalTok{message}\SpecialCharTok{$}\NormalTok{chat\_id,}
                  \AttributeTok{text =} \StringTok{\textquotesingle{}Выберите команду\textquotesingle{}}\NormalTok{,}
                  \AttributeTok{reply\_markup =}\NormalTok{ RKM)}

\NormalTok{\}}

\DocumentationTok{\#\# метод возвразающий id чата}
\NormalTok{chat\_id }\OtherTok{\textless{}{-}} \ControlFlowTok{function}\NormalTok{(bot, update) \{}

\NormalTok{  bot}\SpecialCharTok{$}\FunctionTok{sendMessage}\NormalTok{(update}\SpecialCharTok{$}\NormalTok{message}\SpecialCharTok{$}\NormalTok{chat\_id,}
                  \AttributeTok{text =} \FunctionTok{paste0}\NormalTok{(}\StringTok{"Чат id этого диалога: "}\NormalTok{, update}\SpecialCharTok{$}\NormalTok{message}\SpecialCharTok{$}\NormalTok{chat\_id),}
                  \AttributeTok{parse\_mode =} \StringTok{"Markdown"}\NormalTok{)}

\NormalTok{\}}

\DocumentationTok{\#\# метод возвращающий имя}
\NormalTok{my\_name }\OtherTok{\textless{}{-}} \ControlFlowTok{function}\NormalTok{(bot, update) \{}

\NormalTok{  bot}\SpecialCharTok{$}\FunctionTok{sendMessage}\NormalTok{(update}\SpecialCharTok{$}\NormalTok{message}\SpecialCharTok{$}\NormalTok{chat\_id,}
                  \AttributeTok{text =} \FunctionTok{paste0}\NormalTok{(}\StringTok{"Вас зовут "}\NormalTok{, update}\SpecialCharTok{$}\NormalTok{message}\SpecialCharTok{$}\NormalTok{from}\SpecialCharTok{$}\NormalTok{first\_name),}
                  \AttributeTok{parse\_mode =} \StringTok{"Markdown"}\NormalTok{)}

\NormalTok{\}}

\DocumentationTok{\#\# метод возвращающий логин}
\NormalTok{my\_username }\OtherTok{\textless{}{-}} \ControlFlowTok{function}\NormalTok{(bot, update) \{}

\NormalTok{  bot}\SpecialCharTok{$}\FunctionTok{sendMessage}\NormalTok{(update}\SpecialCharTok{$}\NormalTok{message}\SpecialCharTok{$}\NormalTok{chat\_id,}
                  \AttributeTok{text =} \FunctionTok{paste0}\NormalTok{(}\StringTok{"Ваш логин "}\NormalTok{, update}\SpecialCharTok{$}\NormalTok{message}\SpecialCharTok{$}\NormalTok{from}\SpecialCharTok{$}\NormalTok{username),}
                  \AttributeTok{parse\_mode =} \StringTok{"Markdown"}\NormalTok{)}

\NormalTok{\}}

\CommentTok{\# создаём фильтры}
\DocumentationTok{\#\# сообщения с текстом Чат ID}
\NormalTok{MessageFilters}\SpecialCharTok{$}\NormalTok{chat\_id }\OtherTok{\textless{}{-}} \FunctionTok{BaseFilter}\NormalTok{(}\ControlFlowTok{function}\NormalTok{(message) \{}

  \CommentTok{\# проверяем текст сообщения}
\NormalTok{  message}\SpecialCharTok{$}\NormalTok{text }\SpecialCharTok{==} \StringTok{"Чат ID"}

\NormalTok{\}}
\NormalTok{)}

\DocumentationTok{\#\# сообщения с текстом Моё имя}
\NormalTok{MessageFilters}\SpecialCharTok{$}\NormalTok{name }\OtherTok{\textless{}{-}} \FunctionTok{BaseFilter}\NormalTok{(}\ControlFlowTok{function}\NormalTok{(message) \{}

  \CommentTok{\# проверяем текст сообщения}
\NormalTok{  message}\SpecialCharTok{$}\NormalTok{text }\SpecialCharTok{==} \StringTok{"Моё имя"}

\NormalTok{\}}
\NormalTok{)}

\DocumentationTok{\#\# сообщения с текстом Мой логин}
\NormalTok{MessageFilters}\SpecialCharTok{$}\NormalTok{username }\OtherTok{\textless{}{-}} \FunctionTok{BaseFilter}\NormalTok{(}\ControlFlowTok{function}\NormalTok{(message) \{}

  \CommentTok{\# проверяем текст сообщения}
\NormalTok{  message}\SpecialCharTok{$}\NormalTok{text }\SpecialCharTok{==} \StringTok{"Мой логин"}
\ErrorTok{)}

\CommentTok{\# создаём обработчики}
\NormalTok{h\_start    }\OtherTok{\textless{}{-}} \FunctionTok{CommandHandler}\NormalTok{(}\StringTok{\textquotesingle{}start\textquotesingle{}}\NormalTok{, start)}
\NormalTok{h\_chat\_id  }\OtherTok{\textless{}{-}} \FunctionTok{MessageHandler}\NormalTok{(chat\_id, }\AttributeTok{filters =}\NormalTok{ MessageFilters}\SpecialCharTok{$}\NormalTok{chat\_id)}
\NormalTok{h\_name     }\OtherTok{\textless{}{-}} \FunctionTok{MessageHandler}\NormalTok{(my\_name, }\AttributeTok{filters =}\NormalTok{ MessageFilters}\SpecialCharTok{$}\NormalTok{name)}
\NormalTok{h\_username }\OtherTok{\textless{}{-}} \FunctionTok{MessageHandler}\NormalTok{(my\_username, }\AttributeTok{filters =}\NormalTok{ MessageFilters}\SpecialCharTok{$}\NormalTok{username)}

\CommentTok{\# добавляем обработчики в диспетчер}
\NormalTok{updater }\OtherTok{\textless{}{-}}\NormalTok{ updater }\SpecialCharTok{+}
\NormalTok{            h\_start }\SpecialCharTok{+}
\NormalTok{            h\_chat\_id }\SpecialCharTok{+}
\NormalTok{            h\_name }\SpecialCharTok{+}
\NormalTok{            h\_username}

\CommentTok{\# запускаем бота}
\NormalTok{updater}\SpecialCharTok{$}\FunctionTok{start\_polling}\NormalTok{()}
\end{Highlighting}
\end{Shaded}

\begin{quote}
Запустите приведённый выше пример кода, предварительно заменив `ТОКЕН ВАШЕГО БОТА' на реальный токен, который вы получили при создании бота через \emph{BotFather}.
\end{quote}

После запуска задайте боту команду \texttt{/start}, т.к. именно её мы определили для запуска клавиатуры.

\includegraphics{https://img.netpeak.ua/alsey/159860932526_kiss_163kb.png}

Если на данный момент вам сложно разобрать приведённый пример кода, с созданием методов, фильтров и обработчиков, то следует вернуться к предыдущей главе, в которой я подробно всё это описал.

Мы создали 4 метода:

\begin{itemize}
\tightlist
\item
  start - Запуск клавиатуры
\item
  chat\_id - Запрос идентификатора чата
\item
  my\_name - Запрос своего имени
\item
  my\_username - Запрос своего логина
\end{itemize}

В объект \emph{MessageFilters} добавили 3 фильтра сообщений, по их тексту:

\begin{itemize}
\tightlist
\item
  chat\_id - Сообщения с текстом \texttt{"Чат\ ID"}
\item
  name - Сообщения с текстом \texttt{"Моё\ имя"}
\item
  username - Сообщения с текстом \texttt{"Мой\ логин"}
\end{itemize}

И создали 4 обработчика, которые по заданным командам и фильтрам будут выполнять указанные методы.

\begin{Shaded}
\begin{Highlighting}[]
\CommentTok{\# создаём обработчики}
\NormalTok{h\_start    }\OtherTok{\textless{}{-}} \FunctionTok{CommandHandler}\NormalTok{(}\StringTok{\textquotesingle{}start\textquotesingle{}}\NormalTok{, start)}
\NormalTok{h\_chat\_id  }\OtherTok{\textless{}{-}} \FunctionTok{MessageHandler}\NormalTok{(chat\_id, }\AttributeTok{filters =}\NormalTok{ MessageFilters}\SpecialCharTok{$}\NormalTok{chat\_id)}
\NormalTok{h\_name     }\OtherTok{\textless{}{-}} \FunctionTok{MessageHandler}\NormalTok{(my\_name, }\AttributeTok{filters =}\NormalTok{ MessageFilters}\SpecialCharTok{$}\NormalTok{name)}
\NormalTok{h\_username }\OtherTok{\textless{}{-}} \FunctionTok{MessageHandler}\NormalTok{(my\_username, }\AttributeTok{filters =}\NormalTok{ MessageFilters}\SpecialCharTok{$}\NormalTok{username)}
\end{Highlighting}
\end{Shaded}

Сама клавиатура создаётся внутри метода \texttt{start()} командой \texttt{ReplyKeyboardMarkup()}.

\begin{Shaded}
\begin{Highlighting}[]
\NormalTok{RKM }\OtherTok{\textless{}{-}} \FunctionTok{ReplyKeyboardMarkup}\NormalTok{(}
    \AttributeTok{keyboard =} \FunctionTok{list}\NormalTok{(}
      \FunctionTok{list}\NormalTok{(}\FunctionTok{KeyboardButton}\NormalTok{(}\StringTok{"Чат ID"}\NormalTok{)),}
      \FunctionTok{list}\NormalTok{(}\FunctionTok{KeyboardButton}\NormalTok{(}\StringTok{"Моё имя"}\NormalTok{)),}
      \FunctionTok{list}\NormalTok{(}\FunctionTok{KeyboardButton}\NormalTok{(}\StringTok{"Мой логин"}\NormalTok{))}
\NormalTok{    ),}
    \AttributeTok{resize\_keyboard =} \ConstantTok{FALSE}\NormalTok{,}
    \AttributeTok{one\_time\_keyboard =} \ConstantTok{TRUE}
\NormalTok{)}
\end{Highlighting}
\end{Shaded}

В нашем случае все кнопки мы расположили друг под другом, но мы можем расположить их в один ряд, внеся изменения в список списков кнопок. Т.к. один ряд внутри клавиатуры создаётся через вложенный список кнопок, то для того, что бы вывести наши кнопки в один ряд надо переписать часть кода по построению клавиатуры вот так:

\begin{Shaded}
\begin{Highlighting}[]
\NormalTok{RKM }\OtherTok{\textless{}{-}} \FunctionTok{ReplyKeyboardMarkup}\NormalTok{(}
    \AttributeTok{keyboard =} \FunctionTok{list}\NormalTok{(}
      \FunctionTok{list}\NormalTok{(}
          \FunctionTok{KeyboardButton}\NormalTok{(}\StringTok{"Чат ID"}\NormalTok{),}
          \FunctionTok{KeyboardButton}\NormalTok{(}\StringTok{"Моё имя"}\NormalTok{),}
          \FunctionTok{KeyboardButton}\NormalTok{(}\StringTok{"Мой логин"}\NormalTok{)}
\NormalTok{     )}
\NormalTok{    ),}
    \AttributeTok{resize\_keyboard =} \ConstantTok{FALSE}\NormalTok{,}
    \AttributeTok{one\_time\_keyboard =} \ConstantTok{TRUE}
\NormalTok{)}
\end{Highlighting}
\end{Shaded}

\includegraphics{http://img.netpeak.ua/alsey/159861075655_kiss_153kb.png}

Отправляется клавиатура в чат методом \texttt{sendMessage()}, в аргументе \texttt{reply\_markup}.

\begin{Shaded}
\begin{Highlighting}[]
\NormalTok{  bot}\SpecialCharTok{$}\FunctionTok{sendMessage}\NormalTok{(update}\SpecialCharTok{$}\NormalTok{message}\SpecialCharTok{$}\NormalTok{chat\_id,}
                  \AttributeTok{text =} \StringTok{\textquotesingle{}Выберите команду\textquotesingle{}}\NormalTok{,}
                  \AttributeTok{reply\_markup =}\NormalTok{ RKM)}
\end{Highlighting}
\end{Shaded}

\hypertarget{inline-ux43aux43bux430ux432ux438ux430ux442ux443ux440ux430}{%
\section{Inline клавиатура}\label{inline-ux43aux43bux430ux432ux438ux430ux442ux443ux440ux430}}

Как я уже писал выше, Inline клавиатура привязана к конкретному сообщению. С ней работать несколько сложнее чем с основной клавиатурой.

Изначально вам необходимо добавить боту метод, для вызова Inline клавиатуры.

Для ответа на нажатие Inline кнопки также можно использовать метод бота \texttt{answerCallbackQuery()}, который может вывести уведомление в интерфейсе telegram, пользователю нажавшему Inline кнопку.

Данные отправленные с Inline кнопки не являются текстом, поэтому для их обработки необходимо создать специальный обработчик с помощью команды \texttt{CallbackQueryHandler()}.

Код построения Inline клавиатуры который приводится в официальной справке пакета \texttt{telegram.bot}.

\begin{Shaded}
\begin{Highlighting}[]
\CommentTok{\# Initialize bot}
\NormalTok{bot }\OtherTok{\textless{}{-}} \FunctionTok{Bot}\NormalTok{(}\AttributeTok{token =} \StringTok{"TOKEN"}\NormalTok{)}
\NormalTok{chat\_id }\OtherTok{\textless{}{-}} \StringTok{"CHAT\_ID"}

\CommentTok{\# Create Inline Keyboard}
\NormalTok{text }\OtherTok{\textless{}{-}} \StringTok{"Could you type their phone number, please?"}
\NormalTok{IKM }\OtherTok{\textless{}{-}} \FunctionTok{InlineKeyboardMarkup}\NormalTok{(}
  \AttributeTok{inline\_keyboard =} \FunctionTok{list}\NormalTok{(}
    \FunctionTok{list}\NormalTok{(}
      \FunctionTok{InlineKeyboardButton}\NormalTok{(}\DecValTok{1}\NormalTok{),}
      \FunctionTok{InlineKeyboardButton}\NormalTok{(}\DecValTok{2}\NormalTok{),}
      \FunctionTok{InlineKeyboardButton}\NormalTok{(}\DecValTok{3}\NormalTok{)}
\NormalTok{    ),}
    \FunctionTok{list}\NormalTok{(}
      \FunctionTok{InlineKeyboardButton}\NormalTok{(}\DecValTok{4}\NormalTok{),}
      \FunctionTok{InlineKeyboardButton}\NormalTok{(}\DecValTok{5}\NormalTok{),}
      \FunctionTok{InlineKeyboardButton}\NormalTok{(}\DecValTok{6}\NormalTok{)}
\NormalTok{    ),}
    \FunctionTok{list}\NormalTok{(}
      \FunctionTok{InlineKeyboardButton}\NormalTok{(}\DecValTok{7}\NormalTok{),}
      \FunctionTok{InlineKeyboardButton}\NormalTok{(}\DecValTok{8}\NormalTok{),}
      \FunctionTok{InlineKeyboardButton}\NormalTok{(}\DecValTok{9}\NormalTok{)}
\NormalTok{    ),}
    \FunctionTok{list}\NormalTok{(}
      \FunctionTok{InlineKeyboardButton}\NormalTok{(}\StringTok{"*"}\NormalTok{),}
      \FunctionTok{InlineKeyboardButton}\NormalTok{(}\DecValTok{0}\NormalTok{),}
      \FunctionTok{InlineKeyboardButton}\NormalTok{(}\StringTok{"\#"}\NormalTok{)}
\NormalTok{    )}
\NormalTok{  )}
\NormalTok{)}

\CommentTok{\# Send Inline Keyboard}
\NormalTok{bot}\SpecialCharTok{$}\FunctionTok{sendMessage}\NormalTok{(chat\_id, text, }\AttributeTok{reply\_markup =}\NormalTok{ IKM)}
\end{Highlighting}
\end{Shaded}

Строить Inline клавиатуру необходимо с помощью команды \texttt{InlineKeyboardMarkup()}, по такому же принципу, как и Reply клавиатуру. В \texttt{InlineKeyboardMarkup()} необходимо передать список, списков Inline кнопок, каждая отдельная кнопка создаётся функцией \texttt{InlineKeyboardButton()}.

Inline кнопка может либо передавать боту какие-то данные с помощью аргумента \texttt{callback\_data}, либо открывать какую-либо HTML страницу, заданную с помощью аргумента \texttt{url}.

В результате будет список, в котором каждый элемент так же является списком Inline кнопок, которые необходимо объединить в один ряд.

Далее мы рассмотрим несколько примеров ботов с Inline кнопками.

\hypertarget{ux43fux440ux438ux43cux435ux440-ux43fux440ux43eux441ux442ux435ux439ux448ux435ux433ux43e-ux431ux43eux442ux430-ux441-ux43fux43eux434ux434ux435ux440ux436ux43aux43eux439-inline-ux43aux43dux43eux43fux43eux43a}{%
\subsection{Пример простейшего бота с поддержкой InLine кнопок}\label{ux43fux440ux438ux43cux435ux440-ux43fux440ux43eux441ux442ux435ux439ux448ux435ux433ux43e-ux431ux43eux442ux430-ux441-ux43fux43eux434ux434ux435ux440ux436ux43aux43eux439-inline-ux43aux43dux43eux43fux43eux43a}}

Для начала мы напишем бота для экспресс тестирования на covid-19. По команде \texttt{/test}, он будет отправлять вам клавиатуру с двумя кнопками, в зависимости от нажатой кнопки он будет присылать вам сообщение с результатами вашего тестирования.

\begin{Shaded}
\begin{Highlighting}[]
\FunctionTok{library}\NormalTok{(telegram.bot)}

\CommentTok{\# создаём экземпляр класса Updater}
\NormalTok{updater }\OtherTok{\textless{}{-}} \FunctionTok{Updater}\NormalTok{(}\StringTok{\textquotesingle{}ТОКЕН ВАШЕГО БОТА\textquotesingle{}}\NormalTok{)}

\CommentTok{\# метод для отправки InLine клавиатуры}
\NormalTok{test }\OtherTok{\textless{}{-}} \ControlFlowTok{function}\NormalTok{(bot, update) \{}


  \CommentTok{\# создаём InLine клавиатуру}
\NormalTok{  IKM }\OtherTok{\textless{}{-}} \FunctionTok{InlineKeyboardMarkup}\NormalTok{(}
    \AttributeTok{inline\_keyboard =} \FunctionTok{list}\NormalTok{(}
      \FunctionTok{list}\NormalTok{(}
        \FunctionTok{InlineKeyboardButton}\NormalTok{(}\StringTok{"Да"}\NormalTok{, }\AttributeTok{callback\_data =} \StringTok{\textquotesingle{}yes\textquotesingle{}}\NormalTok{),}
        \FunctionTok{InlineKeyboardButton}\NormalTok{(}\StringTok{"Нет"}\NormalTok{, }\AttributeTok{callback\_data =} \StringTok{\textquotesingle{}no\textquotesingle{}}\NormalTok{)}
\NormalTok{      )}
\NormalTok{    )}
\NormalTok{  )}

  \CommentTok{\# Отправляем клавиатуру в чат}
\NormalTok{  bot}\SpecialCharTok{$}\FunctionTok{sendMessage}\NormalTok{(update}\SpecialCharTok{$}\NormalTok{message}\SpecialCharTok{$}\NormalTok{chat\_id,}
                  \AttributeTok{text =} \StringTok{"Вы болете коронавирусом?"}\NormalTok{,}
                  \AttributeTok{reply\_markup =}\NormalTok{ IKM)}
\NormalTok{\}}

\CommentTok{\# метод для обработки нажатия кнопки}
\NormalTok{answer\_cb }\OtherTok{\textless{}{-}} \ControlFlowTok{function}\NormalTok{(bot, update) \{}

  \CommentTok{\# полученные данные с кнопки}
\NormalTok{  data }\OtherTok{\textless{}{-}}\NormalTok{ update}\SpecialCharTok{$}\NormalTok{callback\_query}\SpecialCharTok{$}\NormalTok{data}

  \CommentTok{\# получаем имя пользователя, нажавшего кнопку}
\NormalTok{  uname }\OtherTok{\textless{}{-}}\NormalTok{ update}\SpecialCharTok{$}\FunctionTok{effective\_user}\NormalTok{()}\SpecialCharTok{$}\NormalTok{first\_name}

  \CommentTok{\# обработка результата}
  \ControlFlowTok{if}\NormalTok{ ( data }\SpecialCharTok{==} \StringTok{\textquotesingle{}no\textquotesingle{}}\NormalTok{ ) \{}

\NormalTok{    msg }\OtherTok{\textless{}{-}} \FunctionTok{paste0}\NormalTok{(uname, }\StringTok{", поздравляю, ваш тест на covid{-}19 отрицательный."}\NormalTok{)}

\NormalTok{  \} }\ControlFlowTok{else}\NormalTok{ \{}

\NormalTok{    msg }\OtherTok{\textless{}{-}} \FunctionTok{paste0}\NormalTok{(uname, }\StringTok{", к сожалени ваш тест на covid{-}19 положительный."}\NormalTok{)}

\NormalTok{  \}}


  \CommentTok{\# Отправка сообщения}
\NormalTok{  bot}\SpecialCharTok{$}\FunctionTok{sendMessage}\NormalTok{(}\AttributeTok{chat\_id =}\NormalTok{ update}\SpecialCharTok{$}\FunctionTok{from\_chat\_id}\NormalTok{(),}
                  \AttributeTok{text =}\NormalTok{ msg)}

  \CommentTok{\# сообщаем боту, что запрос с кнопки принят}
\NormalTok{  bot}\SpecialCharTok{$}\FunctionTok{answerCallbackQuery}\NormalTok{(}\AttributeTok{callback\_query\_id =}\NormalTok{ update}\SpecialCharTok{$}\NormalTok{callback\_query}\SpecialCharTok{$}\NormalTok{id)}
\NormalTok{\}}

\CommentTok{\# создаём обработчики}
\NormalTok{inline\_h      }\OtherTok{\textless{}{-}} \FunctionTok{CommandHandler}\NormalTok{(}\StringTok{\textquotesingle{}test\textquotesingle{}}\NormalTok{, test)}
\NormalTok{query\_handler }\OtherTok{\textless{}{-}} \FunctionTok{CallbackQueryHandler}\NormalTok{(answer\_cb)}

\CommentTok{\# добавляем обработчики в диспетчер}
\NormalTok{updater }\OtherTok{\textless{}{-}}\NormalTok{ updater }\SpecialCharTok{+}\NormalTok{ inline\_h }\SpecialCharTok{+}\NormalTok{ query\_handler}

\CommentTok{\# запускаем бота}
\NormalTok{updater}\SpecialCharTok{$}\FunctionTok{start\_polling}\NormalTok{()}
\end{Highlighting}
\end{Shaded}

\begin{quote}
Запустите приведённый выше пример кода, предварительно заменив `ТОКЕН ВАШЕГО БОТА' на реальный токен, который вы получили при создании бота через \emph{BotFather}.
\end{quote}

Результат:
\includegraphics{https://img.netpeak.ua/alsey/159912494522_kiss_227kb.png}

Мы создали два метода:

\begin{itemize}
\tightlist
\item
  \emph{test} - Для отправки в чат Inline клавиатуры
\item
  \emph{answer\_cb} - Для обработки отправленных с клавиатуры данных.
\end{itemize}

Данные, которые будут отправлены с каждой кнопки задаются в аргументе \texttt{callback\_data}, при создании кнопки. Получить отправленные с кнопки данные можно с помощью конструкции \texttt{update\$callback\_query\$data}, внутри метода \emph{answer\_cb}.

Что бы бот реагировал на Inline клавиатуру, метод \emph{answer\_cb} обрабатывается специальным обработчиком: \texttt{CallbackQueryHandler(answer\_cb)}. Который запускает указанный метод по нажатию Inline кнопки. Обработчик \textbf{CallbackQueryHandler} принимает два аргумента:

\begin{itemize}
\tightlist
\item
  \texttt{callback} - Метод который необходимо запустить
\item
  \texttt{pattern} - Фильтр по данным, которые привязаны к кнопке с помощью аргумента \texttt{callback\_data}.
\end{itemize}

Соответвенно с помощью аргумента \texttt{pattern} мы можем под нажатие каждой кнопки написать отдельный метод:

\begin{Shaded}
\begin{Highlighting}[]
\FunctionTok{library}\NormalTok{(telegram.bot)}

\CommentTok{\# создаём экземпляр класса Updater}
\NormalTok{updater }\OtherTok{\textless{}{-}} \FunctionTok{Updater}\NormalTok{(}\StringTok{\textquotesingle{}ТОКЕН ВАШЕГО БОТА\textquotesingle{}}\NormalTok{)}

\CommentTok{\# метод для отправки InLine клавиатуры}
\NormalTok{test }\OtherTok{\textless{}{-}} \ControlFlowTok{function}\NormalTok{(bot, update) \{}

  \CommentTok{\# создаём InLine клавиатуру}
\NormalTok{  IKM }\OtherTok{\textless{}{-}} \FunctionTok{InlineKeyboardMarkup}\NormalTok{(}
    \AttributeTok{inline\_keyboard =} \FunctionTok{list}\NormalTok{(}
      \FunctionTok{list}\NormalTok{(}
        \FunctionTok{InlineKeyboardButton}\NormalTok{(}\StringTok{"Да"}\NormalTok{, }\AttributeTok{callback\_data =} \StringTok{\textquotesingle{}yes\textquotesingle{}}\NormalTok{),}
        \FunctionTok{InlineKeyboardButton}\NormalTok{(}\StringTok{"Нет"}\NormalTok{, }\AttributeTok{callback\_data =} \StringTok{\textquotesingle{}no\textquotesingle{}}\NormalTok{)}
\NormalTok{      )}
\NormalTok{    )}
\NormalTok{  )}

  \CommentTok{\# Отправляем клавиатуру в чат}
\NormalTok{  bot}\SpecialCharTok{$}\FunctionTok{sendMessage}\NormalTok{(update}\SpecialCharTok{$}\NormalTok{message}\SpecialCharTok{$}\NormalTok{chat\_id,}
                  \AttributeTok{text =} \StringTok{"Вы болете коронавирусом?"}\NormalTok{,}
                  \AttributeTok{reply\_markup =}\NormalTok{ IKM)}
\NormalTok{\}}

\CommentTok{\# метод для обработки нажатия кнопки Да}
\NormalTok{answer\_cb\_yes }\OtherTok{\textless{}{-}} \ControlFlowTok{function}\NormalTok{(bot, update) \{}

  \CommentTok{\# получаем имя пользователя, нажавшего кнопку}
\NormalTok{  uname }\OtherTok{\textless{}{-}}\NormalTok{ update}\SpecialCharTok{$}\FunctionTok{effective\_user}\NormalTok{()}\SpecialCharTok{$}\NormalTok{first\_name}

  \CommentTok{\# обработка результата}
\NormalTok{  msg }\OtherTok{\textless{}{-}} \FunctionTok{paste0}\NormalTok{(uname, }\StringTok{", к сожалени ваш текст на covid{-}19 положительный."}\NormalTok{)}

  \CommentTok{\# Отправка сообщения}
\NormalTok{  bot}\SpecialCharTok{$}\FunctionTok{sendMessage}\NormalTok{(}\AttributeTok{chat\_id =}\NormalTok{ update}\SpecialCharTok{$}\FunctionTok{from\_chat\_id}\NormalTok{(),}
                  \AttributeTok{text =}\NormalTok{ msg)}

  \CommentTok{\# сообщаем боту, что запрос с кнопки принят}
\NormalTok{  bot}\SpecialCharTok{$}\FunctionTok{answerCallbackQuery}\NormalTok{(}\AttributeTok{callback\_query\_id =}\NormalTok{ update}\SpecialCharTok{$}\NormalTok{callback\_query}\SpecialCharTok{$}\NormalTok{id)}
\NormalTok{\}}

\CommentTok{\# метод для обработки нажатия кнопки Нет}
\NormalTok{answer\_cb\_no }\OtherTok{\textless{}{-}} \ControlFlowTok{function}\NormalTok{(bot, update) \{}

  \CommentTok{\# получаем имя пользователя, нажавшего кнопку}
\NormalTok{  uname }\OtherTok{\textless{}{-}}\NormalTok{ update}\SpecialCharTok{$}\FunctionTok{effective\_user}\NormalTok{()}\SpecialCharTok{$}\NormalTok{first\_name}

\NormalTok{  msg }\OtherTok{\textless{}{-}} \FunctionTok{paste0}\NormalTok{(uname, }\StringTok{", поздравляю, ваш текст на covid{-}19 отрицательный."}\NormalTok{)}

  \CommentTok{\# Отправка сообщения}
\NormalTok{  bot}\SpecialCharTok{$}\FunctionTok{sendMessage}\NormalTok{(}\AttributeTok{chat\_id =}\NormalTok{ update}\SpecialCharTok{$}\FunctionTok{from\_chat\_id}\NormalTok{(),}
                  \AttributeTok{text =}\NormalTok{ msg)}

  \CommentTok{\# сообщаем боту, что запрос с кнопки принят}
\NormalTok{  bot}\SpecialCharTok{$}\FunctionTok{answerCallbackQuery}\NormalTok{(}\AttributeTok{callback\_query\_id =}\NormalTok{ update}\SpecialCharTok{$}\NormalTok{callback\_query}\SpecialCharTok{$}\NormalTok{id)}
\NormalTok{\}}

\CommentTok{\# создаём обработчики}
\NormalTok{inline\_h          }\OtherTok{\textless{}{-}} \FunctionTok{CommandHandler}\NormalTok{(}\StringTok{\textquotesingle{}test\textquotesingle{}}\NormalTok{, test)}
\NormalTok{query\_handler\_yes }\OtherTok{\textless{}{-}} \FunctionTok{CallbackQueryHandler}\NormalTok{(answer\_cb\_yes, }\AttributeTok{pattern =} \StringTok{\textquotesingle{}yes\textquotesingle{}}\NormalTok{)}
\NormalTok{query\_handler\_no  }\OtherTok{\textless{}{-}} \FunctionTok{CallbackQueryHandler}\NormalTok{(answer\_cb\_no, }\AttributeTok{pattern =} \StringTok{\textquotesingle{}no\textquotesingle{}}\NormalTok{)}

\CommentTok{\# добавляем обработчики в диспетчер}
\NormalTok{updater }\OtherTok{\textless{}{-}}\NormalTok{ updater }\SpecialCharTok{+}
\NormalTok{            inline\_h }\SpecialCharTok{+}
\NormalTok{            query\_handler\_yes }\SpecialCharTok{+}
\NormalTok{            query\_handler\_no}

\CommentTok{\# запускаем бота}
\NormalTok{updater}\SpecialCharTok{$}\FunctionTok{start\_polling}\NormalTok{()}
\end{Highlighting}
\end{Shaded}

\begin{quote}
Запустите приведённый выше пример кода, предварительно заменив `ТОКЕН ВАШЕГО БОТА' на реальный токен, который вы получили при создании бота через \emph{BotFather}.
\end{quote}

Теперь мы написали 2 отдельных метода, т.е. по одному методу, под нажатие каждой кнопки, и использовали аргумент \texttt{pattern}, при создании их обработчиков:

\begin{Shaded}
\begin{Highlighting}[]
\NormalTok{query\_handler\_yes }\OtherTok{\textless{}{-}} \FunctionTok{CallbackQueryHandler}\NormalTok{(answer\_cb\_yes, }\AttributeTok{pattern =} \StringTok{\textquotesingle{}yes\textquotesingle{}}\NormalTok{)}
\NormalTok{query\_handler\_no  }\OtherTok{\textless{}{-}} \FunctionTok{CallbackQueryHandler}\NormalTok{(answer\_cb\_no, }\AttributeTok{pattern =} \StringTok{\textquotesingle{}no\textquotesingle{}}\NormalTok{)}
\end{Highlighting}
\end{Shaded}

Заканчивается код метода \emph{answer\_cb} командой \texttt{bot\$answerCallbackQuery(callback\_query\_id\ =\ update\$callback\_query\$id)}, которая сообщает боту, что данные с inline клавиатуры получены.

\hypertarget{ux43fux440ux438ux43cux435ux440-ux431ux43eux442ux430-ux43aux43eux442ux43eux440ux44bux439-ux441ux43eux43eux431ux449ux430ux435ux442-ux442ux435ux43aux443ux449ux443ux44e-ux43fux43eux433ux43eux434ux443-ux43fux43e-ux432ux44bux431ux440ux430ux43dux43dux43eux43cux443-ux433ux43eux440ux43eux434ux443}{%
\subsection{Пример бота, который сообщает текущую погоду по выбранному городу}\label{ux43fux440ux438ux43cux435ux440-ux431ux43eux442ux430-ux43aux43eux442ux43eux440ux44bux439-ux441ux43eux43eux431ux449ux430ux435ux442-ux442ux435ux43aux443ux449ux443ux44e-ux43fux43eux433ux43eux434ux443-ux43fux43e-ux432ux44bux431ux440ux430ux43dux43dux43eux43cux443-ux433ux43eux440ux43eux434ux443}}

Давайте попробуем написать бота, который запрашивает данные о погоде.

Логика его работы будет следующая. Изначально командой \texttt{/start} вы вызываете основную клавиатуру, в которой присутствует всего одна кнопка ``Погода''. Нажав на эту кнопку вы получаете сообщение с Inline клавиатурой, для выбора города, по которому требуется узнать текущую погоду. Выбираете один из городов, и получаете текущую погоду.

В этом примере кода мы будем использовать несколько дополнительных пакетов:

\begin{itemize}
\tightlist
\item
  \texttt{httr} - пакет для работы с HTTP запросами, на основе которых построена работа с любым API. В нашем случае мы будем использовать бесплатный API \href{https://openweathermap.org/api}{openweathermap.org}.
\item
  \texttt{stringr} - пакет для работы с текстом, в нашем случае мы будем его использовать для формирования сообщения о погоде в выбранном городе.
\end{itemize}

\emph{Код бота, который сообщает текущую погоду по выбранному городу}

\begin{Shaded}
\begin{Highlighting}[]
\FunctionTok{library}\NormalTok{(telegram.bot)}
\FunctionTok{library}\NormalTok{(httr)}
\FunctionTok{library}\NormalTok{(stringr)}

\CommentTok{\# создаём экземпляр класса Updater}
\NormalTok{updater }\OtherTok{\textless{}{-}} \FunctionTok{Updater}\NormalTok{(}\StringTok{\textquotesingle{}ТОКЕН ВАШЕГО БОТА\textquotesingle{}}\NormalTok{)}

\CommentTok{\# создаём методы}
\DocumentationTok{\#\# метод для запуска основной клавиатуры}
\NormalTok{start }\OtherTok{\textless{}{-}} \ControlFlowTok{function}\NormalTok{(bot, update) \{}

  \CommentTok{\# создаём клавиатуру}
\NormalTok{  RKM }\OtherTok{\textless{}{-}} \FunctionTok{ReplyKeyboardMarkup}\NormalTok{(}
    \AttributeTok{keyboard =} \FunctionTok{list}\NormalTok{(}
      \FunctionTok{list}\NormalTok{(}
        \FunctionTok{KeyboardButton}\NormalTok{(}\StringTok{"Погода"}\NormalTok{)}
\NormalTok{      )}
\NormalTok{    ),}
    \AttributeTok{resize\_keyboard =} \ConstantTok{TRUE}\NormalTok{,}
    \AttributeTok{one\_time\_keyboard =} \ConstantTok{TRUE}
\NormalTok{  )}

  \CommentTok{\# отправляем клавиатуру}
\NormalTok{  bot}\SpecialCharTok{$}\FunctionTok{sendMessage}\NormalTok{(update}\SpecialCharTok{$}\NormalTok{message}\SpecialCharTok{$}\NormalTok{chat\_id,}
                  \AttributeTok{text =} \StringTok{\textquotesingle{}Выберите команду\textquotesingle{}}\NormalTok{,}
                  \AttributeTok{reply\_markup =}\NormalTok{ RKM)}

\NormalTok{\}}

\DocumentationTok{\#\# Метод вызова Inine клавиатуры}
\NormalTok{weather }\OtherTok{\textless{}{-}} \ControlFlowTok{function}\NormalTok{(bot, update) \{}

\NormalTok{  IKM }\OtherTok{\textless{}{-}} \FunctionTok{InlineKeyboardMarkup}\NormalTok{(}
    \AttributeTok{inline\_keyboard =} \FunctionTok{list}\NormalTok{(}
      \FunctionTok{list}\NormalTok{(}
        \FunctionTok{InlineKeyboardButton}\NormalTok{(}\AttributeTok{text =} \StringTok{\textquotesingle{}Москва\textquotesingle{}}\NormalTok{, }\AttributeTok{callback\_data =} \StringTok{\textquotesingle{}New York,us\textquotesingle{}}\NormalTok{),}
        \FunctionTok{InlineKeyboardButton}\NormalTok{(}\AttributeTok{text =} \StringTok{\textquotesingle{}Санкт{-}Петербург\textquotesingle{}}\NormalTok{, }\AttributeTok{callback\_data =} \StringTok{\textquotesingle{}Saint Petersburg\textquotesingle{}}\NormalTok{),}
        \FunctionTok{InlineKeyboardButton}\NormalTok{(}\AttributeTok{text =} \StringTok{\textquotesingle{}Нью{-}Йорк\textquotesingle{}}\NormalTok{, }\AttributeTok{callback\_data =} \StringTok{\textquotesingle{}New York\textquotesingle{}}\NormalTok{)}
\NormalTok{      ),}
      \FunctionTok{list}\NormalTok{(}
        \FunctionTok{InlineKeyboardButton}\NormalTok{(}\AttributeTok{text =} \StringTok{\textquotesingle{}Екатеринбург\textquotesingle{}}\NormalTok{, }\AttributeTok{callback\_data =} \StringTok{\textquotesingle{}Yekaterinburg,ru\textquotesingle{}}\NormalTok{),}
        \FunctionTok{InlineKeyboardButton}\NormalTok{(}\AttributeTok{text =} \StringTok{\textquotesingle{}Берлин\textquotesingle{}}\NormalTok{, }\AttributeTok{callback\_data =} \StringTok{\textquotesingle{}Berlin,de\textquotesingle{}}\NormalTok{),}
        \FunctionTok{InlineKeyboardButton}\NormalTok{(}\AttributeTok{text =} \StringTok{\textquotesingle{}Париж\textquotesingle{}}\NormalTok{, }\AttributeTok{callback\_data =} \StringTok{\textquotesingle{}Paris,fr\textquotesingle{}}\NormalTok{)}
\NormalTok{      ),}
      \FunctionTok{list}\NormalTok{(}
        \FunctionTok{InlineKeyboardButton}\NormalTok{(}\AttributeTok{text =} \StringTok{\textquotesingle{}Рим\textquotesingle{}}\NormalTok{, }\AttributeTok{callback\_data =} \StringTok{\textquotesingle{}Rome,it\textquotesingle{}}\NormalTok{),}
        \FunctionTok{InlineKeyboardButton}\NormalTok{(}\AttributeTok{text =} \StringTok{\textquotesingle{}Одесса\textquotesingle{}}\NormalTok{, }\AttributeTok{callback\_data =} \StringTok{\textquotesingle{}Odessa,ua\textquotesingle{}}\NormalTok{),}
        \FunctionTok{InlineKeyboardButton}\NormalTok{(}\AttributeTok{text =} \StringTok{\textquotesingle{}Киев\textquotesingle{}}\NormalTok{, }\AttributeTok{callback\_data =} \StringTok{\textquotesingle{}Kyiv,ua\textquotesingle{}}\NormalTok{)}
\NormalTok{      ),}
      \FunctionTok{list}\NormalTok{(}
        \FunctionTok{InlineKeyboardButton}\NormalTok{(}\AttributeTok{text =} \StringTok{\textquotesingle{}Токио\textquotesingle{}}\NormalTok{, }\AttributeTok{callback\_data =} \StringTok{\textquotesingle{}Tokyo\textquotesingle{}}\NormalTok{),}
        \FunctionTok{InlineKeyboardButton}\NormalTok{(}\AttributeTok{text =} \StringTok{\textquotesingle{}Амстердам\textquotesingle{}}\NormalTok{, }\AttributeTok{callback\_data =} \StringTok{\textquotesingle{}Amsterdam,nl\textquotesingle{}}\NormalTok{),}
        \FunctionTok{InlineKeyboardButton}\NormalTok{(}\AttributeTok{text =} \StringTok{\textquotesingle{}Вашингтон\textquotesingle{}}\NormalTok{, }\AttributeTok{callback\_data =} \StringTok{\textquotesingle{}Washington,us\textquotesingle{}}\NormalTok{)}
\NormalTok{      )}
\NormalTok{    )}
\NormalTok{  )}

  \CommentTok{\# Send Inline Keyboard}
\NormalTok{  bot}\SpecialCharTok{$}\FunctionTok{sendMessage}\NormalTok{(}\AttributeTok{chat\_id =}\NormalTok{ update}\SpecialCharTok{$}\NormalTok{message}\SpecialCharTok{$}\NormalTok{chat\_id,}
                  \AttributeTok{text =} \StringTok{"Выберите город"}\NormalTok{,}
                  \AttributeTok{reply\_markup =}\NormalTok{ IKM)}
\NormalTok{\}}

\CommentTok{\# метод для сообщения погоды}
\NormalTok{answer\_cb }\OtherTok{\textless{}{-}} \ControlFlowTok{function}\NormalTok{(bot, update) \{}

  \CommentTok{\# получаем из сообщения город}
\NormalTok{  city }\OtherTok{\textless{}{-}}\NormalTok{ update}\SpecialCharTok{$}\NormalTok{callback\_query}\SpecialCharTok{$}\NormalTok{data}

  \CommentTok{\# отправляем запрос}
\NormalTok{  ans }\OtherTok{\textless{}{-}} \FunctionTok{GET}\NormalTok{(}\StringTok{\textquotesingle{}https://api.openweathermap.org/data/2.5/weather\textquotesingle{}}\NormalTok{,}
             \AttributeTok{query =} \FunctionTok{list}\NormalTok{(}\AttributeTok{q     =}\NormalTok{ city,}
                          \AttributeTok{lang  =} \StringTok{\textquotesingle{}ru\textquotesingle{}}\NormalTok{,}
                          \AttributeTok{units =} \StringTok{\textquotesingle{}metric\textquotesingle{}}\NormalTok{,}
                          \AttributeTok{appid =} \StringTok{\textquotesingle{}4776568ccea136ffe4cda9f1969af340\textquotesingle{}}\NormalTok{))}

  \CommentTok{\# парсим ответ}
\NormalTok{  result }\OtherTok{\textless{}{-}} \FunctionTok{content}\NormalTok{(ans)}

  \CommentTok{\# формируем сообщение}
\NormalTok{  msg }\OtherTok{\textless{}{-}} \FunctionTok{str\_glue}\NormalTok{(}\StringTok{"\{result$name\} погода:}\SpecialCharTok{\textbackslash{}n}\StringTok{"}\NormalTok{,}
                  \StringTok{"Текущая температура: \{result$main$temp\}}\SpecialCharTok{\textbackslash{}n}\StringTok{"}\NormalTok{,}
                  \StringTok{"Скорость ветра: \{result$wind$speed\}}\SpecialCharTok{\textbackslash{}n}\StringTok{"}\NormalTok{,}
                  \StringTok{"Описание: \{result$weather[[1]]$description\}"}\NormalTok{)}

  \CommentTok{\# отправляем информацию о погоде}
\NormalTok{  bot}\SpecialCharTok{$}\FunctionTok{sendMessage}\NormalTok{(}\AttributeTok{chat\_id =}\NormalTok{ update}\SpecialCharTok{$}\FunctionTok{from\_chat\_id}\NormalTok{(),}
                  \AttributeTok{text    =}\NormalTok{ msg)}


\NormalTok{  bot}\SpecialCharTok{$}\FunctionTok{answerCallbackQuery}\NormalTok{(}\AttributeTok{callback\_query\_id =}\NormalTok{ update}\SpecialCharTok{$}\NormalTok{callback\_query}\SpecialCharTok{$}\NormalTok{id)}
\NormalTok{\}}

\CommentTok{\# создаём фильтры}
\DocumentationTok{\#\# сообщения с текстом Погода}
\NormalTok{MessageFilters}\SpecialCharTok{$}\NormalTok{weather }\OtherTok{\textless{}{-}} \FunctionTok{BaseFilter}\NormalTok{(}\ControlFlowTok{function}\NormalTok{(message) \{}

  \CommentTok{\# проверяем текст сообщения}
\NormalTok{  message}\SpecialCharTok{$}\NormalTok{text }\SpecialCharTok{==} \StringTok{"Погода"}

\NormalTok{\}}
\NormalTok{)}

\CommentTok{\# создаём обработчики}
\NormalTok{h\_start         }\OtherTok{\textless{}{-}} \FunctionTok{CommandHandler}\NormalTok{(}\StringTok{\textquotesingle{}start\textquotesingle{}}\NormalTok{, start)}
\NormalTok{h\_weather       }\OtherTok{\textless{}{-}} \FunctionTok{MessageHandler}\NormalTok{(weather, }\AttributeTok{filters =}\NormalTok{ MessageFilters}\SpecialCharTok{$}\NormalTok{weather)}
\NormalTok{h\_query\_handler }\OtherTok{\textless{}{-}} \FunctionTok{CallbackQueryHandler}\NormalTok{(answer\_cb)}

\CommentTok{\# добавляем обработчики в диспетчер}
\NormalTok{updater }\OtherTok{\textless{}{-}}\NormalTok{ updater }\SpecialCharTok{+}
\NormalTok{              h\_start }\SpecialCharTok{+}
\NormalTok{              h\_weather }\SpecialCharTok{+}
\NormalTok{              h\_query\_handler}

\CommentTok{\# запускаем бота}
\NormalTok{updater}\SpecialCharTok{$}\FunctionTok{start\_polling}\NormalTok{()}
\end{Highlighting}
\end{Shaded}

\begin{quote}
Запустите приведённый выше пример кода, предварительно заменив `ТОКЕН ВАШЕГО БОТА' на реальный токен, который вы получили при создании бота через \emph{BotFather}.
\end{quote}

В результате наш бот будет работать примерно так:
\includegraphics{https://img.netpeak.ua/alsey/159863902887_kiss_184kb.png}

Схематически данного бота можно изобрать вот так:
\includegraphics{http://img.netpeak.ua/alsey/159906484732_kiss_23kb.png}

Мы создали 3 метода, доступные внутри нашего погодного бота:

\begin{itemize}
\tightlist
\item
  \emph{start} - Запуск основной клавиатуры бота
\item
  \emph{weather} - Запуск Inline клавиатуры для выбора города
\item
  \emph{answer\_cb} - Основной метод, который по заданному городу запрашивает в API погоду, и отправляет её в чат.
\end{itemize}

Метод \emph{start} у нас запускается командой \texttt{/start}, что реализовано обработчиком \texttt{CommandHandler(\textquotesingle{}start\textquotesingle{},\ start)}.

Для запуска метода \emph{weather} мы создали одноимённый фильтр:

\begin{Shaded}
\begin{Highlighting}[]
\CommentTok{\# создаём фильтры}
\DocumentationTok{\#\# сообщения с текстом Погода}
\NormalTok{MessageFilters}\SpecialCharTok{$}\NormalTok{weather }\OtherTok{\textless{}{-}} \FunctionTok{BaseFilter}\NormalTok{(}\ControlFlowTok{function}\NormalTok{(message) \{}

  \CommentTok{\# проверяем текст сообщения}
\NormalTok{  message}\SpecialCharTok{$}\NormalTok{text }\SpecialCharTok{==} \StringTok{"Погода"}

\NormalTok{\}}
\NormalTok{)}
\end{Highlighting}
\end{Shaded}

И вызываем этот метод следующим обработчиком сообщений: \texttt{MessageHandler(weather,\ filters\ =\ MessageFilters\$weather)}.

И в конце концов, основной наш метод \emph{answer\_cb} реагирует на нажатие Inline кнопок, что реализовано специальным обработчиком: \texttt{CallbackQueryHandler(answer\_cb)}.

Внутри метода \emph{answer\_cb}, мы считываем отправленные с клавиатуры данные и записываем их в переменную \texttt{city}: \texttt{city\ \textless{}-\ update\$callback\_query\$data}. После чего запрашиваем из API данные о погоде, формируем и отправляем сообщение, и в конце концов используем метод \texttt{answerCallbackQuery} для того, что бы сообщить боту, о том, что мы обработали нажатие Inline кнопки.

\hypertarget{ux43fux440ux438ux43cux435ux440-ux431ux43eux442ux430-ux43aux43eux442ux43eux440ux44bux439-ux432ux44bux432ux43eux434ux438ux442-ux441ux43fux438ux441ux43eux43a-ux441ux430ux43cux44bux445-ux441ux432ux435ux436ux438ux445-ux441ux442ux430ux442ux435ux439-ux441ux43e-ux441ux441ux44bux43bux43aux430ux43cux438-ux43fux43e-ux443ux43aux430ux437ux430ux43dux43dux43eux43cux443-ux445ux430ux431ux443-ux438ux437-habr.com.}{%
\subsection{\texorpdfstring{Пример бота, который выводит список самых свежих статей со ссылками по-указанному Хабу из \href{https://habr.com/}{habr.com}.}{Пример бота, который выводит список самых свежих статей со ссылками по-указанному Хабу из habr.com.}}\label{ux43fux440ux438ux43cux435ux440-ux431ux43eux442ux430-ux43aux43eux442ux43eux440ux44bux439-ux432ux44bux432ux43eux434ux438ux442-ux441ux43fux438ux441ux43eux43a-ux441ux430ux43cux44bux445-ux441ux432ux435ux436ux438ux445-ux441ux442ux430ux442ux435ux439-ux441ux43e-ux441ux441ux44bux43bux43aux430ux43cux438-ux43fux43e-ux443ux43aux430ux437ux430ux43dux43dux43eux43cux443-ux445ux430ux431ux443-ux438ux437-habr.com.}}

Данного бота я привожу для того, что бы показать вам, как вывести Inline кнопки которые ведут на веб страницы.

Логика данного бота схожа с предыдущим, изначально мы запускаем основную клавиатуру командой \texttt{/start}. Далее бот даёт нам на выбор список из 6 хабов, мы выбираем интересующий нас хаб, и получаем 5 самых свежих публикаций из выбранного Хаба.

Как вы понимаете, в данном случае нам необходимо получить список статей, и для этого мы будем использовать специальный пакет \texttt{habR}, который позволяет запрашивать из хабры статьи и некоторую статистику по ним в R.

Установить пакет \texttt{habR} можно только из github, для чего вам понадобится дополнительный пакет \texttt{devtools}. Для установки воспользуйтесь приведённым ниже кодом.

\begin{Shaded}
\begin{Highlighting}[]
\FunctionTok{install.packages}\NormalTok{(}\StringTok{\textquotesingle{}devtools\textquotesingle{}}\NormalTok{)}
\NormalTok{devtools}\SpecialCharTok{::}\FunctionTok{install\_github}\NormalTok{(}\StringTok{\textquotesingle{}selesnow/habR\textquotesingle{}}\NormalTok{)}
\end{Highlighting}
\end{Shaded}

Теперь рассмотрим код построения описанного выше бота:

\emph{Код бот который выводит список наиболее свежих статей по выбранному Хабу}

\begin{Shaded}
\begin{Highlighting}[]
\FunctionTok{library}\NormalTok{(telegram.bot)}
\FunctionTok{library}\NormalTok{(habR)}

\CommentTok{\# создаём экземпляр класса Updater}
\NormalTok{updater }\OtherTok{\textless{}{-}} \FunctionTok{Updater}\NormalTok{(}\StringTok{\textquotesingle{}ТОКЕН ВАШЕГО БОТА\textquotesingle{}}\NormalTok{)}

\CommentTok{\# создаём методы}
\DocumentationTok{\#\# метод для запуска основной клавиатуры}
\NormalTok{start }\OtherTok{\textless{}{-}} \ControlFlowTok{function}\NormalTok{(bot, update) \{}

  \CommentTok{\# создаём клавиатуру}
\NormalTok{  RKM }\OtherTok{\textless{}{-}} \FunctionTok{ReplyKeyboardMarkup}\NormalTok{(}
    \AttributeTok{keyboard =} \FunctionTok{list}\NormalTok{(}
      \FunctionTok{list}\NormalTok{(}
        \FunctionTok{KeyboardButton}\NormalTok{(}\StringTok{"Список статей"}\NormalTok{)}
\NormalTok{      )}
\NormalTok{    ),}
    \AttributeTok{resize\_keyboard =} \ConstantTok{TRUE}\NormalTok{,}
    \AttributeTok{one\_time\_keyboard =} \ConstantTok{TRUE}
\NormalTok{  )}

  \CommentTok{\# отправляем клавиатуру}
\NormalTok{  bot}\SpecialCharTok{$}\FunctionTok{sendMessage}\NormalTok{(update}\SpecialCharTok{$}\NormalTok{message}\SpecialCharTok{$}\NormalTok{chat\_id,}
                  \AttributeTok{text =} \StringTok{\textquotesingle{}Выберите команду\textquotesingle{}}\NormalTok{,}
                  \AttributeTok{reply\_markup =}\NormalTok{ RKM)}

\NormalTok{\}}

\DocumentationTok{\#\# Метод вызова Inine клавиатуры}
\NormalTok{habs }\OtherTok{\textless{}{-}} \ControlFlowTok{function}\NormalTok{(bot, update) \{}

\NormalTok{  IKM }\OtherTok{\textless{}{-}} \FunctionTok{InlineKeyboardMarkup}\NormalTok{(}
    \AttributeTok{inline\_keyboard =} \FunctionTok{list}\NormalTok{(}
      \FunctionTok{list}\NormalTok{(}
        \FunctionTok{InlineKeyboardButton}\NormalTok{(}\AttributeTok{text =} \StringTok{\textquotesingle{}R\textquotesingle{}}\NormalTok{, }\AttributeTok{callback\_data =} \StringTok{\textquotesingle{}R\textquotesingle{}}\NormalTok{),}
        \FunctionTok{InlineKeyboardButton}\NormalTok{(}\AttributeTok{text =} \StringTok{\textquotesingle{}Data Mining\textquotesingle{}}\NormalTok{, }\AttributeTok{callback\_data =} \StringTok{\textquotesingle{}data\_mining\textquotesingle{}}\NormalTok{),}
        \FunctionTok{InlineKeyboardButton}\NormalTok{(}\AttributeTok{text =} \StringTok{\textquotesingle{}Data Engineering\textquotesingle{}}\NormalTok{, }\AttributeTok{callback\_data =} \StringTok{\textquotesingle{}data\_engineering\textquotesingle{}}\NormalTok{)}
\NormalTok{      ),}
      \FunctionTok{list}\NormalTok{(}
        \FunctionTok{InlineKeyboardButton}\NormalTok{(}\AttributeTok{text =} \StringTok{\textquotesingle{}Big Data\textquotesingle{}}\NormalTok{, }\AttributeTok{callback\_data =} \StringTok{\textquotesingle{}bigdata\textquotesingle{}}\NormalTok{),}
        \FunctionTok{InlineKeyboardButton}\NormalTok{(}\AttributeTok{text =} \StringTok{\textquotesingle{}Python\textquotesingle{}}\NormalTok{, }\AttributeTok{callback\_data =} \StringTok{\textquotesingle{}python\textquotesingle{}}\NormalTok{),}
        \FunctionTok{InlineKeyboardButton}\NormalTok{(}\AttributeTok{text =} \StringTok{\textquotesingle{}Визуализация данных\textquotesingle{}}\NormalTok{, }\AttributeTok{callback\_data =} \StringTok{\textquotesingle{}data\_visualization\textquotesingle{}}\NormalTok{)}
\NormalTok{      )}
\NormalTok{    )}
\NormalTok{  )}

  \CommentTok{\# Send Inline Keyboard}
\NormalTok{  bot}\SpecialCharTok{$}\FunctionTok{sendMessage}\NormalTok{(}\AttributeTok{chat\_id =}\NormalTok{ update}\SpecialCharTok{$}\NormalTok{message}\SpecialCharTok{$}\NormalTok{chat\_id,}
                  \AttributeTok{text =} \StringTok{"Выберите Хаб"}\NormalTok{,}
                  \AttributeTok{reply\_markup =}\NormalTok{ IKM)}
\NormalTok{\}}

\CommentTok{\# метод для сообщения погоды}
\NormalTok{answer\_cb }\OtherTok{\textless{}{-}} \ControlFlowTok{function}\NormalTok{(bot, update) \{}

  \CommentTok{\# получаем из сообщения город}
\NormalTok{  hub }\OtherTok{\textless{}{-}}\NormalTok{ update}\SpecialCharTok{$}\NormalTok{callback\_query}\SpecialCharTok{$}\NormalTok{data}

  \CommentTok{\# сообщение о том, что данные по кнопке получены}
\NormalTok{  bot}\SpecialCharTok{$}\FunctionTok{answerCallbackQuery}\NormalTok{(}\AttributeTok{callback\_query\_id =}\NormalTok{ update}\SpecialCharTok{$}\NormalTok{callback\_query}\SpecialCharTok{$}\NormalTok{id,}
                          \AttributeTok{text =} \StringTok{\textquotesingle{}Подождите несколько минут, запрос обрабатывается\textquotesingle{}}\NormalTok{)}

  \CommentTok{\# сообщение о том, что надо подождать пока бот получит данные}
\NormalTok{  mid }\OtherTok{\textless{}{-}}\NormalTok{ bot}\SpecialCharTok{$}\FunctionTok{sendMessage}\NormalTok{(}\AttributeTok{chat\_id =}\NormalTok{ update}\SpecialCharTok{$}\FunctionTok{from\_chat\_id}\NormalTok{(),}
                         \AttributeTok{text    =} \StringTok{"Подождите несколько минут пока, я соберу данные по выбранному Хабу"}\NormalTok{)}

  \CommentTok{\# парсим Хабр}
\NormalTok{  posts }\OtherTok{\textless{}{-}} \FunctionTok{head}\NormalTok{(}\FunctionTok{habr\_hub\_posts}\NormalTok{(hub, }\DecValTok{1}\NormalTok{), }\DecValTok{5}\NormalTok{)}

  \CommentTok{\# удаляем сообщение о том, что надо подождать}
\NormalTok{  bot}\SpecialCharTok{$}\FunctionTok{deleteMessage}\NormalTok{(update}\SpecialCharTok{$}\FunctionTok{from\_chat\_id}\NormalTok{(), mid}\SpecialCharTok{$}\NormalTok{message\_id)}

  \CommentTok{\# формируем список кнопок}
\NormalTok{  keys }\OtherTok{\textless{}{-}} \FunctionTok{lapply}\NormalTok{(}\DecValTok{1}\SpecialCharTok{:}\DecValTok{5}\NormalTok{, }\ControlFlowTok{function}\NormalTok{(x) }\FunctionTok{list}\NormalTok{(}\FunctionTok{InlineKeyboardButton}\NormalTok{(posts}\SpecialCharTok{$}\NormalTok{title[x], }\AttributeTok{url =}\NormalTok{ posts}\SpecialCharTok{$}\NormalTok{link[x])))}

  \CommentTok{\# формируем клавиатуру}
\NormalTok{  IKM }\OtherTok{\textless{}{-}} \FunctionTok{InlineKeyboardMarkup}\NormalTok{(}
    \AttributeTok{inline\_keyboard =}\NormalTok{  keys}
\NormalTok{    )}

  \CommentTok{\# отправляем информацию о погоде}
\NormalTok{  bot}\SpecialCharTok{$}\FunctionTok{sendMessage}\NormalTok{(}\AttributeTok{chat\_id =}\NormalTok{ update}\SpecialCharTok{$}\FunctionTok{from\_chat\_id}\NormalTok{(),}
                  \AttributeTok{text    =} \FunctionTok{paste0}\NormalTok{(}\StringTok{"5 наиболее свежих статей из Хаба "}\NormalTok{, hub),}
                  \AttributeTok{reply\_markup =}\NormalTok{ IKM)}

\NormalTok{\}}

\CommentTok{\# создаём фильтры}
\DocumentationTok{\#\# сообщения с текстом Погода}
\NormalTok{MessageFilters}\SpecialCharTok{$}\NormalTok{hubs }\OtherTok{\textless{}{-}} \FunctionTok{BaseFilter}\NormalTok{(}\ControlFlowTok{function}\NormalTok{(message) \{}

  \CommentTok{\# проверяем текст сообщения}
\NormalTok{  message}\SpecialCharTok{$}\NormalTok{text }\SpecialCharTok{==} \StringTok{"Список статей"}

\NormalTok{\}}
\NormalTok{)}

\CommentTok{\# создаём обработчики}
\NormalTok{h\_start         }\OtherTok{\textless{}{-}} \FunctionTok{CommandHandler}\NormalTok{(}\StringTok{\textquotesingle{}start\textquotesingle{}}\NormalTok{, start)}
\NormalTok{h\_hubs          }\OtherTok{\textless{}{-}} \FunctionTok{MessageHandler}\NormalTok{(habs, }\AttributeTok{filters =}\NormalTok{ MessageFilters}\SpecialCharTok{$}\NormalTok{hubs)}
\NormalTok{h\_query\_handler }\OtherTok{\textless{}{-}} \FunctionTok{CallbackQueryHandler}\NormalTok{(answer\_cb)}

\CommentTok{\# добавляем обработчики в диспетчер}
\NormalTok{updater }\OtherTok{\textless{}{-}}\NormalTok{ updater }\SpecialCharTok{+}
\NormalTok{  h\_start }\SpecialCharTok{+}
\NormalTok{  h\_hubs  }\SpecialCharTok{+}
\NormalTok{  h\_query\_handler}

\CommentTok{\# запускаем бота}
\NormalTok{updater}\SpecialCharTok{$}\FunctionTok{start\_polling}\NormalTok{()}
\end{Highlighting}
\end{Shaded}

\begin{quote}
Запустите приведённый выше пример кода, предварительно заменив `ТОКЕН ВАШЕГО БОТА' на реальный токен, который вы получили при создании бота через \emph{BotFather}.
\end{quote}

В итоге мы получим вот такой результат:
\includegraphics{https://img.netpeak.ua/alsey/159905964234_kiss_178kb.png}

Список доступных для выбора Хабов мы вбили хардкодом, в методе \texttt{habs}:

\begin{Shaded}
\begin{Highlighting}[]
\DocumentationTok{\#\# Метод вызова Inine клавиатуры}
\NormalTok{habs }\OtherTok{\textless{}{-}} \ControlFlowTok{function}\NormalTok{(bot, update) \{}

\NormalTok{  IKM }\OtherTok{\textless{}{-}} \FunctionTok{InlineKeyboardMarkup}\NormalTok{(}
    \AttributeTok{inline\_keyboard =} \FunctionTok{list}\NormalTok{(}
      \FunctionTok{list}\NormalTok{(}
        \FunctionTok{InlineKeyboardButton}\NormalTok{(}\AttributeTok{text =} \StringTok{\textquotesingle{}R\textquotesingle{}}\NormalTok{, }\AttributeTok{callback\_data =} \StringTok{\textquotesingle{}r\textquotesingle{}}\NormalTok{),}
        \FunctionTok{InlineKeyboardButton}\NormalTok{(}\AttributeTok{text =} \StringTok{\textquotesingle{}Data Mining\textquotesingle{}}\NormalTok{, }\AttributeTok{callback\_data =} \StringTok{\textquotesingle{}data\_mining\textquotesingle{}}\NormalTok{),}
        \FunctionTok{InlineKeyboardButton}\NormalTok{(}\AttributeTok{text =} \StringTok{\textquotesingle{}Data Engineering\textquotesingle{}}\NormalTok{, }\AttributeTok{callback\_data =} \StringTok{\textquotesingle{}data\_engineering\textquotesingle{}}\NormalTok{)}
\NormalTok{      ),}
      \FunctionTok{list}\NormalTok{(}
        \FunctionTok{InlineKeyboardButton}\NormalTok{(}\AttributeTok{text =} \StringTok{\textquotesingle{}Big Data\textquotesingle{}}\NormalTok{, }\AttributeTok{callback\_data =} \StringTok{\textquotesingle{}bigdata\textquotesingle{}}\NormalTok{),}
        \FunctionTok{InlineKeyboardButton}\NormalTok{(}\AttributeTok{text =} \StringTok{\textquotesingle{}Python\textquotesingle{}}\NormalTok{, }\AttributeTok{callback\_data =} \StringTok{\textquotesingle{}python\textquotesingle{}}\NormalTok{),}
        \FunctionTok{InlineKeyboardButton}\NormalTok{(}\AttributeTok{text =} \StringTok{\textquotesingle{}Визуализация данных\textquotesingle{}}\NormalTok{, }\AttributeTok{callback\_data =} \StringTok{\textquotesingle{}data\_visualization\textquotesingle{}}\NormalTok{)}
\NormalTok{      )}
\NormalTok{    )}
\NormalTok{  )}

  \CommentTok{\# Send Inline Keyboard}
\NormalTok{  bot}\SpecialCharTok{$}\FunctionTok{sendMessage}\NormalTok{(}\AttributeTok{chat\_id =}\NormalTok{ update}\SpecialCharTok{$}\NormalTok{message}\SpecialCharTok{$}\NormalTok{chat\_id,}
                  \AttributeTok{text =} \StringTok{"Выберите Хаб"}\NormalTok{,}
                  \AttributeTok{reply\_markup =}\NormalTok{ IKM)}
\NormalTok{\}}
\end{Highlighting}
\end{Shaded}

Список статей из указанного Хаба мы получаем командой \texttt{habr\_hub\_posts()}, из пакета \texttt{habR}. При этом указываем, что нам не требуется список статей за всё время, а только первая страница на которой располагаются 20 статей. Из полученной таблицы с помощью команды \texttt{head()} оставляем только 5 самых верхних, которые и являются самыми свежими статьями.

\begin{Shaded}
\begin{Highlighting}[]
  \CommentTok{\# парсим Хабр}
\NormalTok{  posts }\OtherTok{\textless{}{-}} \FunctionTok{head}\NormalTok{(}\FunctionTok{habr\_hub\_posts}\NormalTok{(hub, }\DecValTok{1}\NormalTok{), }\DecValTok{5}\NormalTok{)}
\end{Highlighting}
\end{Shaded}

Логика очень схожа с предыдущим ботом, но в данном случае Inline клавиатуру со списком статей мы генерируем динамически с помощью функции \texttt{lapply()}.

\begin{Shaded}
\begin{Highlighting}[]
  \CommentTok{\# формируем список кнопок}
\NormalTok{  keys }\OtherTok{\textless{}{-}} \FunctionTok{lapply}\NormalTok{(}\DecValTok{1}\SpecialCharTok{:}\DecValTok{5}\NormalTok{, }\ControlFlowTok{function}\NormalTok{(x) }\FunctionTok{list}\NormalTok{(}\FunctionTok{InlineKeyboardButton}\NormalTok{(posts}\SpecialCharTok{$}\NormalTok{title[x], }\AttributeTok{url =}\NormalTok{ posts}\SpecialCharTok{$}\NormalTok{link[x])))}

  \CommentTok{\# формируем клавиатуру}
\NormalTok{  IKM }\OtherTok{\textless{}{-}} \FunctionTok{InlineKeyboardMarkup}\NormalTok{(}
    \AttributeTok{inline\_keyboard =}\NormalTok{  keys}
\NormalTok{    )}
\end{Highlighting}
\end{Shaded}

В текст кнопки мы подставляем название статьи \texttt{posts\$title{[}x{]}}, а в аргумент \texttt{url} ссылку на статью: \texttt{url\ =\ posts\$link{[}x{]}}.

Далее, создаём фильтр, обработчики и запускаем нашего бота.

\hypertarget{ux437ux430ux43aux43bux44eux447ux435ux43dux438ux435-2}{%
\section{Заключение}\label{ux437ux430ux43aux43bux44eux447ux435ux43dux438ux435-2}}

Теперь написанные вами боты будут значительно удобней в работе, за счёт того, что управление ими будет осуществляться с клавиатуры, а не вводом команд. Как минимум при взаимодействии с ботом через смартфон клавиатура ощутимо упростит процесс его использования.

В следующей главе мы разберёмся как строить логический диалог с ботом, и работать с базами данных.

\hypertarget{ux442ux435ux441ux442ux44b-ux438-ux437ux430ux434ux430ux43dux438ux44f-2}{%
\section{Тесты и задания}\label{ux442ux435ux441ux442ux44b-ux438-ux437ux430ux434ux430ux43dux438ux44f-2}}

\hypertarget{ux442ux435ux441ux442ux44b-2}{%
\subsection{Тесты}\label{ux442ux435ux441ux442ux44b-2}}

Для закрепления материла рекомендую вам пройти тест доступный по \href{https://onlinetestpad.com/t/build-tg-bot-in-r-3}{ссылке}.

\hypertarget{ux437ux430ux434ux430ux43dux438ux44f-2}{%
\subsection{Задания}\label{ux437ux430ux434ux430ux43dux438ux44f-2}}

\begin{enumerate}
\def\labelenumi{\arabic{enumi}.}
\tightlist
\item
  Создайте бота, который будет поддерживать Reply клавиатуру. На Reply клавиатуре будет всего одна кнопка ``Время''. По нажатию на неё будет появляться Inline клавиатура с выбором из 6 часовых поясов.
\end{enumerate}

\begin{itemize}
\tightlist
\item
  Africa/Cairo
\item
  America/Chicago
\item
  Europe/Moscow
\item
  Asia/Bangkok
\item
  Europe/Kiev
\item
  Australia/Sydney
\end{itemize}

Кнопки Inline клавиатуры необходимо расположить по 2 в ряд, соответвенно в три ряда.

По нажатию на одну из кнопки Inline клавиатуры бот будет запрашивать информацию по текущему времени из API \href{http://worldtimeapi.org/}{worldtimeapi.org}.

Формат запроса к API: \texttt{http://worldtimeapi.org/api/timezone/\{area\}/:\{location\}}.

Где \texttt{\{area\}} это континент, например Europe, а \texttt{\{location\}} это город, например Kiev. Дату и время надо брать в ответе из компонента \texttt{datetime}.

Если вы всё сделали правильно результат будет такой:

\includegraphics{http://img.netpeak.ua/alsey/160410549780_kiss_146kb.png}

\hypertarget{ux43fux43eux441ux442ux440ux43eux435ux43dux438ux435-ux43fux43eux441ux43bux435ux434ux43eux432ux430ux442ux435ux43bux44cux43dux43eux433ux43e-ux43bux43eux433ux438ux447ux435ux441ux43aux43eux433ux43e-ux434ux438ux430ux43bux43eux433ux430-ux441-ux431ux43eux442ux43eux43c-4}{%
\chapter{Построение последовательного, логического диалога с ботом (4)}\label{ux43fux43eux441ux442ux440ux43eux435ux43dux438ux435-ux43fux43eux441ux43bux435ux434ux43eux432ux430ux442ux435ux43bux44cux43dux43eux433ux43e-ux43bux43eux433ux438ux447ux435ux441ux43aux43eux433ux43e-ux434ux438ux430ux43bux43eux433ux430-ux441-ux431ux43eux442ux43eux43c-4}}

В этой главе мы с вами научимся писать бота, который будет поддерживать последовательный диалог. Т.е. бот будет задавать вам вопросы, и ждать от вас ввода какой-либо информации. В зависимости от введённых вами данных бот будет выполнять некоторые действия.

Также в данной главе мы научимся использовать под капотом бота базы данных, в нашем примере это будет SQLite, но вы можете использовать любую другую СУБД. Более подробно о взаимодействии с базами данных на языке R я писал в \href{https://habr.com/ru/post/469215/}{статье на Хабре}.

\hypertarget{ux432ux432ux435ux434ux435ux43dux438ux435-1}{%
\section{Введение}\label{ux432ux432ux435ux434ux435ux43dux438ux435-1}}

Для того, что бы бот мог запрашивать от вас данные, и ждать ввод какой-либо информации вам потребуется фиксировать текущее состояние диалога. Лучший способ это делать, использовать какую нибудь встраиваемую базу данных, например SQLite.

Т.е. логика будет следующей. Мы вызываем метод бота, и бот последовательно запрашивает у нас какую-то информацию, при этом на каждом шаге он ждёт ввод этой информации, и может осуществлять её проверку.

Мы напишем максимально простого бота, сначала он будет спрашивать ваше имя, потом возраст, полученные данные будет сохранять в базу данных. При запросе возраста будет проверять, что бы введённые данные были числом, а не текстом.

Такой простой диалог будет иметь всего три состояния:
1. start - обычное состояние бота, в котором он не ждёт от вас никакой информации
2. wait\_name - состояние, при котором бот ожидает ввод имени
3. wait\_age - состояние, при котором бот ожидает ввод вашего возраста, количество полных лет.

\hypertarget{ux43fux440ux43eux446ux435ux441ux441-ux43fux43eux441ux442ux440ux43eux435ux43dux438ux44f-ux431ux43eux442ux430}{%
\section{Процесс построения бота}\label{ux43fux440ux43eux446ux435ux441ux441-ux43fux43eux441ux442ux440ux43eux435ux43dux438ux44f-ux431ux43eux442ux430}}

В ходе статьи мы с вами шаг за шагом построим бота, весь процесс схематически можно изобразить следующим образом:
\includegraphics{http://img.netpeak.ua/alsey/160071071193_kiss_48kb.png}

\begin{enumerate}
\def\labelenumi{\arabic{enumi}.}
\tightlist
\item
  Создаём конфиг бота, в котором будем хранить некоторые настройки. В нашем случае токен бота, и путь к файлу базы данных.
\item
  Создаём переменную среды, в которой будет хранится путь к проекту с ботом.
\item
  Создаём саму базу данных, и ряд функций для того, что бы бот мог взаимодействовать с ней.
\item
  Пишем методы бота, т.е. функции которые он будет выполнять.
\item
  Добавляем фильтры сообщений. С помощью которых бот будет обращаться к нужным методам, в зависимости от текущего состояния чата.
\item
  Добавляем обработчики, которые свяжут команды и сообщения с нужными методами бота.
\item
  Запускаем бота.
\end{enumerate}

\hypertarget{ux441ux442ux440ux443ux43aux442ux443ux440ux430-ux43fux440ux43eux435ux43aux442ux430-ux431ux43eux442ux430}{%
\section{Структура проекта бота}\label{ux441ux442ux440ux443ux43aux442ux443ux440ux430-ux43fux440ux43eux435ux43aux442ux430-ux431ux43eux442ux430}}

Для удобства мы разобъём код нашего бота, и прочие связанные с ним файлы на следующую структуру.

\begin{itemize}
\tightlist
\item
  \emph{bot.R} - основной код нашего бота
\item
  \emph{db\_bot\_function.R} - блок кода с функциями для работы с базой данных
\item
  \emph{bot\_methods.R} - код методов бота
\item
  \emph{message\_filters.R} - фильтры сообщений
\item
  \emph{handlers.R} - обработчики
\item
  \emph{config.cfg} - конфиг бота
\item
  \emph{create\_db\_data.sql} - SQL скрипт создания таблицы с данными чата в базе данных
\item
  \emph{create\_db\_state.sql} - SQL скрипт создания таблицы текущего состояния чата в базе данных
\item
  \emph{bot.db} - база данных бота
\end{itemize}

Весь проект бота можно посмотреть, или \href{https://github.com/selesnow/logical_tg_bot/archive/master.zip}{скачать} из моего \href{https://github.com/selesnow/logical_tg_bot}{репозитория на GitHub}.

\hypertarget{ux43aux43eux43dux444ux438ux433-ux431ux43eux442ux430}{%
\section{Конфиг бота}\label{ux43aux43eux43dux444ux438ux433-ux431ux43eux442ux430}}

В качестве конфига мы будем использовать обычный \href{https://ru.wikipedia.org/wiki/.ini}{ini файл}, следующего вида:

\begin{verbatim}
[bot_settings]
bot_token=ТОКЕН_ВАШЕГО_БОТА

[db_settings]
db_path=C:/ПУТЬ/К/ПАПКЕ/ПРОЕКТА/bot.db
\end{verbatim}

В конфиг мы записываем токен бота, и путь к базе данных, т.е. к файлу bot.db, сам файл мы будем создавать на следующем шаге.

Для более сложных ботов можно создавать и более сложные конфиги, к тому же необязательно писать именно ini конфиг, можете использовать любой другой формат включая JSON.

\hypertarget{ux441ux43eux437ux434ux430ux451ux43c-ux43fux435ux440ux435ux43cux435ux43dux43dux443ux44e-ux441ux440ux435ux434ux44b}{%
\section{Создаём переменную среды}\label{ux441ux43eux437ux434ux430ux451ux43c-ux43fux435ux440ux435ux43cux435ux43dux43dux443ux44e-ux441ux440ux435ux434ux44b}}

На каждом ПК папка с проектом бота может располагаться в разных директориях, и на разных дисках, поэтому в коде путь к папке проекта будет задан через переменную среды \texttt{TG\_BOT\_PATH}.

Создать переменную среды можно несколькими способами, наиболее простой - прописать её в файле \emph{.Renviron}.

Создать, или редактировать данный файл можно с помощью команды \texttt{file.edit(path.expand(file.path("\textasciitilde{}",\ ".Renviron")))}. Выполните её и добавьте в файл одну строку:

\begin{verbatim}
TG_BOT_PATH=C:/ПУТЬ/К/ВАШЕМУ/ПРОЕКТУ
\end{verbatim}

Далее сохраните файл \emph{.Renviron} и перезапустите RStudio.

\hypertarget{ux441ux43eux437ux434ux430ux451ux43c-ux431ux430ux437ux443-ux434ux430ux43dux43dux44bux445}{%
\section{Создаём базу данных}\label{ux441ux43eux437ux434ux430ux451ux43c-ux431ux430ux437ux443-ux434ux430ux43dux43dux44bux445}}

Следующий шаг - создание базы данных. Нам понадобится 2 таблицы:

\begin{itemize}
\tightlist
\item
  chat\_data - данные которые бот запросил у пользователя
\item
  chat\_state - текущее состояние всех чатов
\end{itemize}

Создать эти таблицы можно с помощью следующего SQL запроса:

\begin{Shaded}
\begin{Highlighting}[]
\KeywordTok{CREATE} \KeywordTok{TABLE}\NormalTok{ chat\_data (}
\NormalTok{    chat\_id BIGINT  }\KeywordTok{PRIMARY} \KeywordTok{KEY}
                    \KeywordTok{UNIQUE}\NormalTok{,}
\NormalTok{    name    TEXT,}
\NormalTok{    age     }\DataTypeTok{INTEGER}
\NormalTok{);}

\KeywordTok{CREATE} \KeywordTok{TABLE}\NormalTok{ chat\_state (}
\NormalTok{    chat\_id BIGINT }\KeywordTok{PRIMARY} \KeywordTok{KEY}
                   \KeywordTok{UNIQUE}\NormalTok{,}
\NormalTok{    state   TEXT}
\NormalTok{);}
\end{Highlighting}
\end{Shaded}

Если вы скачали проект бота с \href{https://github.com/selesnow/logical_tg_bot/archive/master.zip}{GitHub}, то для создания базы можете воспользоваться следующим кодом на языке R.

\begin{Shaded}
\begin{Highlighting}[]
\CommentTok{\# Скрипт создания базы данных}
\FunctionTok{library}\NormalTok{(DBI)     }\CommentTok{\# интерфейс для работы с СУБД}
\FunctionTok{library}\NormalTok{(configr) }\CommentTok{\# чтение конфига}
\FunctionTok{library}\NormalTok{(readr)   }\CommentTok{\# чтение текстовых SQL файлов}
\FunctionTok{library}\NormalTok{(RSQLite) }\CommentTok{\# драйвер для подключения к SQLite}

\CommentTok{\# директория проекта}
\FunctionTok{setwd}\NormalTok{(}\FunctionTok{Sys.getenv}\NormalTok{(}\StringTok{\textquotesingle{}TG\_BOT\_PATH\textquotesingle{}}\NormalTok{))}

\CommentTok{\# чтение конфига}
\NormalTok{cfg }\OtherTok{\textless{}{-}} \FunctionTok{read.config}\NormalTok{(}\StringTok{\textquotesingle{}config.cfg\textquotesingle{}}\NormalTok{)}

\CommentTok{\# подключение к SQLite}
\NormalTok{con }\OtherTok{\textless{}{-}} \FunctionTok{dbConnect}\NormalTok{(}\FunctionTok{SQLite}\NormalTok{(), cfg}\SpecialCharTok{$}\NormalTok{db\_settings}\SpecialCharTok{$}\NormalTok{db\_path)}

\CommentTok{\# Создание таблиц в базе}
\FunctionTok{dbExecute}\NormalTok{(con, }\AttributeTok{statement =} \FunctionTok{read\_file}\NormalTok{(}\StringTok{\textquotesingle{}create\_db\_data.sql\textquotesingle{}}\NormalTok{))}
\FunctionTok{dbExecute}\NormalTok{(con, }\AttributeTok{statement =} \FunctionTok{read\_file}\NormalTok{(}\StringTok{\textquotesingle{}create\_db\_state.sql\textquotesingle{}}\NormalTok{))}
\end{Highlighting}
\end{Shaded}

\hypertarget{ux43fux438ux448ux435ux43c-ux444ux443ux43dux43aux446ux438ux438-ux434ux43bux44f-ux440ux430ux431ux43eux442ux44b-ux441-ux431ux430ux437ux43eux439-ux434ux430ux43dux43dux44bux445}{%
\section{Пишем функции для работы с базой данных}\label{ux43fux438ux448ux435ux43c-ux444ux443ux43dux43aux446ux438ux438-ux434ux43bux44f-ux440ux430ux431ux43eux442ux44b-ux441-ux431ux430ux437ux43eux439-ux434ux430ux43dux43dux44bux445}}

У нас уже готов файл конфигурации и создана база данных. Теперь необходимо написать функции для чтения и записи данных в эту базу.

Если вы скачали проект из \href{https://github.com/selesnow/logical_tg_bot/archive/master.zip}{GitHub}, то функции вы можете найти в файле \emph{db\_bot\_function.R}.

\begin{Shaded}
\begin{Highlighting}[]
\CommentTok{\# \#\#\#\#\#\#\#\#\#\#\#\#\#\#\#\#\#\#\#\#\#\#\#\#\#\#\#\#\#\#\#\#\#\#\#\#\#\#\#\#\#\#\#\#\#\#\#\#\#\#\#\#\#\#\#\#\#\#\#}
\CommentTok{\# Function for work bot with database}

\CommentTok{\# получить текущее состояние чата}
\NormalTok{get\_state }\OtherTok{\textless{}{-}} \ControlFlowTok{function}\NormalTok{(chat\_id) \{}
  
\NormalTok{  con }\OtherTok{\textless{}{-}} \FunctionTok{dbConnect}\NormalTok{(}\FunctionTok{SQLite}\NormalTok{(), cfg}\SpecialCharTok{$}\NormalTok{db\_settings}\SpecialCharTok{$}\NormalTok{db\_path)}
  
\NormalTok{  chat\_state }\OtherTok{\textless{}{-}} \FunctionTok{dbGetQuery}\NormalTok{(con, }\FunctionTok{str\_interp}\NormalTok{(}\StringTok{"SELECT state FROM chat\_state WHERE chat\_id == $\{chat\_id\}"}\NormalTok{))}\SpecialCharTok{$}\NormalTok{state}
  
  \FunctionTok{return}\NormalTok{(}\FunctionTok{unlist}\NormalTok{(chat\_state))}
  
  \FunctionTok{dbDisconnect}\NormalTok{(con)}
\NormalTok{\}}

\CommentTok{\# установить текущее состояние чата}
\NormalTok{set\_state }\OtherTok{\textless{}{-}} \ControlFlowTok{function}\NormalTok{(chat\_id, state) \{}
  
\NormalTok{  con }\OtherTok{\textless{}{-}} \FunctionTok{dbConnect}\NormalTok{(}\FunctionTok{SQLite}\NormalTok{(), cfg}\SpecialCharTok{$}\NormalTok{db\_settings}\SpecialCharTok{$}\NormalTok{db\_path)}
  
  \CommentTok{\# upsert состояние чата}
  \FunctionTok{dbExecute}\NormalTok{(con, }
            \FunctionTok{str\_interp}\NormalTok{(}\StringTok{"}
\StringTok{            INSERT INTO chat\_state (chat\_id, state)}
\StringTok{                VALUES($\{chat\_id\}, \textquotesingle{}$\{state\}\textquotesingle{}) }
\StringTok{                ON CONFLICT(chat\_id) }
\StringTok{                DO UPDATE SET state=\textquotesingle{}$\{state\}\textquotesingle{};}
\StringTok{            "}\NormalTok{)}
\NormalTok{  )}
  
  \FunctionTok{dbDisconnect}\NormalTok{(con)}
  
\NormalTok{\}}

\CommentTok{\# запись полученных данных в базу}
\NormalTok{set\_chat\_data }\OtherTok{\textless{}{-}} \ControlFlowTok{function}\NormalTok{(chat\_id, field, value) \{}
  
  
\NormalTok{  con }\OtherTok{\textless{}{-}} \FunctionTok{dbConnect}\NormalTok{(}\FunctionTok{SQLite}\NormalTok{(), cfg}\SpecialCharTok{$}\NormalTok{db\_settings}\SpecialCharTok{$}\NormalTok{db\_path)}
  
  \CommentTok{\# upsert состояние чата}
  \FunctionTok{dbExecute}\NormalTok{(con, }
            \FunctionTok{str\_interp}\NormalTok{(}\StringTok{"}
\StringTok{            INSERT INTO chat\_data (chat\_id, $\{field\})}
\StringTok{                VALUES($\{chat\_id\}, \textquotesingle{}$\{value\}\textquotesingle{}) }
\StringTok{                ON CONFLICT(chat\_id) }
\StringTok{                DO UPDATE SET $\{field\}=\textquotesingle{}$\{value\}\textquotesingle{};}
\StringTok{            "}\NormalTok{)}
\NormalTok{  )}
  
  \FunctionTok{dbDisconnect}\NormalTok{(con)}
  
\NormalTok{\}}

\CommentTok{\# read chat data}
\NormalTok{get\_chat\_data }\OtherTok{\textless{}{-}} \ControlFlowTok{function}\NormalTok{(chat\_id, field) \{}
  
  
\NormalTok{  con }\OtherTok{\textless{}{-}} \FunctionTok{dbConnect}\NormalTok{(}\FunctionTok{SQLite}\NormalTok{(), cfg}\SpecialCharTok{$}\NormalTok{db\_settings}\SpecialCharTok{$}\NormalTok{db\_path)}
  
  \CommentTok{\# upsert состояние чата}
\NormalTok{  data }\OtherTok{\textless{}{-}} \FunctionTok{dbGetQuery}\NormalTok{(con, }
                     \FunctionTok{str\_interp}\NormalTok{(}\StringTok{"}
\StringTok{            SELECT $\{field\}}
\StringTok{            FROM chat\_data}
\StringTok{            WHERE chat\_id = $\{chat\_id\};}
\StringTok{            "}\NormalTok{)}
\NormalTok{  )}
  
  \FunctionTok{dbDisconnect}\NormalTok{(con)}
  
  \FunctionTok{return}\NormalTok{(data[[field]])}
  
\NormalTok{\}}
\end{Highlighting}
\end{Shaded}

Мы создали 4 простые функции:
* \texttt{get\_state()} - получить текущее состояние чата из БД
* \texttt{set\_state()} - записать текущее состояние чата в БД
* \texttt{get\_chat\_data()} - получить данные отправленные пользователем
* \texttt{set\_chat\_data()} - записать данные полученные от пользователя

Все функции достаточно простые, они либо читают данные из базы с помощью команды \texttt{dbGetQuery()}, либо совершают \texttt{UPSERT} операцию (изменение существующих данных или запись новых данных в БД), с помощью функции \texttt{dbExecute()}.

Синтаксис UPSERT операции выглядит следующим образом:

\begin{Shaded}
\begin{Highlighting}[]
\KeywordTok{INSERT} \KeywordTok{INTO}\NormalTok{ chat\_data (chat\_id, $\{field\})}
\KeywordTok{VALUES}\NormalTok{($\{chat\_id\}, }\StringTok{\textquotesingle{}$\{value\}\textquotesingle{}}\NormalTok{) }
\KeywordTok{ON}\NormalTok{ CONFLICT(chat\_id) }
\NormalTok{DO }\KeywordTok{UPDATE} \KeywordTok{SET}\NormalTok{ $\{field\}}\OperatorTok{=}\StringTok{\textquotesingle{}$\{value\}\textquotesingle{}}\NormalTok{;}
\end{Highlighting}
\end{Shaded}

Т.е. в наших таблицах поле \emph{chat\_id} имеет ограничение по уникальности и является первичным ключом таблиц. Изначально мы пробуем добавить информацию в таблицу, и получаем ошибку если данные по текущему чату уже присутствуют, в таком случае мы просто обновляем информацию по данному чату.

Далее эти функции мы будем использовать в методах и фильтрах бота.

\hypertarget{ux43cux435ux442ux43eux434ux44b-ux431ux43eux442ux430}{%
\section{Методы бота}\label{ux43cux435ux442ux43eux434ux44b-ux431ux43eux442ux430}}

Следующим шагом в построении нашего бота будет создание методов. Если вы скачали проект с \href{https://github.com/selesnow/logical_tg_bot/archive/master.zip}{GitHub}, то все методы находятся в файле \emph{bot\_methods.R}.

\begin{Shaded}
\begin{Highlighting}[]
\CommentTok{\# \#\#\#\#\#\#\#\#\#\#\#\#\#\#\#\#\#\#\#\#\#\#\#\#\#\#\#\#\#\#\#\#\#\#\#\#\#\#\#\#\#\#\#\#\#\#\#\#\#\#\#\#\#\#\#\#\#\#\#}
\CommentTok{\# bot methods}

\CommentTok{\# start dialog}
\NormalTok{start }\OtherTok{\textless{}{-}} \ControlFlowTok{function}\NormalTok{(bot, update) \{}
  
  \CommentTok{\# }
  
  \CommentTok{\# Send query}
\NormalTok{  bot}\SpecialCharTok{$}\FunctionTok{sendMessage}\NormalTok{(update}\SpecialCharTok{$}\NormalTok{message}\SpecialCharTok{$}\NormalTok{chat\_id, }
                  \AttributeTok{text =} \StringTok{"Введи своё имя"}\NormalTok{)}
  
  \CommentTok{\# переключаем состояние диалога в режим ожидания ввода имени}
  \FunctionTok{set\_state}\NormalTok{(}\AttributeTok{chat\_id =}\NormalTok{ update}\SpecialCharTok{$}\NormalTok{message}\SpecialCharTok{$}\NormalTok{chat\_id, }\AttributeTok{state =} \StringTok{\textquotesingle{}wait\_name\textquotesingle{}}\NormalTok{)}
  
\NormalTok{\}}

\CommentTok{\# get current chat state}
\NormalTok{state }\OtherTok{\textless{}{-}} \ControlFlowTok{function}\NormalTok{(bot, update) \{}
  
\NormalTok{  chat\_state }\OtherTok{\textless{}{-}} \FunctionTok{get\_state}\NormalTok{(update}\SpecialCharTok{$}\NormalTok{message}\SpecialCharTok{$}\NormalTok{chat\_id)}
  
  \CommentTok{\# Send state}
\NormalTok{  bot}\SpecialCharTok{$}\FunctionTok{sendMessage}\NormalTok{(update}\SpecialCharTok{$}\NormalTok{message}\SpecialCharTok{$}\NormalTok{chat\_id, }
                  \AttributeTok{text =} \FunctionTok{unlist}\NormalTok{(chat\_state))}
  
\NormalTok{\}}

\CommentTok{\# reset dialog state}
\NormalTok{reset }\OtherTok{\textless{}{-}} \ControlFlowTok{function}\NormalTok{(bot, update) \{}
  
  \FunctionTok{set\_state}\NormalTok{(}\AttributeTok{chat\_id =}\NormalTok{ update}\SpecialCharTok{$}\NormalTok{message}\SpecialCharTok{$}\NormalTok{chat\_id, }\AttributeTok{state =} \StringTok{\textquotesingle{}start\textquotesingle{}}\NormalTok{)}
  
\NormalTok{\}}

\CommentTok{\# enter username}
\NormalTok{enter\_name }\OtherTok{\textless{}{-}} \ControlFlowTok{function}\NormalTok{(bot, update) \{}
  
\NormalTok{  uname }\OtherTok{\textless{}{-}}\NormalTok{ update}\SpecialCharTok{$}\NormalTok{message}\SpecialCharTok{$}\NormalTok{text}
  
  \CommentTok{\# Send message with name}
\NormalTok{  bot}\SpecialCharTok{$}\FunctionTok{sendMessage}\NormalTok{(update}\SpecialCharTok{$}\NormalTok{message}\SpecialCharTok{$}\NormalTok{chat\_id, }
                  \AttributeTok{text =} \FunctionTok{paste0}\NormalTok{(uname, }\StringTok{", приятно познакомится, я бот!"}\NormalTok{))}
  
  \CommentTok{\# Записываем имя в глобальную переменную}
  \CommentTok{\#username \textless{}\textless{}{-} uname}
  \FunctionTok{set\_chat\_data}\NormalTok{(update}\SpecialCharTok{$}\NormalTok{message}\SpecialCharTok{$}\NormalTok{chat\_id, }\StringTok{\textquotesingle{}name\textquotesingle{}}\NormalTok{, uname) }
  
  \CommentTok{\# Справшиваем возраст}
\NormalTok{  bot}\SpecialCharTok{$}\FunctionTok{sendMessage}\NormalTok{(update}\SpecialCharTok{$}\NormalTok{message}\SpecialCharTok{$}\NormalTok{chat\_id, }
                  \AttributeTok{text =} \StringTok{"Сколько тебе лет?"}\NormalTok{)}
  
  \CommentTok{\# Меняем состояние на ожидание ввода имени}
  \FunctionTok{set\_state}\NormalTok{(}\AttributeTok{chat\_id =}\NormalTok{ update}\SpecialCharTok{$}\NormalTok{message}\SpecialCharTok{$}\NormalTok{chat\_id, }\AttributeTok{state =} \StringTok{\textquotesingle{}wait\_age\textquotesingle{}}\NormalTok{)}
  
\NormalTok{\}}

\CommentTok{\# enter user age}
\NormalTok{enter\_age }\OtherTok{\textless{}{-}} \ControlFlowTok{function}\NormalTok{(bot, update) \{}
  
\NormalTok{  uage }\OtherTok{\textless{}{-}} \FunctionTok{as.numeric}\NormalTok{(update}\SpecialCharTok{$}\NormalTok{message}\SpecialCharTok{$}\NormalTok{text)}
  
  \CommentTok{\# проверяем было введено число или нет}
  \ControlFlowTok{if}\NormalTok{ ( }\FunctionTok{is.na}\NormalTok{(uage) ) \{}
    
    \CommentTok{\# если введено не число то переспрашиваем возраст}
\NormalTok{    bot}\SpecialCharTok{$}\FunctionTok{sendMessage}\NormalTok{(update}\SpecialCharTok{$}\NormalTok{message}\SpecialCharTok{$}\NormalTok{chat\_id, }
                    \AttributeTok{text =} \StringTok{"Ты ввёл некорректные данные, введи число"}\NormalTok{)}
    
\NormalTok{  \} }\ControlFlowTok{else}\NormalTok{ \{}
    
    \CommentTok{\# если введено число сообщаем что возраст принят}
\NormalTok{    bot}\SpecialCharTok{$}\FunctionTok{sendMessage}\NormalTok{(update}\SpecialCharTok{$}\NormalTok{message}\SpecialCharTok{$}\NormalTok{chat\_id, }
                    \AttributeTok{text =} \StringTok{"ОК, возраст принят"}\NormalTok{)}
    
    \CommentTok{\# записываем глобальную переменную с возрастом}
    \CommentTok{\#userage \textless{}\textless{}{-} uage}
    \FunctionTok{set\_chat\_data}\NormalTok{(update}\SpecialCharTok{$}\NormalTok{message}\SpecialCharTok{$}\NormalTok{chat\_id, }\StringTok{\textquotesingle{}age\textquotesingle{}}\NormalTok{, uage) }
    
    \CommentTok{\# сообщаем какие данные были собраны}
\NormalTok{    username }\OtherTok{\textless{}{-}} \FunctionTok{get\_chat\_data}\NormalTok{(update}\SpecialCharTok{$}\NormalTok{message}\SpecialCharTok{$}\NormalTok{chat\_id, }\StringTok{\textquotesingle{}name\textquotesingle{}}\NormalTok{)}
\NormalTok{    userage  }\OtherTok{\textless{}{-}} \FunctionTok{get\_chat\_data}\NormalTok{(update}\SpecialCharTok{$}\NormalTok{message}\SpecialCharTok{$}\NormalTok{chat\_id, }\StringTok{\textquotesingle{}age\textquotesingle{}}\NormalTok{)}
    
\NormalTok{    bot}\SpecialCharTok{$}\FunctionTok{sendMessage}\NormalTok{(update}\SpecialCharTok{$}\NormalTok{message}\SpecialCharTok{$}\NormalTok{chat\_id, }
                    \AttributeTok{text =} \FunctionTok{paste0}\NormalTok{(}\StringTok{"Тебя зовут "}\NormalTok{, username, }\StringTok{" и тебе "}\NormalTok{, userage, }\StringTok{" лет. Будем знакомы"}\NormalTok{))}
    
    \CommentTok{\# возвращаем диалог в исходное состояние}
    \FunctionTok{set\_state}\NormalTok{(}\AttributeTok{chat\_id =}\NormalTok{ update}\SpecialCharTok{$}\NormalTok{message}\SpecialCharTok{$}\NormalTok{chat\_id, }\AttributeTok{state =} \StringTok{\textquotesingle{}start\textquotesingle{}}\NormalTok{)}
\NormalTok{  \}}
  
\NormalTok{\}}
\end{Highlighting}
\end{Shaded}

Мы создали 5 методов:

\begin{itemize}
\tightlist
\item
  start - Запуск диалога
\item
  state - Получить текущее состояние чата
\item
  reset - Сбросить текущее состояние чата
\item
  enter\_name - Бот запрашивает ваше имя
\item
  enter\_age - Бот запрашивает ваш возраст
\end{itemize}

Метод \texttt{start} запрашивает ваше имя, и переводит состояние чата в \emph{wait\_name}, т.е. в режим ожидания ввода вашего имени.

Далее, вы отправляете имя и оно обрабатывается методом \texttt{enter\_name}, бот с вами здоровается, записывает полученное имя в базу, и переводит чат в состояние \emph{wait\_age}.

На этом этапе бот ждёт от вас ввода вашего возраста. Вы отправляете ваш возраст, бот проверяет сообщение, если вы вместо числа отправили какой-то текст он скажет: \texttt{Ты\ ввёл\ некорректные\ данные,\ введи\ число}, и будет ждать от вас повторного ввода данных. В случае если вы отправили число, бот сообщит о том, что он принял ваш возраст, запишет полученные данные в базу, сообщит все полученные от вас данные и переведёт состояние чата в исходное положение, т.е. в \texttt{start}.

Вызвав метод \texttt{state} вы в любой момент можете запросить текущее состояние чата, а методом \texttt{reset} перевести чат в исходное состояние.

\hypertarget{ux444ux438ux43bux44cux442ux440ux44b-ux441ux43eux43eux431ux449ux435ux43dux438ux439}{%
\section{Фильтры сообщений}\label{ux444ux438ux43bux44cux442ux440ux44b-ux441ux43eux43eux431ux449ux435ux43dux438ux439}}

В нашем случае это одна из наиболее важных частей в построении бота. Именно с помощью фильтров сообщений бот будет понимать какую информацию он от вас ждёт, и как её надо обрабатывать.

В проекте на \href{https://github.com/selesnow/logical_tg_bot}{GitHub} фильтры прописаны в файле \emph{message\_filters.R}.

Код фильтров сообщений:

\begin{Shaded}
\begin{Highlighting}[]
\CommentTok{\# \#\#\#\#\#\#\#\#\#\#\#\#\#\#\#\#\#\#\#\#\#\#\#\#\#\#\#\#\#\#\#\#\#\#\#\#\#\#\#\#\#\#\#\#\#\#\#\#\#\#\#\#\#\#\#\#\#\#\#}
\CommentTok{\# message state filters}

\CommentTok{\# фильтр сообщений в состоянии ожидания имени}
\NormalTok{MessageFilters}\SpecialCharTok{$}\NormalTok{wait\_name }\OtherTok{\textless{}{-}} \FunctionTok{BaseFilter}\NormalTok{(}\ControlFlowTok{function}\NormalTok{(message) \{}
  \FunctionTok{get\_state}\NormalTok{( message}\SpecialCharTok{$}\NormalTok{chat\_id )  }\SpecialCharTok{==} \StringTok{"wait\_name"}
\NormalTok{\}}
\NormalTok{)}

\CommentTok{\# фильтр сообщений в состоянии ожидания возраста}
\NormalTok{MessageFilters}\SpecialCharTok{$}\NormalTok{wait\_age }\OtherTok{\textless{}{-}} \FunctionTok{BaseFilter}\NormalTok{(}\ControlFlowTok{function}\NormalTok{(message) \{}
  \FunctionTok{get\_state}\NormalTok{( message}\SpecialCharTok{$}\NormalTok{chat\_id )   }\SpecialCharTok{==} \StringTok{"wait\_age"}
\NormalTok{\}}
\NormalTok{)}
\end{Highlighting}
\end{Shaded}

В фильтрах мы используем написанную ранее функцию \texttt{get\_state()}, для того, что бы запрашивать текущее состояние чата. Данна функция требует всего 1 аргумент, id чата.

Далее фильтр \emph{wait\_name} обрабатывает сообщения когда чат находится в состоянии \texttt{wait\_name}, и соответственно фильтр \emph{wait\_age} обрабатывает сообщения когда чат находится в состоянии \texttt{wait\_age}.

\hypertarget{ux43eux431ux440ux430ux431ux43eux442ux447ux438ux43aux438}{%
\section{Обработчики}\label{ux43eux431ux440ux430ux431ux43eux442ux447ux438ux43aux438}}

Файл с обработчиками называется \emph{handlers.R}, и имеет следующий код:

\begin{Shaded}
\begin{Highlighting}[]
\CommentTok{\# \#\#\#\#\#\#\#\#\#\#\#\#\#\#\#\#\#\#\#\#\#\#\#\#\#\#\#\#\#\#\#\#\#\#\#\#\#\#\#\#\#\#\#\#\#\#\#\#\#\#\#\#\#\#\#\#\#\#\#}
\CommentTok{\# handlers}

\CommentTok{\# command handlers}
\NormalTok{start\_h }\OtherTok{\textless{}{-}} \FunctionTok{CommandHandler}\NormalTok{(}\StringTok{\textquotesingle{}start\textquotesingle{}}\NormalTok{, start)}
\NormalTok{state\_h }\OtherTok{\textless{}{-}} \FunctionTok{CommandHandler}\NormalTok{(}\StringTok{\textquotesingle{}state\textquotesingle{}}\NormalTok{, state)}
\NormalTok{reset\_h }\OtherTok{\textless{}{-}} \FunctionTok{CommandHandler}\NormalTok{(}\StringTok{\textquotesingle{}reset\textquotesingle{}}\NormalTok{, reset)}

\CommentTok{\# message handlers}
\DocumentationTok{\#\# !MessageFilters$command {-} означает что команды данные обработчики не обрабатывают, }
\DocumentationTok{\#\# только текстовые сообщения}
\NormalTok{wait\_age\_h  }\OtherTok{\textless{}{-}} \FunctionTok{MessageHandler}\NormalTok{(enter\_age,  MessageFilters}\SpecialCharTok{$}\NormalTok{wait\_age  }\SpecialCharTok{\&} \SpecialCharTok{!}\NormalTok{MessageFilters}\SpecialCharTok{$}\NormalTok{command)}
\NormalTok{wait\_name\_h }\OtherTok{\textless{}{-}} \FunctionTok{MessageHandler}\NormalTok{(enter\_name, MessageFilters}\SpecialCharTok{$}\NormalTok{wait\_name }\SpecialCharTok{\&} \SpecialCharTok{!}\NormalTok{MessageFilters}\SpecialCharTok{$}\NormalTok{command)}
\end{Highlighting}
\end{Shaded}

Сначала мы создаём обработчики команд, которые позволят вам запускать методы для начала диалога, его сброса, и запроса текущего состояния.

Далее мы создаём 2 обработчика сообщений с использованием созданных на прошлом шаге фильтров, и добавляем к ним фильтр \texttt{!MessageFilters\$command}, для того, что бы мы в любом состоянии чата могли использовать команды.

\hypertarget{ux43aux43eux434-ux437ux430ux43fux443ux441ux43aux430-ux431ux43eux442ux430}{%
\section{Код запуска бота}\label{ux43aux43eux434-ux437ux430ux43fux443ux441ux43aux430-ux431ux43eux442ux430}}

Теперь у нас всё готово к запуску, основной код запуска бота находится в файле \emph{bot.R}.

\begin{Shaded}
\begin{Highlighting}[]
\FunctionTok{library}\NormalTok{(telegram.bot)}
\FunctionTok{library}\NormalTok{(tidyverse)}
\FunctionTok{library}\NormalTok{(RSQLite)}
\FunctionTok{library}\NormalTok{(DBI)}
\FunctionTok{library}\NormalTok{(configr)}

\CommentTok{\# переходим в папку проекта}
\FunctionTok{setwd}\NormalTok{(}\FunctionTok{Sys.getenv}\NormalTok{(}\StringTok{\textquotesingle{}TG\_BOT\_PATH\textquotesingle{}}\NormalTok{))}

\CommentTok{\# читаем конфиг}
\NormalTok{cfg }\OtherTok{\textless{}{-}} \FunctionTok{read.config}\NormalTok{(}\StringTok{\textquotesingle{}config.cfg\textquotesingle{}}\NormalTok{)}

\CommentTok{\# создаём экземпляр бота}
\NormalTok{updater }\OtherTok{\textless{}{-}} \FunctionTok{Updater}\NormalTok{(cfg}\SpecialCharTok{$}\NormalTok{bot\_settings}\SpecialCharTok{$}\NormalTok{bot\_token)}

\CommentTok{\# Загрузка компонентов бота}
\FunctionTok{source}\NormalTok{(}\StringTok{\textquotesingle{}db\_bot\_function.R\textquotesingle{}}\NormalTok{) }\CommentTok{\# функции для работы с БД}
\FunctionTok{source}\NormalTok{(}\StringTok{\textquotesingle{}bot\_methods.R\textquotesingle{}}\NormalTok{)     }\CommentTok{\# методы бота}
\FunctionTok{source}\NormalTok{(}\StringTok{\textquotesingle{}message\_filters.R\textquotesingle{}}\NormalTok{) }\CommentTok{\# фильтры сообщений}
\FunctionTok{source}\NormalTok{(}\StringTok{\textquotesingle{}handlers.R\textquotesingle{}}\NormalTok{) }\CommentTok{\# обработчики сообщений}

\CommentTok{\# Добавляем обработчики в диспетчер}
\NormalTok{updater }\OtherTok{\textless{}{-}}\NormalTok{ updater }\SpecialCharTok{+}
\NormalTok{  start\_h }\SpecialCharTok{+}
\NormalTok{  wait\_age\_h }\SpecialCharTok{+}
\NormalTok{  wait\_name\_h }\SpecialCharTok{+}
\NormalTok{  state\_h }\SpecialCharTok{+}
\NormalTok{  reset\_h}

\CommentTok{\# Запускаем бота}
\NormalTok{updater}\SpecialCharTok{$}\FunctionTok{start\_polling}\NormalTok{()}
\end{Highlighting}
\end{Shaded}

В результате, у нас получился вот такой бот:
\includegraphics{https://img.netpeak.ua/alsey/160069507819_kiss_188kb.png}

В любой момент с помощью команды \texttt{/state} мы можем запрашивать текущее состояние чата, а с помощью команды \texttt{/reset} переводить чат в исходное состояние и начинать диалог заново.

\hypertarget{ux437ux430ux43aux43bux44eux447ux435ux43dux438ux435-3}{%
\section{Заключение}\label{ux437ux430ux43aux43bux44eux447ux435ux43dux438ux435-3}}

В этой главе мы разобрались как использовать внутри бота базы данных, и как строить последовательные логические диалоги за счёт фиксации состояния чата.

В данном случае мы рассмотрели самый примитивный пример, для того, что бы вам проще было понять идею построения таких ботов, на практике вы можете строить гораздо более сложные диалоги.

В следующей \href{https://habr.com/ru/post/520208/}{статье} из этой серии мы научимся ограничивать пользователям бота права на использования различных его методов.

\hypertarget{ux442ux435ux441ux442ux44b-ux438-ux437ux430ux434ux430ux43dux438ux44f-3}{%
\section{Тесты и задания}\label{ux442ux435ux441ux442ux44b-ux438-ux437ux430ux434ux430ux43dux438ux44f-3}}

\hypertarget{ux442ux435ux441ux442ux44b-3}{%
\subsection{Тесты}\label{ux442ux435ux441ux442ux44b-3}}

Для закрепления материла рекомендую вам пройти тест доступный по \href{https://onlinetestpad.com/t/build-tg-bot-in-r-4}{ссылке}.

\hypertarget{ux437ux430ux434ux430ux43dux438ux44f-3}{%
\subsection{Задания}\label{ux437ux430ux434ux430ux43dux438ux44f-3}}

\begin{enumerate}
\def\labelenumi{\arabic{enumi}.}
\tightlist
\item
  Постройте бота который будет поддерживать игру угадай число. Т.е. по команде \texttt{/start} бот будет загадывать число от 1 до 50. Далее у вас будет 5 попыток угадать это число.
\end{enumerate}

Вы по очереди в каждой из попыток вводите числа, если введённое число меньше чем то, которое загадал бот то бот пишет ``моё число больше'', иначе бот пишет ``моё число меньше''. Если вы ввели правильное число то бот пишет что вы выйграли, и переводит диалог в исходное состояние.

Если вы всё сделали правильно, бот будет выглядеть так:

\emph{Победа с 5 попытки:}

\includegraphics{http://img.netpeak.ua/alsey/160495888357_kiss_258kb.png}

\emph{Пройгрыш}
\includegraphics{http://img.netpeak.ua/alsey/160495881567_kiss_266kb.png}

\hypertarget{ux443ux43fux440ux430ux432ux43bux435ux43dux438ux435-ux43fux440ux430ux432ux430ux43cux438-ux43fux43eux43bux44cux437ux43eux432ux430ux442ux435ux43bux435ux439-ux431ux43eux442ux430-5}{%
\chapter{Управление правами пользователей бота (5)}\label{ux443ux43fux440ux430ux432ux43bux435ux43dux438ux435-ux43fux440ux430ux432ux430ux43cux438-ux43fux43eux43bux44cux437ux43eux432ux430ux442ux435ux43bux435ux439-ux431ux43eux442ux430-5}}

В этой главе мы разберёмся с тем, как управлять правами использования отдельных методов бота на различных уровнях.

\hypertarget{ux432ux432ux435ux434ux435ux43dux438ux435-2}{%
\section{Введение}\label{ux432ux432ux435ux434ux435ux43dux438ux435-2}}

Ваш бот может выполнять совершенно любые задачи, и автоматизировать как некоторые внутренние процессы, так и наладить коммуникации с клиентами.

Т.е. бот может использоваться в многопользовательском режиме. При этом, вам может понадобиться разграничить права на использование бота. Например, некоторые пользователи смогут использовать абсолютно все возможности бота, а некоторым вы предоставите ограниченные права.

Ограничить права можно не только на уровне пользователя, но и на уровне отдельных чатов.

Мы создадим простейшего бота, у которого в арсенале будет всего 2 метода:

\begin{itemize}
\tightlist
\item
  \texttt{say\_hello} - команда приветствия
\item
  \texttt{what\_time} - команда, по которой бот сообщает текущую дату и время
\end{itemize}

\begin{Shaded}
\begin{Highlighting}[]
\FunctionTok{library}\NormalTok{(telegram.bot)}

\CommentTok{\# создаём экземпляр класса Updater}
\NormalTok{updater }\OtherTok{\textless{}{-}} \FunctionTok{Updater}\NormalTok{(}\StringTok{\textquotesingle{}ТОКЕН ВАШЕГО БОТА\textquotesingle{}}\NormalTok{)}

\CommentTok{\# Пишем метод для приветсвия}
\DocumentationTok{\#\# команда приветствия}
\NormalTok{say\_hello }\OtherTok{\textless{}{-}} \ControlFlowTok{function}\NormalTok{(bot, update) \{}
  
  \CommentTok{\# Имя пользователя с которым надо поздароваться}
\NormalTok{  user\_name }\OtherTok{\textless{}{-}}\NormalTok{ update}\SpecialCharTok{$}\NormalTok{message}\SpecialCharTok{$}\NormalTok{from}\SpecialCharTok{$}\NormalTok{first\_name}
  
  \CommentTok{\# Отправка сообщения}
\NormalTok{  bot}\SpecialCharTok{$}\FunctionTok{sendMessage}\NormalTok{(update}\SpecialCharTok{$}\NormalTok{message}\SpecialCharTok{$}\NormalTok{chat\_id, }
                  \AttributeTok{text =} \FunctionTok{paste0}\NormalTok{(}\StringTok{"Моё почтение, "}\NormalTok{, user\_name, }\StringTok{"!"}\NormalTok{),}
                  \AttributeTok{parse\_mode =} \StringTok{"Markdown"}\NormalTok{,}
                  \AttributeTok{reply\_to\_message\_id =}\NormalTok{ update}\SpecialCharTok{$}\NormalTok{message}\SpecialCharTok{$}\NormalTok{message\_id)}

\NormalTok{\}}

\DocumentationTok{\#\# команда по которой бот возвращает системную дату и время}
\NormalTok{what\_time }\OtherTok{\textless{}{-}} \ControlFlowTok{function}\NormalTok{(bot, update) \{}
  
  \CommentTok{\# Запрашиваем текущее время}
\NormalTok{  cur\_time }\OtherTok{\textless{}{-}} \FunctionTok{as.character}\NormalTok{(}\FunctionTok{Sys.time}\NormalTok{())}
  
  \CommentTok{\# Отправка сообщения}
\NormalTok{  bot}\SpecialCharTok{$}\FunctionTok{sendMessage}\NormalTok{(update}\SpecialCharTok{$}\NormalTok{message}\SpecialCharTok{$}\NormalTok{chat\_id, }
                  \AttributeTok{text =} \FunctionTok{paste0}\NormalTok{(}\StringTok{"Текущее время, "}\NormalTok{, cur\_time),}
                                \AttributeTok{parse\_mode =} \StringTok{"Markdown"}\NormalTok{,}
                                \AttributeTok{reply\_to\_message\_id =}\NormalTok{ update}\SpecialCharTok{$}\NormalTok{message}\SpecialCharTok{$}\NormalTok{message\_id)}
                  
  
\NormalTok{\}}

\CommentTok{\# обработчики}
\NormalTok{h\_hello }\OtherTok{\textless{}{-}} \FunctionTok{CommandHandler}\NormalTok{(}\StringTok{\textquotesingle{}say\_hello\textquotesingle{}}\NormalTok{, say\_hello)}
\NormalTok{h\_time  }\OtherTok{\textless{}{-}} \FunctionTok{CommandHandler}\NormalTok{(}\StringTok{\textquotesingle{}what\_time\textquotesingle{}}\NormalTok{, what\_time)}

\CommentTok{\# добавляем обработчики в диспетчер}
\NormalTok{updater }\OtherTok{\textless{}{-}}\NormalTok{ updater }\SpecialCharTok{+}\NormalTok{ h\_hello }\SpecialCharTok{+}\NormalTok{ h\_time}

\CommentTok{\# запускаем бота }
\NormalTok{updater}\SpecialCharTok{$}\FunctionTok{start\_polling}\NormalTok{()}
\end{Highlighting}
\end{Shaded}

\begin{quote}
Запустите приведённый выше пример кода, предварительно заменив `ТОКЕН ВАШЕГО БОТА' на реальный токен, который вы получили при создании бота через \emph{BotFather} (о создании бота я рассказывал в \href{https://habr.com/ru/post/511222/\#sozdanie-telegram-bota}{первой статье}).
\end{quote}

В данной главе мы разберёмся с тем, как разными способами, и на разных уровнях ограничить использование методов этого бота.

\hypertarget{ux43eux433ux440ux430ux43dux438ux447ux438ux432ux430ux435ux43c-ux43fux440ux430ux432ux430-ux43fux43eux43bux44cux437ux43eux432ux430ux442ux435ux43bux44f-ux441-ux43fux43eux43cux43eux449ux44cux44e-ux444ux438ux43bux44cux442ux440ux43eux432-ux441ux43eux43eux431ux449ux435ux43dux438ux439}{%
\section{Ограничиваем права пользователя с помощью фильтров сообщений}\label{ux43eux433ux440ux430ux43dux438ux447ux438ux432ux430ux435ux43c-ux43fux440ux430ux432ux430-ux43fux43eux43bux44cux437ux43eux432ux430ux442ux435ux43bux44f-ux441-ux43fux43eux43cux43eux449ux44cux44e-ux444ux438ux43bux44cux442ux440ux43eux432-ux441ux43eux43eux431ux449ux435ux43dux438ux439}}

Из предыдущих публикаций мы уже разобрались с тем, что такое фильтры сообщений. Но ранее мы использовали их в основном для, того, что бы вызывать какие-то методы бота через обычное сообщение, а не команду и для прослушиваний сообщений в определённом состоянии чата.

В этот раз мы научимся с помощью фильтров ограничивать возможности по использованию методов бота, на разных уровнях.

\hypertarget{ux43eux433ux440ux430ux43dux438ux447ux438ux432ux430ux435ux43c-ux43fux440ux430ux432ux430-ux43dux430-ux443ux440ux43eux432ux43dux435-ux438ux43cux435ux43dux438-ux43fux43eux43bux44cux437ux43eux432ux430ux442ux435ux43bux44f}{%
\subsection{Ограничиваем права на уровне имени пользователя}\label{ux43eux433ux440ux430ux43dux438ux447ux438ux432ux430ux435ux43c-ux43fux440ux430ux432ux430-ux43dux430-ux443ux440ux43eux432ux43dux435-ux438ux43cux435ux43dux438-ux43fux43eux43bux44cux437ux43eux432ux430ux442ux435ux43bux44f}}

Для создания собственных фильтров вам необходимо с помощью функции \texttt{BaseFilter()} добавить новый элемент в объект \texttt{MessageFilters}. Более подробно об этом я рассказывал во \href{https://habr.com/ru/post/515148/}{второй статье} из данной серии.

В анонимную функцию, которую вы прописываете внутри \texttt{BaseFilter()} передаётся всего один аргумент - \texttt{message}. Это сообщение которое вы, или другой пользователей отправляет боту, со всеми его метаданными. Данный объект имеет следующую структуру:

\begin{verbatim}
$message_id
[1] 1174

$from
$from$id
[1] 194336771

$from$is_bot
[1] FALSE

$from$first_name
[1] "Alexey"

$from$last_name
[1] "Seleznev"

$from$username
[1] "AlexeySeleznev"

$from$language_code
[1] "ru"


$chat
$chat$id
[1] 194336771

$chat$first_name
[1] "Alexey"

$chat$last_name
[1] "Seleznev"

$chat$username
[1] "AlexeySeleznev"

$chat$type
[1] "private"


$date
[1] 1601295189

$text
[1] "отправленный пользователем текст"

$chat_id
[1] 194336771

$from_user
[1] 194336771
\end{verbatim}

Более подробно описание ответа и всех его компонентов можно узнать из официальной документации:

\begin{itemize}
\tightlist
\item
  \href{https://tlgrm.ru/docs/bots/api\#user}{User} - Этот объект представляет бота или пользователя Telegram.
\item
  \href{https://tlgrm.ru/docs/bots/api\#chat}{Chat} - Этот объект представляет собой чат.
\item
  \href{https://tlgrm.ru/docs/bots/api\#message}{Message} - Этот объект представляет собой сообщение.
\end{itemize}

Соответственно вы можете разделять права пользователей бота, и его методов используя любую, отправляемую с сообщением информацию, т.е. любые элементы объекта \texttt{message}. Для того, что бы ограничить круг пользователей которые могут использовать методы вашего бота достаточно создать фильтр:

\begin{Shaded}
\begin{Highlighting}[]
\DocumentationTok{\#\# список пользователей, с полными правами}
\NormalTok{MessageFilters}\SpecialCharTok{$}\NormalTok{admins }\OtherTok{\textless{}{-}} \FunctionTok{BaseFilter}\NormalTok{(}
  \ControlFlowTok{function}\NormalTok{(message) \{}
    
    \CommentTok{\# проверяем от кого отправлено сообщение}
\NormalTok{    message}\SpecialCharTok{$}\NormalTok{from}\SpecialCharTok{$}\NormalTok{username }\SpecialCharTok{\%in\%} \FunctionTok{c}\NormalTok{(}\StringTok{\textquotesingle{}AlexeySeleznev\textquotesingle{}}\NormalTok{, }\StringTok{\textquotesingle{}user1\textquotesingle{}}\NormalTok{, }\StringTok{\textquotesingle{}user2\textquotesingle{}}\NormalTok{)}
    
\NormalTok{ \}}
\NormalTok{)}
\end{Highlighting}
\end{Shaded}

Где \texttt{c(\textquotesingle{}AlexeySeleznev\textquotesingle{},\ \textquotesingle{}user1\textquotesingle{},\ \textquotesingle{}user2\textquotesingle{})} - вектор, с именами пользователей, которые могут использовать все функции бота. Далее этот фильтр мы используем при создании обработчиков.

\begin{Shaded}
\begin{Highlighting}[]
\DocumentationTok{\#\# фильтр для вызова команды say\_hello}
\NormalTok{MessageFilters}\SpecialCharTok{$}\NormalTok{say\_hello }\OtherTok{\textless{}{-}} \FunctionTok{BaseFilter}\NormalTok{(}
  \ControlFlowTok{function}\NormalTok{(message) \{}
    
    \CommentTok{\# проверяем от кого отправлено сообщение}
\NormalTok{    message}\SpecialCharTok{$}\NormalTok{text }\SpecialCharTok{==} \StringTok{\textquotesingle{}/say\_hallo\textquotesingle{}}
    
\NormalTok{  \}}
\NormalTok{)}

\DocumentationTok{\#\# фильтр для вызова команды what\_time}
\NormalTok{MessageFilters}\SpecialCharTok{$}\NormalTok{what\_time }\OtherTok{\textless{}{-}} \FunctionTok{BaseFilter}\NormalTok{(}
  \ControlFlowTok{function}\NormalTok{(message) \{}
    
    \CommentTok{\# проверяем от кого отправлено сообщение}
\NormalTok{    message}\SpecialCharTok{$}\NormalTok{text }\SpecialCharTok{==} \StringTok{\textquotesingle{}/what\_time\textquotesingle{}}
    
\NormalTok{  \}}
\NormalTok{)}


\CommentTok{\# обработчики}
\NormalTok{h\_hello }\OtherTok{\textless{}{-}} \FunctionTok{MessageHandler}\NormalTok{(say\_hello, MessageFilters}\SpecialCharTok{$}\NormalTok{admins }\SpecialCharTok{\&}\NormalTok{ MessageFilters}\SpecialCharTok{$}\NormalTok{say\_hello)}
\NormalTok{h\_time  }\OtherTok{\textless{}{-}} \FunctionTok{MessageHandler}\NormalTok{(what\_time, MessageFilters}\SpecialCharTok{$}\NormalTok{admins }\SpecialCharTok{\&}\NormalTok{ MessageFilters}\SpecialCharTok{$}\NormalTok{what\_time)}
\end{Highlighting}
\end{Shaded}

Теперь нашего бота могут использовать пользователи с логинами AlexeySeleznev, user1, user2. На сообщения отправленные боту другими пользователями он никак не будет реагировать.

Изменённый код нашего бота на данный момент выглядит вот так:

\begin{Shaded}
\begin{Highlighting}[]
\FunctionTok{library}\NormalTok{(telegram.bot)}


\CommentTok{\# создаём экземпляр класса Updater}
\NormalTok{updater }\OtherTok{\textless{}{-}} \FunctionTok{Updater}\NormalTok{(}\StringTok{\textquotesingle{}ТОКЕН ВАШЕГО БОТА\textquotesingle{}}\NormalTok{)}

\CommentTok{\# Пишем метод для приветсвия}
\DocumentationTok{\#\# команда приветствия}
\NormalTok{say\_hello }\OtherTok{\textless{}{-}} \ControlFlowTok{function}\NormalTok{(bot, update) \{}
  
  \CommentTok{\# Имя пользователя с которым надо поздароваться}
\NormalTok{  user\_name }\OtherTok{\textless{}{-}}\NormalTok{ update}\SpecialCharTok{$}\NormalTok{message}\SpecialCharTok{$}\NormalTok{from}\SpecialCharTok{$}\NormalTok{first\_name}
  
  \CommentTok{\# Отправка сообщения}
\NormalTok{  bot}\SpecialCharTok{$}\FunctionTok{sendMessage}\NormalTok{(update}\SpecialCharTok{$}\NormalTok{message}\SpecialCharTok{$}\NormalTok{chat\_id, }
                  \AttributeTok{text =} \FunctionTok{paste0}\NormalTok{(}\StringTok{"Моё почтение, "}\NormalTok{, user\_name, }\StringTok{"!"}\NormalTok{),}
                  \AttributeTok{parse\_mode =} \StringTok{"Markdown"}\NormalTok{,}
                  \AttributeTok{reply\_to\_message\_id =}\NormalTok{ update}\SpecialCharTok{$}\NormalTok{message}\SpecialCharTok{$}\NormalTok{message\_id)}

\NormalTok{\}}

\DocumentationTok{\#\# команда по которой бот возвращает системную дату и время}
\NormalTok{what\_time }\OtherTok{\textless{}{-}} \ControlFlowTok{function}\NormalTok{(bot, update) \{}
  
  \CommentTok{\# Запрашиваем текущее время}
\NormalTok{  cur\_time }\OtherTok{\textless{}{-}} \FunctionTok{as.character}\NormalTok{(}\FunctionTok{Sys.time}\NormalTok{())}
  
  \CommentTok{\# Отправка сообщения}
\NormalTok{  bot}\SpecialCharTok{$}\FunctionTok{sendMessage}\NormalTok{(update}\SpecialCharTok{$}\NormalTok{message}\SpecialCharTok{$}\NormalTok{chat\_id, }
                  \AttributeTok{text =} \FunctionTok{paste0}\NormalTok{(}\StringTok{"Текущее время, "}\NormalTok{, cur\_time),}
                                \AttributeTok{parse\_mode =} \StringTok{"Markdown"}\NormalTok{,}
                                \AttributeTok{reply\_to\_message\_id =}\NormalTok{ update}\SpecialCharTok{$}\NormalTok{message}\SpecialCharTok{$}\NormalTok{message\_id)}
                  
  
\NormalTok{\}}

\CommentTok{\# фильтры}
\DocumentationTok{\#\# список пользователей, с полными правами}
\NormalTok{MessageFilters}\SpecialCharTok{$}\NormalTok{admins }\OtherTok{\textless{}{-}} \FunctionTok{BaseFilter}\NormalTok{(}
  \ControlFlowTok{function}\NormalTok{(message) \{}
    
    \CommentTok{\# проверяем от кого отправлено сообщение}
\NormalTok{    message}\SpecialCharTok{$}\NormalTok{from}\SpecialCharTok{$}\NormalTok{username }\SpecialCharTok{\%in\%} \FunctionTok{c}\NormalTok{(}\StringTok{\textquotesingle{}AlexeySeleznev\textquotesingle{}}\NormalTok{, }\StringTok{\textquotesingle{}user1\textquotesingle{}}\NormalTok{, }\StringTok{\textquotesingle{}user2\textquotesingle{}}\NormalTok{)}
    
\NormalTok{ \}}
\NormalTok{)}

\DocumentationTok{\#\# фильтр для вызова команды say\_hello}
\NormalTok{MessageFilters}\SpecialCharTok{$}\NormalTok{say\_hello }\OtherTok{\textless{}{-}} \FunctionTok{BaseFilter}\NormalTok{(}
  \ControlFlowTok{function}\NormalTok{(message) \{}
    
    \CommentTok{\# проверяем от кого отправлено сообщение}
\NormalTok{    message}\SpecialCharTok{$}\NormalTok{text }\SpecialCharTok{==} \StringTok{\textquotesingle{}/say\_hallo\textquotesingle{}}
    
\NormalTok{  \}}
\NormalTok{)}

\DocumentationTok{\#\# фильтр для вызова команды what\_time}
\NormalTok{MessageFilters}\SpecialCharTok{$}\NormalTok{what\_time }\OtherTok{\textless{}{-}} \FunctionTok{BaseFilter}\NormalTok{(}
  \ControlFlowTok{function}\NormalTok{(message) \{}
    
    \CommentTok{\# проверяем от кого отправлено сообщение}
\NormalTok{    message}\SpecialCharTok{$}\NormalTok{text }\SpecialCharTok{==} \StringTok{\textquotesingle{}/what\_time\textquotesingle{}}
    
\NormalTok{  \}}
\NormalTok{)}


\CommentTok{\# обработчики}
\NormalTok{h\_hello }\OtherTok{\textless{}{-}} \FunctionTok{MessageHandler}\NormalTok{(say\_hello, MessageFilters}\SpecialCharTok{$}\NormalTok{admins }\SpecialCharTok{\&}\NormalTok{ MessageFilters}\SpecialCharTok{$}\NormalTok{say\_hello)}
\NormalTok{h\_time  }\OtherTok{\textless{}{-}} \FunctionTok{MessageHandler}\NormalTok{(what\_time, MessageFilters}\SpecialCharTok{$}\NormalTok{admins }\SpecialCharTok{\&}\NormalTok{ MessageFilters}\SpecialCharTok{$}\NormalTok{what\_time)}

\CommentTok{\# добавляем обработчики в диспетчер}
\NormalTok{updater }\OtherTok{\textless{}{-}}\NormalTok{ updater }\SpecialCharTok{+}\NormalTok{ h\_hello }\SpecialCharTok{+}\NormalTok{ h\_time}

\CommentTok{\# запускаем бота }
\NormalTok{updater}\SpecialCharTok{$}\FunctionTok{start\_polling}\NormalTok{()}
\end{Highlighting}
\end{Shaded}

\hypertarget{ux43eux433ux440ux430ux43dux438ux447ux438ux432ux430ux435ux43c-ux43fux440ux430ux432ux430-ux43dux430-ux443ux440ux43eux432ux43dux435-ux447ux430ux442ux430}{%
\subsection{Ограничиваем права на уровне чата}\label{ux43eux433ux440ux430ux43dux438ux447ux438ux432ux430ux435ux43c-ux43fux440ux430ux432ux430-ux43dux430-ux443ux440ux43eux432ux43dux435-ux447ux430ux442ux430}}

Тем же способом мы можем создать фильтр не только по списку пользователей, но и по конкретному чату. Для этого достаточно создать ещё один фильтр:

\begin{Shaded}
\begin{Highlighting}[]
\DocumentationTok{\#\# список чатов в которых разрешено использовать бота}
\NormalTok{MessageFilters}\SpecialCharTok{$}\NormalTok{chats }\OtherTok{\textless{}{-}} \FunctionTok{BaseFilter}\NormalTok{(}
  \ControlFlowTok{function}\NormalTok{(message) \{}
    
    \CommentTok{\# проверяем от кого отправлено сообщение}
\NormalTok{    message}\SpecialCharTok{$}\NormalTok{chat\_id }\SpecialCharTok{\%in\%} \FunctionTok{c}\NormalTok{(}\DecValTok{194336771}\NormalTok{, }\DecValTok{0}\NormalTok{, }\DecValTok{1}\NormalTok{)}
    
\NormalTok{  \}}
\NormalTok{)}


\DocumentationTok{\#\# фильтр для вызова команды say\_hello}
\NormalTok{MessageFilters}\SpecialCharTok{$}\NormalTok{say\_hello }\OtherTok{\textless{}{-}} \FunctionTok{BaseFilter}\NormalTok{(}
  \ControlFlowTok{function}\NormalTok{(message) \{}
    
    \CommentTok{\# проверяем от кого отправлено сообщение}
\NormalTok{    message}\SpecialCharTok{$}\NormalTok{text }\SpecialCharTok{==} \StringTok{\textquotesingle{}/say\_hallo\textquotesingle{}}
    
\NormalTok{  \}}
\NormalTok{)}

\DocumentationTok{\#\# фильтр для вызова команды what\_time}
\NormalTok{MessageFilters}\SpecialCharTok{$}\NormalTok{what\_time }\OtherTok{\textless{}{-}} \FunctionTok{BaseFilter}\NormalTok{(}
  \ControlFlowTok{function}\NormalTok{(message) \{}
    
    \CommentTok{\# проверяем от кого отправлено сообщение}
\NormalTok{    message}\SpecialCharTok{$}\NormalTok{text }\SpecialCharTok{==} \StringTok{\textquotesingle{}/what\_time\textquotesingle{}}
    
\NormalTok{  \}}
\NormalTok{)}


\CommentTok{\# обработчики}
\NormalTok{h\_hello }\OtherTok{\textless{}{-}} \FunctionTok{MessageHandler}\NormalTok{(say\_hello, MessageFilters}\SpecialCharTok{$}\NormalTok{admins }\SpecialCharTok{\&}\NormalTok{ MessageFilters}\SpecialCharTok{$}\NormalTok{chats }\SpecialCharTok{\&}\NormalTok{ MessageFilters}\SpecialCharTok{$}\NormalTok{say\_hello)}
\NormalTok{h\_time  }\OtherTok{\textless{}{-}} \FunctionTok{MessageHandler}\NormalTok{(what\_time, MessageFilters}\SpecialCharTok{$}\NormalTok{admins }\SpecialCharTok{\&}\NormalTok{ MessageFilters}\SpecialCharTok{$}\NormalTok{chats }\SpecialCharTok{\&}\NormalTok{ MessageFilters}\SpecialCharTok{$}\NormalTok{what\_time)}
\end{Highlighting}
\end{Shaded}

\hypertarget{ux43eux433ux440ux430ux43dux438ux447ux438ux432ux430ux435ux43c-ux43fux440ux430ux432ux430-ux43fux43eux43bux44cux437ux43eux432ux430ux442ux435ux43bux44f-ux432ux43dux443ux442ux440ux438-ux43aux43eux434ux430-ux43cux435ux442ux43eux434ux43eux432}{%
\section{Ограничиваем права пользователя внутри кода методов}\label{ux43eux433ux440ux430ux43dux438ux447ux438ux432ux430ux435ux43c-ux43fux440ux430ux432ux430-ux43fux43eux43bux44cux437ux43eux432ux430ux442ux435ux43bux44f-ux432ux43dux443ux442ux440ux438-ux43aux43eux434ux430-ux43cux435ux442ux43eux434ux43eux432}}

Так же вы можете ограничить использование методов не прибегая к созданию дополнительных фильтров, а прописав все условия внутри каждого метода.

\hypertarget{ux43eux433ux440ux430ux43dux438ux447ux438ux432ux430ux435ux43c-ux43fux440ux430ux432ux430-ux43dux430-ux443ux440ux43eux432ux43dux435-ux438ux43cux435ux43dux438-ux43fux43eux43bux44cux437ux43eux432ux430ux442ux435ux43bux44f-1}{%
\subsection{Ограничиваем права на уровне имени пользователя}\label{ux43eux433ux440ux430ux43dux438ux447ux438ux432ux430ux435ux43c-ux43fux440ux430ux432ux430-ux43dux430-ux443ux440ux43eux432ux43dux435-ux438ux43cux435ux43dux438-ux43fux43eux43bux44cux437ux43eux432ux430ux442ux435ux43bux44f-1}}

Давайте создадим функцию, которая будет принимать данные для проверки имени пользователя.

\begin{Shaded}
\begin{Highlighting}[]
\CommentTok{\# функция для проверки прав пользователя}
\NormalTok{bot\_check\_usernames }\OtherTok{\textless{}{-}} 
  \ControlFlowTok{function}\NormalTok{(admins, username) \{}
  
\NormalTok{   username }\SpecialCharTok{\%in\%}\NormalTok{ admins }
  
\NormalTok{\}}
\end{Highlighting}
\end{Shaded}

В аргумент \texttt{admins} далее нам надо передавать список пользователей, которым разрешено использовать данный метод, а в аргумент \texttt{username} имя пользователя, которого надо проверить в списке.

Далее в каждый метод, который мы хотим защитить с помощью конструкции \texttt{IF} добавляем проверку, разрешено пользователю использовать данный метод или нет. В случае если у пользователя нет таких прав бот будет сообщать об этом в чате.

Давайте, для примера, я исключу себя из списка пользователей, которым разрешено использовать метод \texttt{what\_time}.

\begin{Shaded}
\begin{Highlighting}[]
\FunctionTok{library}\NormalTok{(telegram.bot)}

\CommentTok{\# создаём экземпляр класса Updater}
\NormalTok{updater }\OtherTok{\textless{}{-}} \FunctionTok{Updater}\NormalTok{(}\StringTok{\textquotesingle{}ТОКЕН ВАШЕГО БОТА\textquotesingle{}}\NormalTok{)}

\CommentTok{\# Пишем метод для приветсвия}
\DocumentationTok{\#\# команда приветствия}
\NormalTok{say\_hello }\OtherTok{\textless{}{-}} \ControlFlowTok{function}\NormalTok{(bot, update) \{}
  
  \CommentTok{\# Имя пользователя с которым надо поздароваться}
\NormalTok{  user\_name }\OtherTok{\textless{}{-}}\NormalTok{ update}\SpecialCharTok{$}\NormalTok{message}\SpecialCharTok{$}\NormalTok{from}\SpecialCharTok{$}\NormalTok{username}
  
  \CommentTok{\# проверяем разрешено ли использовать данному пользователю этот метод}
  \ControlFlowTok{if}\NormalTok{ ( }\FunctionTok{bot\_check\_usernames}\NormalTok{(}\FunctionTok{c}\NormalTok{(}\StringTok{\textquotesingle{}AlexeySeleznev\textquotesingle{}}\NormalTok{, }\StringTok{\textquotesingle{}user1\textquotesingle{}}\NormalTok{, }\StringTok{\textquotesingle{}user2\textquotesingle{}}\NormalTok{), user\_name) ) \{}
    
    \CommentTok{\# Отправка сообщения}
\NormalTok{    bot}\SpecialCharTok{$}\FunctionTok{sendMessage}\NormalTok{(update}\SpecialCharTok{$}\NormalTok{message}\SpecialCharTok{$}\NormalTok{chat\_id, }
                    \AttributeTok{text =} \FunctionTok{paste0}\NormalTok{(}\StringTok{"Моё почтение, "}\NormalTok{, user\_name, }\StringTok{"!"}\NormalTok{),}
                    \AttributeTok{parse\_mode =} \StringTok{"Markdown"}\NormalTok{,}
                    \AttributeTok{reply\_to\_message\_id =}\NormalTok{ update}\SpecialCharTok{$}\NormalTok{message}\SpecialCharTok{$}\NormalTok{message\_id)}
      
\NormalTok{  \} }\ControlFlowTok{else}\NormalTok{ \{}
    
    \CommentTok{\# Отправка сообщения}
\NormalTok{    bot}\SpecialCharTok{$}\FunctionTok{sendMessage}\NormalTok{(update}\SpecialCharTok{$}\NormalTok{message}\SpecialCharTok{$}\NormalTok{chat\_id, }
                    \AttributeTok{text =} \FunctionTok{paste0}\NormalTok{(}\StringTok{"У вас нет прав для использования этого метода!"}\NormalTok{),}
                    \AttributeTok{parse\_mode =} \StringTok{"Markdown"}\NormalTok{,}
                    \AttributeTok{reply\_to\_message\_id =}\NormalTok{ update}\SpecialCharTok{$}\NormalTok{message}\SpecialCharTok{$}\NormalTok{message\_id)}
    
\NormalTok{  \}}

\NormalTok{\}}

\DocumentationTok{\#\# команда по которой бот возвращает системную дату и время}
\NormalTok{what\_time }\OtherTok{\textless{}{-}} \ControlFlowTok{function}\NormalTok{(bot, update) \{}
  
  \CommentTok{\# проверяем разрешено ли использовать данному пользователю этот метод}
  \ControlFlowTok{if}\NormalTok{ ( }\FunctionTok{bot\_check\_usernames}\NormalTok{(}\FunctionTok{c}\NormalTok{(}\StringTok{\textquotesingle{}user1\textquotesingle{}}\NormalTok{, }\StringTok{\textquotesingle{}user2\textquotesingle{}}\NormalTok{), update}\SpecialCharTok{$}\NormalTok{message}\SpecialCharTok{$}\NormalTok{from}\SpecialCharTok{$}\NormalTok{username) ) \{}
      
    \CommentTok{\# Запрашиваем текущее время}
\NormalTok{    cur\_time }\OtherTok{\textless{}{-}} \FunctionTok{as.character}\NormalTok{(}\FunctionTok{Sys.time}\NormalTok{())}
    
    \CommentTok{\# Отправка сообщения о том что у пользователя не достаточно прав}
\NormalTok{    bot}\SpecialCharTok{$}\FunctionTok{sendMessage}\NormalTok{(update}\SpecialCharTok{$}\NormalTok{message}\SpecialCharTok{$}\NormalTok{chat\_id, }
                    \AttributeTok{text =} \FunctionTok{paste0}\NormalTok{(}\StringTok{"Текущее время, "}\NormalTok{, cur\_time),}
                                  \AttributeTok{parse\_mode =} \StringTok{"Markdown"}\NormalTok{,}
                                  \AttributeTok{reply\_to\_message\_id =}\NormalTok{ update}\SpecialCharTok{$}\NormalTok{message}\SpecialCharTok{$}\NormalTok{message\_id)}
\NormalTok{  \} }\ControlFlowTok{else}\NormalTok{ \{}
    
    \CommentTok{\# Отправка сообщения о том что у пользователя не достаточно прав}
\NormalTok{    bot}\SpecialCharTok{$}\FunctionTok{sendMessage}\NormalTok{(update}\SpecialCharTok{$}\NormalTok{message}\SpecialCharTok{$}\NormalTok{chat\_id, }
                    \AttributeTok{text =} \FunctionTok{paste0}\NormalTok{(}\StringTok{"У вас нет прав для использования этого метода!"}\NormalTok{),}
                    \AttributeTok{parse\_mode =} \StringTok{"Markdown"}\NormalTok{,}
                    \AttributeTok{reply\_to\_message\_id =}\NormalTok{ update}\SpecialCharTok{$}\NormalTok{message}\SpecialCharTok{$}\NormalTok{message\_id)}
    
\NormalTok{  \}}
                  
  
\NormalTok{\}}

\CommentTok{\# обработчики}
\NormalTok{h\_hello }\OtherTok{\textless{}{-}} \FunctionTok{CommandHandler}\NormalTok{(}\StringTok{\textquotesingle{}say\_hello\textquotesingle{}}\NormalTok{, say\_hello)}
\NormalTok{h\_time  }\OtherTok{\textless{}{-}} \FunctionTok{CommandHandler}\NormalTok{(}\StringTok{\textquotesingle{}what\_time\textquotesingle{}}\NormalTok{, what\_time)}

\CommentTok{\# добавляем обработчики в диспетчер}
\NormalTok{updater }\OtherTok{\textless{}{-}}\NormalTok{ updater }\SpecialCharTok{+}\NormalTok{ h\_hello }\SpecialCharTok{+}\NormalTok{ h\_time}

\CommentTok{\# запускаем бота }
\NormalTok{updater}\SpecialCharTok{$}\FunctionTok{start\_polling}\NormalTok{()}
\end{Highlighting}
\end{Shaded}

Результат будет следующим:

\includegraphics{http://img.netpeak.ua/alsey/160130833437_kiss_162kb.png}

Теперь бот не просто игнорирует сообщения от обычных пользователей, а сообщает им, что у них недостаточно прав использовать какой либо метод.

\hypertarget{ux43eux433ux440ux430ux43dux438ux447ux438ux432ux430ux435ux43c-ux43fux440ux430ux432ux430-ux43dux430-ux443ux440ux43eux432ux43dux435-ux447ux430ux442ux430-1}{%
\subsection{Ограничиваем права на уровне чата}\label{ux43eux433ux440ux430ux43dux438ux447ux438ux432ux430ux435ux43c-ux43fux440ux430ux432ux430-ux43dux430-ux443ux440ux43eux432ux43dux435-ux447ux430ux442ux430-1}}

Думаю у вас уже не возникнет трудностей при доработке ваших методов, таким образом, что бы их можно было использовать только в определённых чатах, тем не менее приведу пример.

Напишем функцию, которая будет проверять входит ли текущий чат в список разрешенных.

\begin{Shaded}
\begin{Highlighting}[]
\NormalTok{bot\_check\_chat\_id }\OtherTok{\textless{}{-}} 
  \ControlFlowTok{function}\NormalTok{(allowed\_chats, current\_chat) \{}

\NormalTok{     current\_chat }\SpecialCharTok{\%in\%}\NormalTok{ allowed\_chats }
    
\NormalTok{\}}
\end{Highlighting}
\end{Shaded}

Далее используем эту функции внутри наших методов:

\begin{Shaded}
\begin{Highlighting}[]
\FunctionTok{library}\NormalTok{(telegram.bot)}


\CommentTok{\# создаём экземпляр класса Updater}
\NormalTok{updater }\OtherTok{\textless{}{-}} \FunctionTok{Updater}\NormalTok{(}\StringTok{\textquotesingle{}ТОКЕН ВАШЕГО БОТА\textquotesingle{}}\NormalTok{)}

\CommentTok{\# Пишем метод для приветсвия}
\DocumentationTok{\#\# команда приветствия}
\NormalTok{say\_hello }\OtherTok{\textless{}{-}} \ControlFlowTok{function}\NormalTok{(bot, update) \{}
  
  \CommentTok{\# Имя пользователя с которым надо поздароваться}
\NormalTok{  user\_name }\OtherTok{\textless{}{-}}\NormalTok{ update}\SpecialCharTok{$}\NormalTok{message}\SpecialCharTok{$}\NormalTok{from}\SpecialCharTok{$}\NormalTok{username}
  
  \CommentTok{\# проверяем разрешено ли использовать данному пользователю этот метод}
  \ControlFlowTok{if}\NormalTok{ ( }\FunctionTok{bot\_check\_usernames}\NormalTok{(}\FunctionTok{c}\NormalTok{(}\StringTok{\textquotesingle{}AlexeySeleznev\textquotesingle{}}\NormalTok{, }\StringTok{\textquotesingle{}user1\textquotesingle{}}\NormalTok{, }\StringTok{\textquotesingle{}user2\textquotesingle{}}\NormalTok{), user\_name) }
       \SpecialCharTok{\&}
       \FunctionTok{bot\_check\_chat\_id}\NormalTok{(}\FunctionTok{c}\NormalTok{(}\DecValTok{194336771}\NormalTok{, }\DecValTok{1}\NormalTok{, }\DecValTok{2}\NormalTok{), update}\SpecialCharTok{$}\NormalTok{message}\SpecialCharTok{$}\NormalTok{chat\_id)) \{}
    
    \CommentTok{\# Отправка сообщения}
\NormalTok{    bot}\SpecialCharTok{$}\FunctionTok{sendMessage}\NormalTok{(update}\SpecialCharTok{$}\NormalTok{message}\SpecialCharTok{$}\NormalTok{chat\_id, }
                    \AttributeTok{text =} \FunctionTok{paste0}\NormalTok{(}\StringTok{"Моё почтение, "}\NormalTok{, user\_name, }\StringTok{"!"}\NormalTok{),}
                    \AttributeTok{parse\_mode =} \StringTok{"Markdown"}\NormalTok{,}
                    \AttributeTok{reply\_to\_message\_id =}\NormalTok{ update}\SpecialCharTok{$}\NormalTok{message}\SpecialCharTok{$}\NormalTok{message\_id)}
      
\NormalTok{  \} }\ControlFlowTok{else}\NormalTok{ \{}
    
    \CommentTok{\# Отправка сообщения}
\NormalTok{    bot}\SpecialCharTok{$}\FunctionTok{sendMessage}\NormalTok{(update}\SpecialCharTok{$}\NormalTok{message}\SpecialCharTok{$}\NormalTok{chat\_id, }
                    \AttributeTok{text =} \FunctionTok{paste0}\NormalTok{(}\StringTok{"У вас нет прав для использования этого метода!"}\NormalTok{),}
                    \AttributeTok{parse\_mode =} \StringTok{"Markdown"}\NormalTok{,}
                    \AttributeTok{reply\_to\_message\_id =}\NormalTok{ update}\SpecialCharTok{$}\NormalTok{message}\SpecialCharTok{$}\NormalTok{message\_id)}
    
\NormalTok{  \}}

\NormalTok{\}}

\DocumentationTok{\#\# команда по которой бот возвращает системную дату и время}
\NormalTok{what\_time }\OtherTok{\textless{}{-}} \ControlFlowTok{function}\NormalTok{(bot, update) \{}
  
  \CommentTok{\# проверяем разрешено ли использовать данному пользователю этот метод}
  \ControlFlowTok{if}\NormalTok{ ( }\FunctionTok{bot\_check\_usernames}\NormalTok{(}\FunctionTok{c}\NormalTok{(}\StringTok{\textquotesingle{}AlexeySeleznev\textquotesingle{}}\NormalTok{, }\StringTok{\textquotesingle{}user1\textquotesingle{}}\NormalTok{, }\StringTok{\textquotesingle{}user2\textquotesingle{}}\NormalTok{), update}\SpecialCharTok{$}\NormalTok{message}\SpecialCharTok{$}\NormalTok{from}\SpecialCharTok{$}\NormalTok{username)}
       \SpecialCharTok{\&}
       \FunctionTok{bot\_check\_chat\_id}\NormalTok{(}\FunctionTok{c}\NormalTok{(}\DecValTok{194336771}\NormalTok{, }\DecValTok{1}\NormalTok{, }\DecValTok{2}\NormalTok{), update}\SpecialCharTok{$}\NormalTok{message}\SpecialCharTok{$}\NormalTok{chat\_id)) \{}
      
    \CommentTok{\# Запрашиваем текущее время}
\NormalTok{    cur\_time }\OtherTok{\textless{}{-}} \FunctionTok{as.character}\NormalTok{(}\FunctionTok{Sys.time}\NormalTok{())}
    
    \CommentTok{\# Отправка сообщения о том что у пользователя не достаточно прав}
\NormalTok{    bot}\SpecialCharTok{$}\FunctionTok{sendMessage}\NormalTok{(update}\SpecialCharTok{$}\NormalTok{message}\SpecialCharTok{$}\NormalTok{chat\_id, }
                    \AttributeTok{text =} \FunctionTok{paste0}\NormalTok{(}\StringTok{"Текущее время, "}\NormalTok{, cur\_time),}
                                  \AttributeTok{parse\_mode =} \StringTok{"Markdown"}\NormalTok{,}
                                  \AttributeTok{reply\_to\_message\_id =}\NormalTok{ update}\SpecialCharTok{$}\NormalTok{message}\SpecialCharTok{$}\NormalTok{message\_id)}
\NormalTok{  \} }\ControlFlowTok{else}\NormalTok{ \{}
    
    \CommentTok{\# Отправка сообщения о том что у пользователя не достаточно прав}
\NormalTok{    bot}\SpecialCharTok{$}\FunctionTok{sendMessage}\NormalTok{(update}\SpecialCharTok{$}\NormalTok{message}\SpecialCharTok{$}\NormalTok{chat\_id, }
                    \AttributeTok{text =} \FunctionTok{paste0}\NormalTok{(}\StringTok{"У вас нет прав для использования этого метода!"}\NormalTok{),}
                    \AttributeTok{parse\_mode =} \StringTok{"Markdown"}\NormalTok{,}
                    \AttributeTok{reply\_to\_message\_id =}\NormalTok{ update}\SpecialCharTok{$}\NormalTok{message}\SpecialCharTok{$}\NormalTok{message\_id)}
    
\NormalTok{  \}}
                  
  
\NormalTok{\}}

\CommentTok{\# обработчики}
\NormalTok{h\_hello }\OtherTok{\textless{}{-}} \FunctionTok{CommandHandler}\NormalTok{(}\StringTok{\textquotesingle{}say\_hello\textquotesingle{}}\NormalTok{, say\_hello)}
\NormalTok{h\_time  }\OtherTok{\textless{}{-}} \FunctionTok{CommandHandler}\NormalTok{(}\StringTok{\textquotesingle{}what\_time\textquotesingle{}}\NormalTok{, what\_time)}

\CommentTok{\# добавляем обработчики в диспетчер}
\NormalTok{updater }\OtherTok{\textless{}{-}}\NormalTok{ updater }\SpecialCharTok{+}\NormalTok{ h\_hello }\SpecialCharTok{+}\NormalTok{ h\_time}

\CommentTok{\# запускаем бота }
\NormalTok{updater}\SpecialCharTok{$}\FunctionTok{start\_polling}\NormalTok{()}
\end{Highlighting}
\end{Shaded}

\hypertarget{ux437ux430ux43aux43bux44eux447ux435ux43dux438ux435-4}{%
\section{Заключение}\label{ux437ux430ux43aux43bux44eux447ux435ux43dux438ux435-4}}

На этом серия статей о построении telegram ботов завершается. Я старался структурировать и подавать материал достаточно сжато, убрав всю воду, но при этом сделать так, что бы материал был вам понятен. Очень надеюсь на то, что мне это удалось.

Успехов вам в ботостроении. В комментариях можете написать примеры ваших ботов, и как вы их на практике используете.

\hypertarget{ux442ux435ux441ux442ux44b-ux438-ux437ux430ux434ux430ux43dux438ux44f-4}{%
\section{Тесты и задания}\label{ux442ux435ux441ux442ux44b-ux438-ux437ux430ux434ux430ux43dux438ux44f-4}}

\hypertarget{ux442ux435ux441ux442ux44b-4}{%
\subsection{Тесты}\label{ux442ux435ux441ux442ux44b-4}}

Для закрепления материла рекомендую вам пройти тест доступный по \href{https://onlinetestpad.com/t/build-tg-bot-in-r-5}{ссылке}.

\hypertarget{ux437ux430ux434ux430ux43dux438ux44f-4}{%
\subsection{Задания}\label{ux437ux430ux434ux430ux43dux438ux44f-4}}

Возьмите \href{-updater.html\#задания-1}{задачу} из второй главы, и ограничьте использование единственного метода, доступного в созданном боте, так, что бы он работал только когда его запрашиваете вы.

\hypertarget{ux43fux43eux432ux44bux448ux430ux435ux43c-ux441ux442ux430ux431ux438ux43bux44cux43dux43eux441ux442ux44c-ux440ux430ux431ux43eux442ux44b-ux431ux43eux442ux430-6}{%
\chapter{Повышаем стабильность работы бота (6)}\label{ux43fux43eux432ux44bux448ux430ux435ux43c-ux441ux442ux430ux431ux438ux43bux44cux43dux43eux441ux442ux44c-ux440ux430ux431ux43eux442ux44b-ux431ux43eux442ux430-6}}

К этому моменту вы знаете уже достаточно для того, что бы решить значительную часть своих задач по ботостроению. Простой бот будет работать достаточно стабильно, но всё равно иногда сервера API Telegram могут давать сбой. Даже если в вашем коде нет ошибок, и пользователи используют его правильно, иногда он может падать.

В этой главе мы поговорим о том, как повысить работоспособность бота за счёт отлавливания и обработки ошибок пуллинга.

\hypertarget{ux43aux43eux43dux441ux442ux440ux443ux43aux446ux438ux44f-trycatch}{%
\section{Конструкция tryCatch()}\label{ux43aux43eux43dux441ux442ux440ux443ux43aux446ux438ux44f-trycatch}}

Повысить работоспособность вашего бота поможет конструкция \texttt{tryCatch()}. Данная конструкция имеет следующий синтаксис:

\begin{Shaded}
\begin{Highlighting}[]
\FunctionTok{tryCatch}\NormalTok{(}\AttributeTok{expr =}\NormalTok{ \{}
  
    \SpecialCharTok{\textasciitilde{}}\NormalTok{ Тут код который будет выполняться }\SpecialCharTok{\textasciitilde{}}
  
\NormalTok{\}, }
  \AttributeTok{error =} \ControlFlowTok{function}\NormalTok{(err) \{}
    
    \SpecialCharTok{\textasciitilde{}}\NormalTok{ код который будет выполняться в случае возникновения ошибки в блоке expr }\SpecialCharTok{\textasciitilde{}}
    
\NormalTok{  \}, }
  \AttributeTok{finally =}\NormalTok{ \{}
    
    \SpecialCharTok{\textasciitilde{}}\NormalTok{ Код который будет выполняться в любом случае, не зависимо от того закончилось выражение expr ошибкой или нет }\SpecialCharTok{\textasciitilde{}}
    
\NormalTok{  \})}
\end{Highlighting}
\end{Shaded}

\hypertarget{ux43bux43eux433ux438ux43aux430-ux440ux430ux431ux43eux442ux44b-ux43aux43eux43dux441ux442ux440ux443ux43aux446ux438ux438-trycatch}{%
\section{Логика работы конструкции tryCatch()}\label{ux43bux43eux433ux438ux43aux430-ux440ux430ux431ux43eux442ux44b-ux43aux43eux43dux441ux442ux440ux443ux43aux446ux438ux438-trycatch}}

Из описанного синтаксиса понятно, что вам необходимо завернуть выражение в фигурные скобки в аргументе \texttt{expr}. Это выражение будет выполняться либо до тех пор, пока не встретится ошибка, либо если ошибки нет, оно будет выполнено полностью.

Если в выражении переданном в \texttt{expr} встречается ошибка, то конструкция \texttt{tryCath()} запустит анонимную функцию, которую вы передали в блоке \texttt{error}.

В любом случае, не зависимо от того, встретилась в выражении \texttt{expr} ошибка или нет, в завершении выполнения будет выполнен код, переданный в аргумент \texttt{finally}.

Если вы хотите более подробно узнать про конструкцию \texttt{tryCatch()} посмотрите этот \href{https://youtu.be/GvmjW34IHu8}{видео урок}.

\hypertarget{ux438ux441ux43fux43eux43bux44cux437ux443ux435ux43c-trycatch-ux432ux43dux443ux442ux440ux438-ux431ux43eux442ux430}{%
\section{Используем tryCatch() внутри бота}\label{ux438ux441ux43fux43eux43bux44cux437ux443ux435ux43c-trycatch-ux432ux43dux443ux442ux440ux438-ux431ux43eux442ux430}}

По большому счёту вы можете использовать \texttt{tryCatch()} внутри каждой функции вашего бота. Но можно убить всех зайцев одним выстрелом.

В разработке ботов слабым местом является пуллинг, т.е. метод \texttt{updater\$start\_polling()}. Пуллинг - это бесконечный цикл, именно он выполняется всё время работы бота, и даёт сбой если пользователь неправильно использовал бота, или API Telegram не отправил вам ответ. Соответственно если завернуть пуллинг в \texttt{tryCatch()}, и перезапускать вашего бота в бота в блоке \texttt{finally} то при любой ошибке он будет самостоятельно перезапускаться.

Перед перезапуском бота не забывайте очистить его апдейты, что бы избавиться от ошибки, которая вызвала падение бота.

Выглядеть такой пуллинг будет следующим образом:

\begin{Shaded}
\begin{Highlighting}[]
\FunctionTok{tryCatch}\NormalTok{(}
  
  \CommentTok{\# запускаем пуллинг}
  \AttributeTok{expr =}\NormalTok{ updater}\SpecialCharTok{$}\FunctionTok{start\_polling}\NormalTok{(), }
  
  \CommentTok{\# действия при ошибке пуллинга}
  \AttributeTok{error =} \ControlFlowTok{function}\NormalTok{(err) \{}
    
    \CommentTok{\# бот для оповещения}
\NormalTok{    bot }\OtherTok{\textless{}{-}} \FunctionTok{Bot}\NormalTok{(}\AttributeTok{token =} \FunctionTok{bot\_token}\NormalTok{(}\StringTok{"Токен вашего бота"}\NormalTok{))}
    
    \CommentTok{\# чат для оповещения}
\NormalTok{    chat\_id }\OtherTok{\textless{}{-}} \StringTok{"Идентификатор чата в который необходимо отправить сообщение"}
    
    \CommentTok{\# сообщение}
\NormalTok{    msg }\OtherTok{\textless{}{-}} \FunctionTok{str\_glue}\NormalTok{(}\StringTok{"*Бот упал*: Ошибка (\_\{err$message\}\_)."}\NormalTok{)}
    
\NormalTok{    bot}\SpecialCharTok{$}\FunctionTok{sendMessage}\NormalTok{(}\AttributeTok{chat\_id =}\NormalTok{ chat\_id, }
                    \AttributeTok{text =}\NormalTok{ msg,}
                    \AttributeTok{parse\_mode =} \StringTok{\textquotesingle{}Markdown\textquotesingle{}}\NormalTok{)}
    
    \CommentTok{\# очищаем полученный апдейт бота, который вызвал ошибку}
\NormalTok{    updater}\SpecialCharTok{$}\NormalTok{bot}\SpecialCharTok{$}\FunctionTok{clean\_updates}\NormalTok{()}
    
    \CommentTok{\# информация о том, что бот будет перезапущен}
\NormalTok{    bot}\SpecialCharTok{$}\FunctionTok{sendMessage}\NormalTok{(}\AttributeTok{chat\_id =}\NormalTok{ chat\_id, }
                    \AttributeTok{text =} \FunctionTok{str\_glue}\NormalTok{(}\StringTok{\textquotesingle{}*Перезапускаю бота* в \{Sys.time()\}\textquotesingle{}}\NormalTok{),}
                    \AttributeTok{parse\_mode =} \StringTok{\textquotesingle{}Markdown\textquotesingle{}}\NormalTok{)}

    
\NormalTok{  \}, }
  \CommentTok{\# действия которые будут выполненны в любом случае}
  \AttributeTok{finally =}\NormalTok{ \{}
    
    \CommentTok{\# останавливаем пулинг}
\NormalTok{    updater}\SpecialCharTok{$}\FunctionTok{stop\_polling}\NormalTok{()}
        
    \CommentTok{\# перезапускаем скрипт бота}
    \FunctionTok{source}\NormalTok{(}\StringTok{\textquotesingle{}C:}\SpecialCharTok{\textbackslash{}\textbackslash{}}\StringTok{telegram\_bot}\SpecialCharTok{\textbackslash{}\textbackslash{}}\StringTok{my\_bot.R\textquotesingle{}}\NormalTok{) }

\NormalTok{  \}}
\NormalTok{)}
\end{Highlighting}
\end{Shaded}

В приведённом выше коде вам необходимо подставить токен созданного вами бота, и указать ID чата, в который бот будет отправлять уведомление о падении пуллинга.

В блок \texttt{expr} мы завернули процесс пуллинга, таким образом он постоянно контролируется конструкцией \texttt{tryCatch}.

Далее в блок \texttt{error} мы передали безымянную функцию, которая принимает всего один аргумент \texttt{err}, т.е. саму ошибку. Сообщение об ошибке мы получаем через \texttt{err\$message}, и отправляем в указанный чат. С помощью \texttt{updater\$bot\$clean\_updates()} мы очищаем очередь апдейтов бота, т.к. последний апдейт вызвал ошибку и падение нашего бота.

В блоке \texttt{finally} мы останавливаем пуллинг, и командой \texttt{source(\textquotesingle{}C:\textbackslash{}\textbackslash{}telegram\_bot\textbackslash{}\textbackslash{}my\_bot.R\textquotesingle{})} занова запускаем скрипт с ботом.

Такая схема позволяет боту очищаться и подниматься при любой ошибке пуллинга.

Очищать апдейты бота с помощью комманды \texttt{updater\$bot\$clean\_updates()} можно так же и при запуске бота, указав эту команду сразу, после инициализации объекта бота.

\hypertarget{ux440ux435ux448ux435ux43dux438ux435-ux437ux430ux434ux430ux447-tasks}{%
\chapter*{Решение задач (tasks)}\label{ux440ux435ux448ux435ux43dux438ux435-ux437ux430ux434ux430ux447-tasks}}
\addcontentsline{toc}{chapter}{Решение задач (tasks)}

В этом разделе книги приведены решения всех, представленных в учебнике задач.

\hypertarget{ux437ux430ux434ux430ux447ux430-1.1}{%
\section*{Задача 1.1}\label{ux437ux430ux434ux430ux447ux430-1.1}}
\addcontentsline{toc}{section}{Задача 1.1}

\begin{enumerate}
\def\labelenumi{\arabic{enumi}.}
\tightlist
\item
  Создайте с помощью \href{http://t.me/BotFather}{BotFather} бота.
\item
  Перейдите к диалогу с ботом, и узнайте идентификатор вашего с ботом чата.
\item
  Отправьте с помощью созданного бота в telegram первые 20 строк из встроенного в R набора данных \texttt{ToothGrowth}.
\end{enumerate}

\emph{Решение:}

\begin{Shaded}
\begin{Highlighting}[]
\FunctionTok{library}\NormalTok{(purrr)}
\FunctionTok{library}\NormalTok{(tidyr)}
\FunctionTok{library}\NormalTok{(stringr)}
\FunctionTok{library}\NormalTok{(telegram.bot)}

\CommentTok{\# функция для перевода data.frame в telegram таблицу }
\NormalTok{to\_tg\_table }\OtherTok{\textless{}{-}} \ControlFlowTok{function}\NormalTok{( table, }\AttributeTok{align =} \ConstantTok{NULL}\NormalTok{, }\AttributeTok{indents =} \DecValTok{3}\NormalTok{, }\AttributeTok{parse\_mode =} \StringTok{\textquotesingle{}Markdown\textquotesingle{}}\NormalTok{ ) \{}
  
  \CommentTok{\# если выравнивание не задано то выравниваем по левому краю}
  \ControlFlowTok{if}\NormalTok{ ( }\FunctionTok{is.null}\NormalTok{(align) ) \{}
    
\NormalTok{    col\_num }\OtherTok{\textless{}{-}} \FunctionTok{length}\NormalTok{(table)}
\NormalTok{    align   }\OtherTok{\textless{}{-}} \FunctionTok{str\_c}\NormalTok{( }\FunctionTok{rep}\NormalTok{(}\StringTok{\textquotesingle{}l\textquotesingle{}}\NormalTok{, col\_num), }\AttributeTok{collapse =} \StringTok{\textquotesingle{}\textquotesingle{}}\NormalTok{ )}
    
\NormalTok{  \}}
  
  \CommentTok{\# проверяем правильно ли заданно выравнивание}
  \ControlFlowTok{if}\NormalTok{ ( }\FunctionTok{length}\NormalTok{(table) }\SpecialCharTok{!=} \FunctionTok{nchar}\NormalTok{(align) ) \{}
    
\NormalTok{    align }\OtherTok{\textless{}{-}} \ConstantTok{NULL}
    
\NormalTok{  \}}
  
  \CommentTok{\# новое выравнивание столбцов }
\NormalTok{  side }\OtherTok{\textless{}{-}} \FunctionTok{sapply}\NormalTok{(}\DecValTok{1}\SpecialCharTok{:}\FunctionTok{nchar}\NormalTok{(align), }
                 \ControlFlowTok{function}\NormalTok{(x) \{ }
\NormalTok{                   letter }\OtherTok{\textless{}{-}} \FunctionTok{substr}\NormalTok{(align, x, x)}
                   \ControlFlowTok{switch}\NormalTok{ (letter,}
                           \StringTok{\textquotesingle{}l\textquotesingle{}} \OtherTok{=} \StringTok{\textquotesingle{}right\textquotesingle{}}\NormalTok{,}
                           \StringTok{\textquotesingle{}r\textquotesingle{}} \OtherTok{=} \StringTok{\textquotesingle{}left\textquotesingle{}}\NormalTok{,}
                           \StringTok{\textquotesingle{}c\textquotesingle{}} \OtherTok{=} \StringTok{\textquotesingle{}both\textquotesingle{}}\NormalTok{,}
                           \StringTok{\textquotesingle{}left\textquotesingle{}}
\NormalTok{                   )}
\NormalTok{                 \})}
  
  \CommentTok{\# сохраняем имена}
\NormalTok{  t\_names      }\OtherTok{\textless{}{-}} \FunctionTok{names}\NormalTok{(table)}
  
  \CommentTok{\# вычисляем ширину столбцов}
\NormalTok{  names\_length }\OtherTok{\textless{}{-}} \FunctionTok{sapply}\NormalTok{(t\_names, nchar) }
\NormalTok{  value\_length }\OtherTok{\textless{}{-}} \FunctionTok{sapply}\NormalTok{(table, }\ControlFlowTok{function}\NormalTok{(x) }\FunctionTok{max}\NormalTok{(}\FunctionTok{nchar}\NormalTok{(}\FunctionTok{as.character}\NormalTok{(x))))}
\NormalTok{  max\_length   }\OtherTok{\textless{}{-}} \FunctionTok{ifelse}\NormalTok{(value\_length }\SpecialCharTok{\textgreater{}}\NormalTok{ names\_length, value\_length, names\_length)}
  
  \CommentTok{\# подгоняем размер имён столбцов под их ширину + указанное в indents к{-}во пробелов }
\NormalTok{  t\_names }\OtherTok{\textless{}{-}} \FunctionTok{mapply}\NormalTok{(str\_pad, }
                    \AttributeTok{string =}\NormalTok{ t\_names, }
                    \AttributeTok{width  =}\NormalTok{ max\_length }\SpecialCharTok{+}\NormalTok{ indents, }
                    \AttributeTok{side   =}\NormalTok{ side)}
  
  \CommentTok{\# объединяем названия столбцов}
\NormalTok{  str\_names }\OtherTok{\textless{}{-}} \FunctionTok{str\_c}\NormalTok{(t\_names, }\AttributeTok{collapse =} \StringTok{\textquotesingle{}\textquotesingle{}}\NormalTok{)}
  
  \CommentTok{\# аргументы для фукнции str\_pad}
\NormalTok{  rules }\OtherTok{\textless{}{-}} \FunctionTok{list}\NormalTok{(}\AttributeTok{string =}\NormalTok{ table, }\AttributeTok{width =}\NormalTok{ max\_length }\SpecialCharTok{+}\NormalTok{ indents, }\AttributeTok{side =}\NormalTok{ side)}
  
  \CommentTok{\# поочереди переводим каждый столбец к нужному виду}
\NormalTok{  t\_str }\OtherTok{\textless{}{-}}   \FunctionTok{pmap\_df}\NormalTok{( rules, str\_pad )}\SpecialCharTok{\%\textgreater{}\%}
    \FunctionTok{unite}\NormalTok{(}\StringTok{"data"}\NormalTok{, }\FunctionTok{everything}\NormalTok{(), }\AttributeTok{remove =} \ConstantTok{TRUE}\NormalTok{, }\AttributeTok{sep =} \StringTok{\textquotesingle{}\textquotesingle{}}\NormalTok{) }\SpecialCharTok{\%\textgreater{}\%}
    \FunctionTok{unlist}\NormalTok{(data) }\SpecialCharTok{\%\textgreater{}\%}
    \FunctionTok{str\_c}\NormalTok{(}\AttributeTok{collapse =} \StringTok{\textquotesingle{}}\SpecialCharTok{\textbackslash{}n}\StringTok{\textquotesingle{}}\NormalTok{) }
  
  \CommentTok{\# если таблица занимает более 4096 символов обрезаем её}
  \ControlFlowTok{if}\NormalTok{ ( }\FunctionTok{nchar}\NormalTok{(t\_str) }\SpecialCharTok{\textgreater{}=} \DecValTok{4021}\NormalTok{ ) \{}
    
    \FunctionTok{warning}\NormalTok{(}\StringTok{\textquotesingle{}Таблица составляет более 4096 символов!\textquotesingle{}}\NormalTok{)}
\NormalTok{    t\_str }\OtherTok{\textless{}{-}} \FunctionTok{substr}\NormalTok{(t\_str, }\DecValTok{1}\NormalTok{, }\DecValTok{4021}\NormalTok{)}
    
\NormalTok{  \}}
  
  \CommentTok{\# символы выделения блока кода согласно выбранной разметке}
\NormalTok{  code\_block }\OtherTok{\textless{}{-}} \ControlFlowTok{switch}\NormalTok{(parse\_mode, }
                       \StringTok{\textquotesingle{}Markdown\textquotesingle{}} \OtherTok{=} \FunctionTok{c}\NormalTok{(}\StringTok{\textquotesingle{}\textasciigrave{}\textasciigrave{}\textasciigrave{}\textquotesingle{}}\NormalTok{, }\StringTok{\textquotesingle{}\textasciigrave{}\textasciigrave{}\textasciigrave{}\textquotesingle{}}\NormalTok{),}
                       \StringTok{\textquotesingle{}HTML\textquotesingle{}} \OtherTok{=} \FunctionTok{c}\NormalTok{(}\StringTok{\textquotesingle{}\textless{}code\textgreater{}\textquotesingle{}}\NormalTok{, }\StringTok{\textquotesingle{}\textless{}/code\textgreater{}\textquotesingle{}}\NormalTok{))}
  
  \CommentTok{\# переводим в code}
\NormalTok{  res }\OtherTok{\textless{}{-}} \FunctionTok{str\_c}\NormalTok{(code\_block[}\DecValTok{1}\NormalTok{], str\_names, t\_str, code\_block[}\DecValTok{2}\NormalTok{], }\AttributeTok{sep =} \StringTok{\textquotesingle{}}\SpecialCharTok{\textbackslash{}n}\StringTok{\textquotesingle{}}\NormalTok{)}
  
  \FunctionTok{return}\NormalTok{(res)}
\NormalTok{\}}

\CommentTok{\# создаём экземпляр бота}
\NormalTok{bot }\OtherTok{\textless{}{-}} \FunctionTok{Bot}\NormalTok{(}\StringTok{\textquotesingle{}1165649194:AAFkDqIzQ6Wq5GV0YU7PmEZcv1gmWIFIB\_8\textquotesingle{}}\NormalTok{)}

\CommentTok{\# получаем ID чата }
\CommentTok{\# (предварительно отправьте боту любое сообщение)}
\NormalTok{chat\_id }\OtherTok{\textless{}{-}}\NormalTok{ bot}\SpecialCharTok{$}\FunctionTok{getUpdates}\NormalTok{()[[}\DecValTok{1}\NormalTok{]]}\SpecialCharTok{$}\FunctionTok{from\_chat\_id}\NormalTok{()}

\CommentTok{\# преоразуем таблицу ToothGrowth}
\NormalTok{TG }\OtherTok{\textless{}{-}} \FunctionTok{to\_tg\_table}\NormalTok{( }\FunctionTok{head}\NormalTok{(ToothGrowth, }\DecValTok{20}\NormalTok{) )}

\CommentTok{\# отправляем таблицу в Telegram}
\NormalTok{bot}\SpecialCharTok{$}\FunctionTok{sendMessage}\NormalTok{(chat\_id, }
\NormalTok{                TG,}
                \StringTok{\textquotesingle{}Markdown\textquotesingle{}}\NormalTok{)}
\end{Highlighting}
\end{Shaded}

\hypertarget{ux437ux430ux434ux430ux447ux430-2.1}{%
\section*{Задача 2.1}\label{ux437ux430ux434ux430ux447ux430-2.1}}
\addcontentsline{toc}{section}{Задача 2.1}

Создайте бота, который будет по команде \texttt{/sum} и переданное в качестве дополнительных параметров произвольное количество перечисленных через пробел чисел, возвращать их сумму.

\emph{Решение:}

\begin{Shaded}
\begin{Highlighting}[]
\FunctionTok{library}\NormalTok{(telegram.bot)}

\CommentTok{\# Создаём жкземпляр класса Updater}
\NormalTok{updater }\OtherTok{\textless{}{-}} \FunctionTok{Updater}\NormalTok{(}\StringTok{\textquotesingle{}ТОКЕН ВАШЕГО БОТА\textquotesingle{}}\NormalTok{)}

\CommentTok{\# Создаём функцию, которая будет суммировать переданные числа}
\NormalTok{summing }\OtherTok{\textless{}{-}} \ControlFlowTok{function}\NormalTok{(bot, update, args) \{}

  \CommentTok{\# Переводим полученный вектор параметров в числа и суммируем}
\NormalTok{  x }\OtherTok{\textless{}{-}} \FunctionTok{sum}\NormalTok{(}\FunctionTok{as.integer}\NormalTok{(args))}

  \CommentTok{\# создаём сообщение}
\NormalTok{  msg }\OtherTok{\textless{}{-}} \FunctionTok{paste0}\NormalTok{(}\StringTok{\textquotesingle{}Сумма переданных чисел: \textquotesingle{}}\NormalTok{, x)}

  \CommentTok{\# отправляем результат}
\NormalTok{  bot}\SpecialCharTok{$}\FunctionTok{sendMessage}\NormalTok{(update}\SpecialCharTok{$}\NormalTok{message}\SpecialCharTok{$}\NormalTok{chat\_id, msg, }\StringTok{\textquotesingle{}Markdown\textquotesingle{}}\NormalTok{)}

\NormalTok{\}}

\CommentTok{\# создаём обработчик}
\NormalTok{h\_sum }\OtherTok{\textless{}{-}} \FunctionTok{CommandHandler}\NormalTok{(}\StringTok{\textquotesingle{}sum\textquotesingle{}}\NormalTok{, summing, }\AttributeTok{pass\_args =} \ConstantTok{TRUE}\NormalTok{)}

\CommentTok{\# добавляем обработчик в диспетчер}
\NormalTok{updater }\OtherTok{\textless{}{-}}\NormalTok{ updater }\SpecialCharTok{+}\NormalTok{ h\_sum}

\CommentTok{\# запускаем бота}
\NormalTok{updater}\SpecialCharTok{$}\FunctionTok{start\_polling}\NormalTok{()}
\end{Highlighting}
\end{Shaded}

\hypertarget{ux437ux430ux434ux430ux447ux430-3.1}{%
\section*{Задача 3.1}\label{ux437ux430ux434ux430ux447ux430-3.1}}
\addcontentsline{toc}{section}{Задача 3.1}

\begin{enumerate}
\def\labelenumi{\arabic{enumi}.}
\tightlist
\item
  Создайте бота, который будет поддерживать Reply клавиатуру. На Reply клавиатуре будет всего одна кнопка ``Время''. По нажатию на неё будет появляться Inline клавиатура с выбором из 6 часовых поясов.
\end{enumerate}

\begin{itemize}
\tightlist
\item
  Africa/Cairo
\item
  America/Chicago
\item
  Europe/Moscow
\item
  Asia/Bangkok
\item
  Europe/Kiev
\item
  Australia/Sydney
\end{itemize}

Кнопки Inline клавиатуры необходимо расположить по 2 в ряд, соответвенно в три ряда.

По нажатию на одну из кнопки Inline клавиатуры бот будет запрашивать информацию по текущему времени из API \href{http://worldtimeapi.org/}{worldtimeapi.org}.

Формат запроса к API: \texttt{http://worldtimeapi.org/api/timezone/\{area\}/:\{location\}}.

Где \texttt{\{area\}} это континент, например Europe, а \texttt{\{location\}} это город, например Kiev. Дату и время надо брать в ответе из компонента \texttt{datetime}.

\emph{Решение:}

\begin{Shaded}
\begin{Highlighting}[]
\FunctionTok{library}\NormalTok{(telegram.bot)}
\FunctionTok{library}\NormalTok{(httr)}
\FunctionTok{library}\NormalTok{(stringr)}

\CommentTok{\# Создаём жкземпляр класса Updater}
\NormalTok{updater }\OtherTok{\textless{}{-}} \FunctionTok{Updater}\NormalTok{(}\StringTok{\textquotesingle{}ТОКЕН ВАШЕГО БОТА\textquotesingle{}}\NormalTok{)}

\CommentTok{\# Запуск клавиатуры}
\NormalTok{start }\OtherTok{\textless{}{-}} \ControlFlowTok{function}\NormalTok{(bot, update) \{}
  
  \CommentTok{\# строим Reply клавиатуру}
\NormalTok{  RKM }\OtherTok{\textless{}{-}} \FunctionTok{ReplyKeyboardMarkup}\NormalTok{(}
    \AttributeTok{keyboard =} \FunctionTok{list}\NormalTok{(}
      \FunctionTok{list}\NormalTok{(}
        \FunctionTok{KeyboardButton}\NormalTok{(}\StringTok{\textquotesingle{}Время\textquotesingle{}}\NormalTok{)}
\NormalTok{      )}
\NormalTok{    ))}
  
  \CommentTok{\# отпралвяем Reply клавиатуру}
\NormalTok{  bot}\SpecialCharTok{$}\FunctionTok{sendMessage}\NormalTok{(update}\SpecialCharTok{$}\NormalTok{message}\SpecialCharTok{$}\NormalTok{chat\_id, }
                  \StringTok{\textquotesingle{}Выберите команду\textquotesingle{}}\NormalTok{, }
                  \StringTok{\textquotesingle{}Markdown\textquotesingle{}}\NormalTok{,}
                  \AttributeTok{reply\_markup =}\NormalTok{ RKM)}
  
\NormalTok{\}}

\CommentTok{\# Отправляем inline клавиатуру}
\NormalTok{inline }\OtherTok{\textless{}{-}} \ControlFlowTok{function}\NormalTok{(bot, update) \{}
  
\NormalTok{  IKM }\OtherTok{\textless{}{-}} \FunctionTok{InlineKeyboardMarkup}\NormalTok{(}
              \AttributeTok{inline\_keyboard =} 
                \FunctionTok{list}\NormalTok{(}
                  \FunctionTok{list}\NormalTok{(}
                    \FunctionTok{InlineKeyboardButton}\NormalTok{(}\AttributeTok{text =} \StringTok{\textquotesingle{}Africa/Cairo\textquotesingle{}}\NormalTok{, }\AttributeTok{callback\_data =} \StringTok{\textquotesingle{}Africa/Cairo\textquotesingle{}}\NormalTok{),}
                    \FunctionTok{InlineKeyboardButton}\NormalTok{(}\AttributeTok{text =} \StringTok{\textquotesingle{}America/Chicago\textquotesingle{}}\NormalTok{, }\AttributeTok{callback\_data =} \StringTok{\textquotesingle{}America/Chicago\textquotesingle{}}\NormalTok{)}
\NormalTok{                  ),}
                  \FunctionTok{list}\NormalTok{(}
                    \FunctionTok{InlineKeyboardButton}\NormalTok{(}\AttributeTok{text =} \StringTok{\textquotesingle{}Europe/Moscow\textquotesingle{}}\NormalTok{, }\AttributeTok{callback\_data =} \StringTok{\textquotesingle{}Europe/Moscow\textquotesingle{}}\NormalTok{),}
                    \FunctionTok{InlineKeyboardButton}\NormalTok{(}\AttributeTok{text =} \StringTok{\textquotesingle{}Asia/Bangkok\textquotesingle{}}\NormalTok{, }\AttributeTok{callback\_data =} \StringTok{\textquotesingle{}Asia/Bangkok\textquotesingle{}}\NormalTok{)}
\NormalTok{                  ),}
                  \FunctionTok{list}\NormalTok{(}
                    \FunctionTok{InlineKeyboardButton}\NormalTok{(}\AttributeTok{text =} \StringTok{\textquotesingle{}Europe/Kiev\textquotesingle{}}\NormalTok{, }\AttributeTok{callback\_data =} \StringTok{\textquotesingle{}Europe/Kiev\textquotesingle{}}\NormalTok{),}
                    \FunctionTok{InlineKeyboardButton}\NormalTok{(}\AttributeTok{text =} \StringTok{\textquotesingle{}Australia/Sydney\textquotesingle{}}\NormalTok{, }\AttributeTok{callback\_data =} \StringTok{\textquotesingle{}Australia/Sydney\textquotesingle{}}\NormalTok{)}
\NormalTok{                  )}
\NormalTok{                ))}
  
  \CommentTok{\# отпралвяем Reply клавиатуру}
\NormalTok{  bot}\SpecialCharTok{$}\FunctionTok{sendMessage}\NormalTok{(update}\SpecialCharTok{$}\NormalTok{message}\SpecialCharTok{$}\NormalTok{chat\_id, }
                  \StringTok{\textquotesingle{}Выберите регион\textquotesingle{}}\NormalTok{, }
                  \StringTok{\textquotesingle{}Markdown\textquotesingle{}}\NormalTok{,}
                  \AttributeTok{reply\_markup =}\NormalTok{ IKM)}
  
\NormalTok{\}}

\CommentTok{\# обрабатываем нажатие на кнопку}
\NormalTok{curtime }\OtherTok{\textless{}{-}} \ControlFlowTok{function}\NormalTok{(bot, update) \{}
  
  \CommentTok{\# сообщаем боту, что запрос с кнопки принят}
\NormalTok{  bot}\SpecialCharTok{$}\FunctionTok{answerCallbackQuery}\NormalTok{(}\AttributeTok{callback\_query\_id =}\NormalTok{ update}\SpecialCharTok{$}\NormalTok{callback\_query}\SpecialCharTok{$}\NormalTok{id) }
  
  \CommentTok{\# данные с кнопки}
\NormalTok{  data }\OtherTok{\textless{}{-}}\NormalTok{ update}\SpecialCharTok{$}\NormalTok{callback\_query}\SpecialCharTok{$}\NormalTok{data}
  
  \CommentTok{\# разбиваем на регион и город}
\NormalTok{  geo }\OtherTok{\textless{}{-}} \FunctionTok{unlist}\NormalTok{(}\FunctionTok{strsplit}\NormalTok{(data, }\AttributeTok{split =} \StringTok{\textquotesingle{}/\textquotesingle{}}\NormalTok{))}
  
  \CommentTok{\# компонуем URL}
\NormalTok{  url }\OtherTok{\textless{}{-}} \FunctionTok{str\_glue}\NormalTok{(}\StringTok{\textquotesingle{}http://worldtimeapi.org/api/timezone/\{geo[1]\}/\{geo[2]\}\textquotesingle{}}\NormalTok{)}
  
  \CommentTok{\# запрос к API}
\NormalTok{  answer }\OtherTok{\textless{}{-}} \FunctionTok{GET}\NormalTok{(url)}
  
  \CommentTok{\# парсим ответ}
\NormalTok{  res }\OtherTok{\textless{}{-}} \FunctionTok{content}\NormalTok{(answer)}
  
  \CommentTok{\# создаём сообщение}
\NormalTok{  msg }\OtherTok{\textless{}{-}} \FunctionTok{str\_glue}\NormalTok{(}\StringTok{\textquotesingle{}Текущее время в \{data\}: \{res$datetime\}\textquotesingle{}}\NormalTok{)}
  
  \CommentTok{\# отправляем сообщение}
\NormalTok{  bot}\SpecialCharTok{$}\FunctionTok{sendMessage}\NormalTok{(update}\SpecialCharTok{$}\FunctionTok{from\_chat\_id}\NormalTok{(), }
\NormalTok{                  msg, }
                  \StringTok{\textquotesingle{}Markdown\textquotesingle{}}\NormalTok{)}
\NormalTok{\}}

\CommentTok{\# Фильтр для Reply клавиатуры}
\NormalTok{MessageFilters}\SpecialCharTok{$}\NormalTok{start }\OtherTok{\textless{}{-}} 
  \FunctionTok{BaseFilter}\NormalTok{(}
    \ControlFlowTok{function}\NormalTok{(message) \{}
\NormalTok{      message}\SpecialCharTok{$}\NormalTok{text }\SpecialCharTok{==} \StringTok{\textquotesingle{}Время\textquotesingle{}}
\NormalTok{    \}}
\NormalTok{  )}

\CommentTok{\# Обработчики}
\NormalTok{h\_start }\OtherTok{\textless{}{-}} \FunctionTok{CommandHandler}\NormalTok{(}\StringTok{\textquotesingle{}start\textquotesingle{}}\NormalTok{, start)}
\NormalTok{h\_time  }\OtherTok{\textless{}{-}} \FunctionTok{MessageHandler}\NormalTok{(inline, MessageFilters}\SpecialCharTok{$}\NormalTok{start)}
\NormalTok{h\_cb    }\OtherTok{\textless{}{-}} \FunctionTok{CallbackQueryHandler}\NormalTok{(curtime)}

\CommentTok{\# Добавляем обработчики в диспетчер}
\NormalTok{updater }\OtherTok{\textless{}{-}}\NormalTok{ updater }\SpecialCharTok{+}\NormalTok{ h\_start }\SpecialCharTok{+}\NormalTok{ h\_time }\SpecialCharTok{+}\NormalTok{ h\_cb}

\CommentTok{\# Запускаем бота}
\NormalTok{updater}\SpecialCharTok{$}\FunctionTok{start\_polling}\NormalTok{()}
\end{Highlighting}
\end{Shaded}

\hypertarget{ux437ux430ux434ux430ux447ux430-4.1}{%
\section*{Задача 4.1}\label{ux437ux430ux434ux430ux447ux430-4.1}}
\addcontentsline{toc}{section}{Задача 4.1}

Постройте бота который будет поддерживать игру угадай число. Т.е. по команде \texttt{/start} бот будет загадывать число от 1 до 50. Далее у вас будет 5 попыток угадать это число.

Вы по очереди в каждой из попыток вводите числа, если введённое число меньше чем то, которое загадал бот то бот пишет ``моё число больше'', иначе бот пишет ``моё число меньше''. Если вы ввели правильное число то бот пишет что вы выйграли, и переводит диалог в исходное состояние.

\emph{Решение:}

Создаём таблицу в базе данных для хранеия числа и текущей попытки.

\begin{Shaded}
\begin{Highlighting}[]
\KeywordTok{CREATE} \KeywordTok{TABLE}\NormalTok{ chat\_data (}
\NormalTok{    chat\_id BIGINT  }\KeywordTok{PRIMARY} \KeywordTok{KEY}
                    \KeywordTok{UNIQUE}\NormalTok{,}
\NormalTok{    attempt    }\DataTypeTok{INTEGER}\NormalTok{,}
    \DataTypeTok{number}     \DataTypeTok{INTEGER}
\NormalTok{);}
\end{Highlighting}
\end{Shaded}

Далее создаём функции для работы с бахой данных.

\begin{Shaded}
\begin{Highlighting}[]
\CommentTok{\# write chat data}
\CommentTok{\# write chat data}
\NormalTok{set\_chat\_data }\OtherTok{\textless{}{-}} \ControlFlowTok{function}\NormalTok{(chat\_id, field, value) \{}
  
  
\NormalTok{  con }\OtherTok{\textless{}{-}} \FunctionTok{dbConnect}\NormalTok{(}\FunctionTok{SQLite}\NormalTok{(), db)}
  
  \CommentTok{\# upsert состояние чата}
  \FunctionTok{dbExecute}\NormalTok{(con, }
            \FunctionTok{str\_interp}\NormalTok{(}\StringTok{"}
\StringTok{            INSERT INTO chat\_data (chat\_id, $\{field\})}
\StringTok{                VALUES($\{chat\_id\}, \textquotesingle{}$\{value\}\textquotesingle{}) }
\StringTok{                ON CONFLICT(chat\_id) }
\StringTok{                DO UPDATE SET $\{field\}=\textquotesingle{}$\{value\}\textquotesingle{};}
\StringTok{            "}\NormalTok{)}
\NormalTok{  )}
  
  \FunctionTok{dbDisconnect}\NormalTok{(con)}
  
\NormalTok{\}}

\CommentTok{\# read chat data}
\NormalTok{get\_chat\_data }\OtherTok{\textless{}{-}} \ControlFlowTok{function}\NormalTok{(chat\_id, field) \{}
  
  
\NormalTok{  con }\OtherTok{\textless{}{-}} \FunctionTok{dbConnect}\NormalTok{(}\FunctionTok{SQLite}\NormalTok{(), db)}
  
  \CommentTok{\# upsert состояние чата}
\NormalTok{  data }\OtherTok{\textless{}{-}} \FunctionTok{dbGetQuery}\NormalTok{(con, }
                     \FunctionTok{str\_interp}\NormalTok{(}\StringTok{"}
\StringTok{            SELECT $\{field\}}
\StringTok{            FROM chat\_data}
\StringTok{            WHERE chat\_id = $\{chat\_id\};}
\StringTok{            "}\NormalTok{)}
\NormalTok{  )}
  
  \FunctionTok{dbDisconnect}\NormalTok{(con)}
  
  \FunctionTok{return}\NormalTok{(data[[field]])}
  
\NormalTok{\}}
\end{Highlighting}
\end{Shaded}

Основной код бота выглядит так:

\begin{Shaded}
\begin{Highlighting}[]
\FunctionTok{library}\NormalTok{(RSQLite)}
\FunctionTok{library}\NormalTok{(DBI)}
\FunctionTok{library}\NormalTok{(telegram.bot)}
\FunctionTok{library}\NormalTok{(stringr)}

\CommentTok{\# Создаём жкземпляр класса Updater}
\NormalTok{updater }\OtherTok{\textless{}{-}} \FunctionTok{Updater}\NormalTok{(}\StringTok{\textquotesingle{}ТОКЕН ВАШЕГО БОТА\textquotesingle{}}\NormalTok{)}

\CommentTok{\# путь к базе}
\NormalTok{db }\OtherTok{\textless{}{-}} \StringTok{"ПУСТЬ К БАЗЕ ДАННЫХ/bot.db"}

\NormalTok{start }\OtherTok{\textless{}{-}} \ControlFlowTok{function}\NormalTok{(bot, update) \{}
  
  \CommentTok{\# бот загадывает число}
\NormalTok{  num }\OtherTok{\textless{}{-}} \FunctionTok{round}\NormalTok{(}\FunctionTok{runif}\NormalTok{(}\DecValTok{1}\NormalTok{, }\DecValTok{1}\NormalTok{, }\DecValTok{50}\NormalTok{), }\DecValTok{0}\NormalTok{)}
  
  \CommentTok{\# записываем данные в базу о начале игры}
  \FunctionTok{set\_chat\_data}\NormalTok{( update}\SpecialCharTok{$}\NormalTok{message}\SpecialCharTok{$}\NormalTok{chat\_id, }\StringTok{\textquotesingle{}number\textquotesingle{}}\NormalTok{, num)}
  \FunctionTok{set\_chat\_data}\NormalTok{( update}\SpecialCharTok{$}\NormalTok{message}\SpecialCharTok{$}\NormalTok{chat\_id, }\StringTok{\textquotesingle{}attempt\textquotesingle{}}\NormalTok{, }\DecValTok{1}\NormalTok{)}
  
  \CommentTok{\# отпралвяем Reply клавиатуру}
\NormalTok{  bot}\SpecialCharTok{$}\FunctionTok{sendMessage}\NormalTok{(update}\SpecialCharTok{$}\NormalTok{message}\SpecialCharTok{$}\NormalTok{chat\_id, }
                  \StringTok{\textquotesingle{}Число загаданно, начинаем игру, ваша первая попытка.\textquotesingle{}}\NormalTok{, }
                  \StringTok{\textquotesingle{}Markdown\textquotesingle{}}\NormalTok{)}
  
\NormalTok{\}}

\NormalTok{attempt }\OtherTok{\textless{}{-}} \ControlFlowTok{function}\NormalTok{(bot, update) \{}
  
\NormalTok{  num }\OtherTok{\textless{}{-}} \FunctionTok{get\_chat\_data}\NormalTok{(update}\SpecialCharTok{$}\NormalTok{message}\SpecialCharTok{$}\NormalTok{chat\_id, }\StringTok{\textquotesingle{}number\textquotesingle{}}\NormalTok{)}
\NormalTok{  att }\OtherTok{\textless{}{-}} \FunctionTok{get\_chat\_data}\NormalTok{(update}\SpecialCharTok{$}\NormalTok{message}\SpecialCharTok{$}\NormalTok{chat\_id, }\StringTok{\textquotesingle{}attempt\textquotesingle{}}\NormalTok{)}
  
\NormalTok{  user\_num }\OtherTok{\textless{}{-}}\NormalTok{ update}\SpecialCharTok{$}\NormalTok{message}\SpecialCharTok{$}\NormalTok{text}
  
  \ControlFlowTok{if}\NormalTok{ ( user\_num }\SpecialCharTok{\textless{}}\NormalTok{ num ) \{}
    
\NormalTok{    bot}\SpecialCharTok{$}\FunctionTok{sendMessage}\NormalTok{(update}\SpecialCharTok{$}\NormalTok{message}\SpecialCharTok{$}\NormalTok{chat\_id, }
                    \FunctionTok{paste0}\NormalTok{(}\StringTok{\textquotesingle{}Номер попытки: \textquotesingle{}}\NormalTok{, att, }\StringTok{". Моё число больше"}\NormalTok{),}
                    \StringTok{\textquotesingle{}Markdown\textquotesingle{}}\NormalTok{)}
    
\NormalTok{  \} }\ControlFlowTok{else} \ControlFlowTok{if}\NormalTok{ ( user\_num }\SpecialCharTok{\textgreater{}}\NormalTok{ num ) \{}
    
\NormalTok{    bot}\SpecialCharTok{$}\FunctionTok{sendMessage}\NormalTok{(update}\SpecialCharTok{$}\NormalTok{message}\SpecialCharTok{$}\NormalTok{chat\_id, }
                    \FunctionTok{paste0}\NormalTok{(}\StringTok{\textquotesingle{}Номер попытки: \textquotesingle{}}\NormalTok{, att, }\StringTok{". Моё число меньше"}\NormalTok{),}
                    \StringTok{\textquotesingle{}Markdown\textquotesingle{}}\NormalTok{)}
    
\NormalTok{  \} }\ControlFlowTok{else}\NormalTok{ \{}
    
\NormalTok{    bot}\SpecialCharTok{$}\FunctionTok{sendMessage}\NormalTok{(update}\SpecialCharTok{$}\NormalTok{message}\SpecialCharTok{$}\NormalTok{chat\_id, }
                    \FunctionTok{paste0}\NormalTok{(}\StringTok{\textquotesingle{}Номер попытки: \textquotesingle{}}\NormalTok{, att, }\StringTok{". Поздравляю, вы угадали число!"}\NormalTok{),}
                    \StringTok{\textquotesingle{}Markdown\textquotesingle{}}\NormalTok{)}
    
    \FunctionTok{set\_chat\_data}\NormalTok{( update}\SpecialCharTok{$}\NormalTok{message}\SpecialCharTok{$}\NormalTok{chat\_id, }\StringTok{\textquotesingle{}attempt\textquotesingle{}}\NormalTok{, }\DecValTok{0}\NormalTok{)}
    
\NormalTok{  \}}
  
  \ControlFlowTok{if}\NormalTok{ ( att }\SpecialCharTok{==} \DecValTok{5} \SpecialCharTok{\&}\NormalTok{  user\_num }\SpecialCharTok{!=}\NormalTok{ num )  \{}
    
\NormalTok{    bot}\SpecialCharTok{$}\FunctionTok{sendMessage}\NormalTok{(update}\SpecialCharTok{$}\NormalTok{message}\SpecialCharTok{$}\NormalTok{chat\_id, }
                    \FunctionTok{paste0}\NormalTok{(}\StringTok{"Вы проиграли, я загадал число "}\NormalTok{, num),}
                    \StringTok{\textquotesingle{}Markdown\textquotesingle{}}\NormalTok{)}
    \FunctionTok{set\_chat\_data}\NormalTok{( update}\SpecialCharTok{$}\NormalTok{message}\SpecialCharTok{$}\NormalTok{chat\_id, }\StringTok{\textquotesingle{}attempt\textquotesingle{}}\NormalTok{, }\DecValTok{0}\NormalTok{)}
    
\NormalTok{  \}}
  
  \FunctionTok{set\_chat\_data}\NormalTok{( update}\SpecialCharTok{$}\NormalTok{message}\SpecialCharTok{$}\NormalTok{chat\_id, }\StringTok{\textquotesingle{}attempt\textquotesingle{}}\NormalTok{, att }\SpecialCharTok{+} \DecValTok{1}\NormalTok{)}
  
\NormalTok{\}}


\CommentTok{\# фильтр сообщение в состоянии ожидания имени}
\NormalTok{MessageFilters}\SpecialCharTok{$}\NormalTok{attempt }\OtherTok{\textless{}{-}} \FunctionTok{BaseFilter}\NormalTok{(}\ControlFlowTok{function}\NormalTok{(message) \{}

\NormalTok{  att }\OtherTok{\textless{}{-}} \FunctionTok{get\_chat\_data}\NormalTok{(message}\SpecialCharTok{$}\NormalTok{chat\_id, }\StringTok{\textquotesingle{}attempt\textquotesingle{}}\NormalTok{) }
  \DecValTok{0} \SpecialCharTok{\textless{}}\NormalTok{ att }\SpecialCharTok{\&}\NormalTok{ att }\SpecialCharTok{\textless{}} \DecValTok{6}
\NormalTok{\}}
\NormalTok{)}

\CommentTok{\# обработчики}
\NormalTok{h\_start   }\OtherTok{\textless{}{-}} \FunctionTok{CommandHandler}\NormalTok{(}\StringTok{\textquotesingle{}start\textquotesingle{}}\NormalTok{, start)}
\NormalTok{h\_attempt }\OtherTok{\textless{}{-}} \FunctionTok{MessageHandler}\NormalTok{(attempt, MessageFilters}\SpecialCharTok{$}\NormalTok{attempt }\SpecialCharTok{\&} \SpecialCharTok{!}\NormalTok{MessageFilters}\SpecialCharTok{$}\NormalTok{command)}

\CommentTok{\# диспетчер}
\NormalTok{updater }\OtherTok{\textless{}{-}}\NormalTok{ updater }\SpecialCharTok{+}\NormalTok{ h\_start }\SpecialCharTok{+}\NormalTok{ h\_attempt}

\CommentTok{\# запуск}
\NormalTok{updater}\SpecialCharTok{$}\FunctionTok{start\_polling}\NormalTok{()}
\end{Highlighting}
\end{Shaded}

\hypertarget{ux437ux430ux434ux430ux447ux430-5.1}{%
\section*{Задача 5.1}\label{ux437ux430ux434ux430ux447ux430-5.1}}
\addcontentsline{toc}{section}{Задача 5.1}

Возьмите \href{-updater.html\#задания-1}{задачу 2.1} из второй главы, и ограничьте использование единственного метода, доступного в созданном боте, так, что бы он работал только когда его запрашиваете вы.

\emph{Решение:}

\begin{Shaded}
\begin{Highlighting}[]
\FunctionTok{library}\NormalTok{(telegram.bot)}

\CommentTok{\# Создаём жкземпляр класса Updater}
\NormalTok{updater }\OtherTok{\textless{}{-}} \FunctionTok{Updater}\NormalTok{(}\StringTok{\textquotesingle{}ТОКЕН ВАШЕГО БОТА\textquotesingle{}}\NormalTok{)}

\CommentTok{\# Создаём функцию, которая будет суммировать переданные числа}
\NormalTok{summing }\OtherTok{\textless{}{-}} \ControlFlowTok{function}\NormalTok{(bot, update, args) \{}
  
  \ControlFlowTok{if}\NormalTok{ ( update}\SpecialCharTok{$}\NormalTok{message}\SpecialCharTok{$}\NormalTok{from}\SpecialCharTok{$}\NormalTok{username }\SpecialCharTok{==} \StringTok{\textquotesingle{}YourUsername\textquotesingle{}}\NormalTok{ ) \{}

    \CommentTok{\# Переводим полученный вектор параметров в числа и суммируем}
\NormalTok{    x }\OtherTok{\textless{}{-}} \FunctionTok{sum}\NormalTok{(}\FunctionTok{as.integer}\NormalTok{(args))}
  
    \CommentTok{\# создаём сообщение}
\NormalTok{    msg }\OtherTok{\textless{}{-}} \FunctionTok{paste0}\NormalTok{(}\StringTok{\textquotesingle{}Сумма переданных чисел: \textquotesingle{}}\NormalTok{, x)}
  
    \CommentTok{\# отправляем результат}
\NormalTok{    bot}\SpecialCharTok{$}\FunctionTok{sendMessage}\NormalTok{(update}\SpecialCharTok{$}\NormalTok{message}\SpecialCharTok{$}\NormalTok{chat\_id, msg, }\StringTok{\textquotesingle{}Markdown\textquotesingle{}}\NormalTok{)}
    
\NormalTok{  \} }\ControlFlowTok{else}\NormalTok{ \{}
    
    \CommentTok{\# отправляем результат}
\NormalTok{    bot}\SpecialCharTok{$}\FunctionTok{sendMessage}\NormalTok{(update}\SpecialCharTok{$}\NormalTok{message}\SpecialCharTok{$}\NormalTok{chat\_id, }
                    \StringTok{\textquotesingle{}У вас не достаточно прав на использование этой функции бота!\textquotesingle{}}\NormalTok{, }
                    \StringTok{\textquotesingle{}Markdown\textquotesingle{}}\NormalTok{)}
    
\NormalTok{  \}}

\NormalTok{\}}

\CommentTok{\# создаём обработчик}
\NormalTok{h\_sum }\OtherTok{\textless{}{-}} \FunctionTok{CommandHandler}\NormalTok{(}\StringTok{\textquotesingle{}sum\textquotesingle{}}\NormalTok{, summing, }\AttributeTok{pass\_args =} \ConstantTok{TRUE}\NormalTok{)}

\CommentTok{\# добавляем обработчик в диспетчер}
\NormalTok{updater }\OtherTok{\textless{}{-}}\NormalTok{ updater }\SpecialCharTok{+}\NormalTok{ h\_sum}

\CommentTok{\# запускаем бота}
\NormalTok{updater}\SpecialCharTok{$}\FunctionTok{start\_polling}\NormalTok{()}
\end{Highlighting}
\end{Shaded}


  \bibliography{book.bib,packages.bib}

\end{document}
